\documentclass[a4paper, 9pt]{extarticle}

\usepackage[notes]{style}

\newcommand{\realtitle}{Session 22 - Trees}

\newcommand{\separator}{\begin{center}%
\noindent\makebox[\linewidth]{\rule{0.75\paperwidth}{0.4pt}}%
\end{center}}

\begin{document}

\makebox[\linewidth]{\rule{\textwidth}{0.4pt}}
UC3M \hfill Alberto Cortés Martín\\
Systems Programming, 2014-2015 \hfill version: \today\\
\makebox[\linewidth]{\rule{\textwidth}{0.4pt}}
\begin{center}
  \Large{\realtitle}\\Lecture Notes
\end{center}
\makebox[\linewidth]{\rule{\textwidth}{0.4pt}}
\vspace{1cm}


\section{Today's Topics}
\begin{blackboard}
Trees
\end{blackboard}

\section{Examples of trees}

Do you remember the recursive and horrible solution to the Fibonacci problem?

\begin{center}
\includegraphics[width=6cm]{./img/fibonacci/fibonacci.eps}
\end{center}

This kind of structure is called a \emph{tree}.

In conputer science trees grow downwards, the node at the top is called the
\emph{root}, and the nodes at the buttom are called the \emph{leaves}.

\separator

Do you remember the object inheritance hierarchy in Java?

\begin{center}
\includegraphics[width=6cm]{./img/object/object.eps}
\end{center}

Well, this is also a tree.

\separator
\newpage

Let us take a look at the directories I have under my music directory:

\begin{center}
\includegraphics[width=8cm]{./img/music/music.eps}
\end{center}

This might not look as a tree, but if you look closely, you will see it is, in
fact, a tree, with the music directory as the root, then, on the first level
the name of the bands, and on the second level the name of the albums.

Problem: Write a graphical Java program that allows me to browse my music collection
by bands and albums and let me play songs by clicking on them.

\separator

Now, let us take a look at this map of flight routes through Europe:

\begin{center}
\includegraphics[width=5cm]{./img/map/map.eps}
\end{center}

Problem: Write a Java program that finds the flight with least connections from Madrid to Rome.

You can create the tree of flight from Madrid to Rome, and look for the path
with least connections. This is, the path with least depth.

\begin{center}
\includegraphics[width=12cm]{./img/madrid-rome/madrid-rome.eps}
\end{center}

\separator

The Knapsack problem.

You are a burglar, stealing some goods in a house.

You carry a bag that can hold up to 95 Kilograms of goods.

\newpage

In the house there are 4 objects, each of them with its own weight and price:

\begin{blackboard}
      w  p
  A  50 60
  B  20 30
  C  30 66
  D  40 40
\end{blackboard}

Write a Java program that calculates the best way to fill your bag.

\begin{center}
\includegraphics[width=13cm]{./img/knapsack/knapsack.eps}
\end{center}

A naive algorithm to solve this problem might look as follows:

\begin{blackboard}
  - build the tree with all the combinations
  - search for the leaf with the biggest price
\end{blackboard}

Now, think of Amazon, sending boxes to Madrid from Paris in a plane, trying to
fill the boxes the best way possible.

Of course Amazon will have thousands of boxes and many 20 flight connections or so.


\separator

All these are toy examples of real programming problems that happens every day in

\begin{itemize}
  \item Computer graphics
  \item Compresion of video, images and music
  \item Optimization problems (in airports, filesystems, companies)
  \item Marketing
  \item Storehouses
  \item Economics
  \item Sequencing the human genome
  \item Internet communications
  \item ...
\end{itemize}

As you can see, tree are pretty useful data structures.


\section{Definition}

Until now, you have been using \emph{linear} data structures, like lists, queues, stacks and decks.

In a linear data structure each element has a clear predecessor and a clear succesor.

Trees are non-linear data structures that store elements in a hierarchy.

Each element (except the root) has a parent.

Each element (except the leafs) has one or more children.

SLIDES FROM 5 to 14.



\section{Terminology}

SLIDES FROM 15 to 18.

Solution to slide 17:

\begin{center}
  \begin{tabular}{|c|c|c|c|c|}
    \hline
    Node & height & depth & size & int. / ext. \\
    \hline
    a & 3 & 0 & 11 & Internal \\
    b & 1 & 1 & 3  & Internal \\
    c & 0 & 1 & 1  & External \\
    \hline
    d & 1 & 1 & 2 & Internal \\
    e & 2 & 1 & 4 & Internal \\
    f & 0 & 2 & 1 & External \\
    \hline
    g & 0 & 2 & 1 & External \\
    h & 0 & 2 & 1 & External \\
    i & 0 & 2 & 1 & External \\
    \hline
    j & 1 & 2 & 2 & Internal \\
    k & 0 & 3 & 1 & External \\
    \hline
  \end{tabular}
\end{center}



\section{Implementation}


\subsection{Complicated Way}

\begin{blackboard}
class TNode<E> {
    E datum;
    TNode<E> parent;
    LinkedQueue<TNode<E>> children;
    ...
}
\end{blackboard}

\includegraphics[width=4cm]{./img/tree1/tree1a.eps}
\includegraphics[width=4cm]{./img/tree1/tree1b.eps}
\includegraphics[width=4cm]{./img/tree1/tree1c.eps}
\includegraphics[width=4cm]{./img/tree1/tree1d.eps}

\begin{center}
\includegraphics[width=12cm]{./img/tree1/tree1d.eps}
\end{center}



\subsection{Simple Way}

\begin{blackboard}
class TNode<E> {
    E datum;
    TNode<E> parent;
    TNode<E> next;
    TNode<E> child;

    ...
}
\end{blackboard}

\includegraphics[width=4cm]{./img/tree2/tree2a.eps}
\includegraphics[width=4cm]{./img/tree2/tree2b.eps}
\includegraphics[width=4cm]{./img/tree2/tree2c.eps}
\includegraphics[width=4cm]{./img/tree2/tree2d.eps}

\begin{center}
\includegraphics[width=12cm]{./img/tree2/tree2d.eps}
\end{center}











\newpage


\section{Binary Trees}

A \emph{binary} tree is an ordered tree, where all its nodes have $<3$ children
(this is, nodes can have 0, 1, or 2 children).

A non-empty binary tree is \emph{complete} when
\begin{itemize}
\item all its levels except posibly the last one are full,
\item and the last level has all its nodes on the left side.
\end{itemize}

A non-empty binary tree is \emph{full} when each node is either a leaf or has
exactly 2 children.


\includegraphics[width=4cm]{./img/btrees/btreesa.eps}
\includegraphics[width=4cm]{./img/btrees/btreesb.eps}
\includegraphics[width=4cm]{./img/btrees/btreesc.eps}
\includegraphics[width=4cm]{./img/btrees/btreesd.eps}

\begin{blackboard}
Non-binary               binary                   binary                   binary
                         non-complete             complete                 complete
                         non-full                 full                     non-full
\end{blackboard}

\includegraphics[width=4cm]{./img/btrees/btreese.eps}
\includegraphics[width=4cm]{./img/btrees/btreesf.eps}
\includegraphics[width=4cm]{./img/btrees/btreesg.eps}
\includegraphics[width=4cm]{./img/btrees/btreesh.eps}

\begin{blackboard}
binary                   binary                   binary                   binary
non-complete             complete                 complete                 complete
non-full                 full                     non-full                 full
\end{blackboard}

\newpage
\subsection{Properties}

For binary trees, let
\begin{itemize}

\item $e \in \mathbb{N}$ be the total number of external nodes (i.e. leafs),

\item $i \in \mathbb{N}$ be the total number of internal nodes,

\item $n \in \mathbb{N}$ be the total number of nodes ($n = e + i$),

\item $n'' \in \mathbb{N}$ be the total number of nodes with two children and

\item $h \in \mathbb{N}$ be the height of a non-empty tree.

\end{itemize}

\separator

Then, for \textbf{generic} (non-empty) binary trees, the number of nodes with
depth $d$ is at least 1 and at most $2^d$, therefore

\begin{equation}
  h + 1 \le n \le 2^{h+1} - 1,
\end{equation}
because
\begin{equation}
  \sum_{i=0}^{h}2^i = 2^{h+1} - 1.
\end{equation}

Also for generic (non-empty) binary trees, as each node with two children opens
a new path, \begin{equation} e = n'' + 1.  \end{equation}

\separator

For \textbf{complete} binary trees
\begin{equation}
  i = \left \lfloor{\frac{n}{2}}\right \rfloor,
\end{equation}
as each pair of consecutive leafs have the same parent.

Also, a complete binary tree has the smallest possible height,
this is, complete binary trees are ``compact''
(they store information as a tree of minimum height):
\begin{equation}
  h = \left \lfloor{\log_2 n}\right \rfloor,
\end{equation}

\separator

For \textbf{full} binary trees
\begin{equation}
  2^h + 1 \le n \le 2^{h+1} - 1,
\end{equation}
as the depest level will have at least 2 leafs.

\newpage

\subsection{Implementations}



\end{document}
