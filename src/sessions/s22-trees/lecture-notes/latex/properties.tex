\subsection{Properties}

For binary trees, let
\begin{itemize}

\item $e \in \mathbb{N}$ be the total number of external nodes (i.e. leafs),

\item $i \in \mathbb{N}$ be the total number of internal nodes,

\item $n \in \mathbb{N}$ be the total number of nodes ($n = e + i$),

\item $n_2 \in \mathbb{N}$ be the total number of nodes with two children and

\item $h \in \mathbb{N}$ be the height of a non-empty tree.

\end{itemize}

Then, for \textbf{generic} (non-empty) binary trees, the number of nodes with
depth $d$ is at least 1 and at most $2^d$, therefore

\begin{equation}
  h + 1 \le n \le 2^{h+1} - 1,
\end{equation}
because
\begin{equation}
  \sum_{i=0}^{h}2^i = 2^{h+1} - 1.
\end{equation}

Also for generic (non-empty) binary trees, as each node with two children opens
a new path, \begin{equation} e = n_2 + 1.  \end{equation}

For \textbf{complete} binary trees
\begin{equation}
  i = \left \lfloor{\frac{n}{2}}\right \rfloor,
\end{equation}
as each pair of consecutive leafs have the same parent.

For \textbf{full} binary trees
\begin{equation}
  2^h + 1 \le n \le 2^{h+1} - 1,
\end{equation}
as the depest level will have at least 2 leafs.
