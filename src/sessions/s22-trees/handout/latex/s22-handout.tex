\documentclass[a4paper, 10pt]{article}

\usepackage[handout]{style}

\newcommand{\realtitle}{Session 22 - Trees}

\newcommand{\separator}{\begin{center}%
\noindent\makebox[\linewidth]{\rule{0.75\paperwidth}{0.4pt}}%
\end{center}}

\begin{document}

\footnotesize{
\makebox[\linewidth]{\rule{\textwidth}{0.4pt}}
UC3M \hfill Alberto Cortés Martín\\
Systems Programming, 2014-2015 \hfill version: \today\\
\makebox[\linewidth]{\rule{\textwidth}{0.4pt}}
\begin{center}
  \small{\realtitle}\\Handout
\end{center}
\makebox[\linewidth]{\rule{\textwidth}{0.4pt}}
}
\vspace{0.2cm}

\begin{multicols}{2}
\begin{blackboard}
class TNode<E> {
    E datum;
    TNode<E> parent;
    LinkedQueue<TNode<E>> children;
}
\end{blackboard}
\includegraphics[width=8cm]{../../lecture-notes/latex/img/tree1/tree1e.eps}
\columnbreak
\begin{blackboard}
class TNode<E> {
    E datum;
    TNode<E> parent;
    TNode<E> next;
    TNode<E> child;
}
\end{blackboard}
\includegraphics[width=8cm]{../../lecture-notes/latex/img/tree2/tree2e.eps}
\end{multicols}


A \emph{binary} tree is an \underline{ordered} tree, where all its nodes have $<3$ children
(this is, nodes can have 0, 1, or 2 children).

A non-empty binary tree is \emph{complete} when
\begin{itemize}
\item all its levels except possibly the last one are full,
\item and the last level has all its nodes on the left side.
\end{itemize}

A non-empty binary tree is \emph{full} when each node is either a leaf or has
exactly 2 children.


\includegraphics[width=4cm]{../../lecture-notes/latex/img/btrees/btreesa.eps}
\includegraphics[width=4cm]{../../lecture-notes/latex/img/btrees/btreesb.eps}
\includegraphics[width=4cm]{../../lecture-notes/latex/img/btrees/btreesd.eps}
\includegraphics[width=4cm]{../../lecture-notes/latex/img/btrees/btreese.eps}

\includegraphics[width=4cm]{../../lecture-notes/latex/img/btrees/btreesf.eps}
\includegraphics[width=4cm]{../../lecture-notes/latex/img/btrees/btreesg.eps}
\includegraphics[width=4cm]{../../lecture-notes/latex/img/btrees/btreesh.eps}
\includegraphics[width=4cm]{../../lecture-notes/latex/img/btrees/btreesc.eps}

\newpage

\begin{minipage}{0.3\linewidth}
For binary trees, let
\begin{itemize}

\item $e \in \mathbb{N}$ be the total number of external nodes (i.e. leafs),

\item $i \in \mathbb{N}$ be the total number of internal nodes,

\item $n \in \mathbb{N}$ be the total number of nodes ($n = e + i$),

\item $n_d \in \mathbb{N}$ be the total number of nodes of depth $d$,

\item $n'' \in \mathbb{N}$ be the total number of nodes with two children and

\item $h \in \mathbb{N}$ be the height of a non-empty tree.

\end{itemize}
Then, for \textbf{generic} (non-empty) binary trees:

\begin{equation}
  1 \le n_d \le 2^d
\end{equation}
\begin{equation}
  h + 1 \le n \le 2^{h+1} - 1
\end{equation}
\begin{equation}
  \sum_{j=0}^{h}2^j = 2^{h+1} - 1
\end{equation}
\begin{equation} e = n'' + 1.  \end{equation}

For \textbf{complete} binary trees:
\begin{equation}
  i = \left \lfloor{\frac{n}{2}}\right \rfloor
\end{equation}
\begin{equation}
  h = \left \lfloor{\log_2 n}\right \rfloor
\end{equation}

For \textbf{full} binary trees:
\begin{equation}
  2^h + 1 \le n \le 2^{h+1} - 1
\end{equation}
\begin{equation}
  i = n''
\end{equation}

\end{minipage}
\begin{minipage}{0.6\linewidth}
  \begin{center}
\includegraphics[width=11cm]{../../lecture-notes/latex/img/linkedbnode/linkedbnode6.eps}
\begin{multicols}{2}
\begin{blackboard}
      PRE_ORDER

// LinkedBNode
void printPreOrder() {
  s.o.p(datum);
  if (left != null) {
    left.printPreOrder();
  }
  if (right != null) {
    right.printPreOrder();
  }
}
\end{blackboard}
\columnbreak
\includegraphics[width=5cm]{../../lecture-notes/latex/img/emptytree/emptytree.eps}
\end{multicols}
\begin{multicols}{2}
\begin{blackboard}
      IN_ORDER

// LinkedBNode
void printPreOrder() {
  if (left != null) {
    left.printPreOrder();
  }
  s.o.p(datum);
  if (right != null) {
    right.printPreOrder();
  }
}
\end{blackboard}
\columnbreak
\includegraphics[width=5cm]{../../lecture-notes/latex/img/emptytree/emptytree.eps}
\end{multicols}
\end{center}
\end{minipage}


\begin{minipage}{0.48\linewidth}
\begin{blackboard}
      PRE_ORDER ITERATIVE

// LinkedBTree
void printPreOrder() {
  if (isEmpty()) {
    return;
  }
  Stack<LinkedBNode<E>> stack =
                   new Stack<LinkedBNode<E>>();
  stack.push(root);
  root.printPreOrder(stack);
}

// LinkedBNode
void printPreOrder(Stack<LinkedBNode<E> stack) {
  while (! stack.isEmpty()) {
    LinkedBNode<E> current = stack.pop();
    System.out.println(current.datum);
    if (right != null) {
      stack.push(current.right);
    }
    if (left != null) {
      stack.push(current.left);
    }
  }
}
\end{blackboard}
\end{minipage}
\begin{minipage}{0.48\linewidth}
\begin{blackboard}
      LEVEL_ORDER ITERATIVE

// LinkedBTree
void printLevelOrder() {
  if (isEmpty()) {
    return;
  }
  Queue<LinkedBNode<E>> queue =
                   new Queue<LinkedBNode<E>>();
  queue.enqueue(root);
  root.printLevelOrder(queue);
}

// LinkedBNode
void printLevelOrder(Queue<LinkedBNode<E>> queue) {
  while (! queue.isEmpty()) {
    LinkedBNode<E> current = queue.dequeue();
    System.out.println(current.datum);
    if (left != null) {
      queue.enqueue(current.left);
    }
    if (right != null) {
      queue.enqueue(current.right);
    }
  }
}
\end{blackboard}
\end{minipage}
\end{document}
