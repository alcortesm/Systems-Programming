\documentclass[a4paper, 11pt]{article}

\usepackage[assessment]{style}

\newcommand{\realtitle}{Session 22 - Trees}

\begin{document}

\makebox[\linewidth]{\rule{\textwidth}{0.4pt}}
UC3M \hfill Alberto Cortés Martín\\
Systems Programming, 2014-2015 \hfill version: \today\\
\makebox[\linewidth]{\rule{\textwidth}{0.4pt}}
\begin{center}
  \Large{\realtitle}\\Self-Assessment
\end{center}
\makebox[\linewidth]{\rule{\textwidth}{0.4pt}}
\vspace{1cm}

\section{Binary trees}

\subsection{}

Implement a binary tree using the source code provided in your lecture as a base.

Delete the body of all the non-static methods provided in the source code, and re-write them yourself.

You can use the provided main method to test your code.

Then download again the source code and compare your solution with the official one.

\subsection{}

Write a method \texttt{isComplete} that returns if a binary tree is complete.

\subsection{}

Write a method \texttt{isFull} that returns if a binary tree is full.

\subsection{}

Write a method \texttt{contains} that returns if the element passed as an argument is found in the tree.

\subsection{}

Write a \texttt{path} method that returns a \emph{list} of elements in the path
from the root to the element passed as an argument if it exists in the tree. Otherwise it will return an empty list.

\subsection{}

Implement a \texttt{duplicate} method for your binary tree that returns an
exact copy of a binary tree, with exact duplicates of all its nodes.

\subsection{}

Now, write the above method as a constructor for your tree class.

\subsection{}

Implement a \texttt{mirror} method for your binary tree that returns a mirrowed
copy of your tree as shown in the figure:

\begin{center}
  \includegraphics[width=8cm]{./img/mirror/mirror.eps}
\end{center}

\subsection{}

Implement an method \texttt{toStringLevelOrder} that
returns a string representation of a tree in level order.

\textsl{Tip 1: a recursive version of this method is pretty difficult to code}.

\textsl{Tip 2: a queue might be usefull to solve this problem}.


\subsection{}

Implement an iterative version of the \texttt{toStringPreOrder}.

\textsl{Tip: an stack might be usefull to solve this problem}.



\subsection{}

Write a program that builds the tree for this expresion: \verb;((1 + 2) / (3 - 4)) * (2 ^ 3);,
and prints the result by calling an \texttt{evaluate} method on
the root of the tree.

\textsl{Tip 1: watch this \href{https://www.youtube.com/watch?v=4F72VULWFvc}{video}}: \verb;https://www.youtube.com/watch?v=4F72VULWFvc; from 00:00 to 17:00.


\subsection{}

Expand your previous expresion tree so it supports the factorial operator.

Test it with the following expresion: \verb;(1 + 2)! * (4 / 2!);


\subsection{}

Add a \texttt{toString} method to your tree from the previous exercise, so that it prints the expresion in the tree as we usually write them in math.

\textsl{Tip: be careful with the parenthesis}.


\subsection{}

Write a program to evaluate mathematical expresions written in postfix notation, \\ for example: 1 2 + ! 4 2 ! / *

\textsl{Tip: you do not need a binary tree to solve this problem, an stack will be enoguh}.






\section{N-ary trees}

\subsection{}

Implement an N-ary tree, and try to build the tree from slide 17 in a main
method. Then print the size, depth, height of each node and if the nodes are
internal or external, so you can have a solution to the table in that same slide.

\textsl{Tip: write a recursive method for your tree that prints, using a
level-order traversal, the infromation of each node (size, height\dots).}

\subsection{}

Write a program that builds the following tree and print the path with the
smallest depth, this is, the fastest way to get from Madrid to Rome.

\begin{center}
  \includegraphics[width=8cm]{../../lecture-notes/latex/img/btrees/btreesa.eps}
\end{center}

\subsection{}

Write a program that prints the list of rooms you will have to visit to get from the entrance of the following maze to the exit.

\begin{center}
  \includegraphics[width=8cm]{./img/maze/maze.eps}
\end{center}


\subsection{}

What would happend in the previous exercise if room E and F were connected as shown in the figure below?

\begin{center}
  \includegraphics[width=8cm]{./img/maze2/maze.eps}
\end{center}

\subsection{}

Write a Java program to solve the following knapstack problem using a generic N-ary tree:

\begin{blackboard}
Item  Weight  Price
A     50      60
B     20      30
C     30      66
D     40      40
\end{blackboard}

Consider that the maximum weight allowed is 95.

\end{document}
