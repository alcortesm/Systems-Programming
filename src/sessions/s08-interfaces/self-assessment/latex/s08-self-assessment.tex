\documentclass[a4paper, 11pt]{article}

\usepackage[assessment]{style}

\newcommand{\realtitle}{Session 08 - Interfaces}

\begin{document}

\makebox[\linewidth]{\rule{\textwidth}{0.4pt}}
UC3M \hfill Alberto Cortés Martín\\
Systems Programming, 2014-2015 \hfill version: \today\\
\makebox[\linewidth]{\rule{\textwidth}{0.4pt}}
\begin{center}
  \Large{\realtitle}\\Self-Assessment
\end{center}
\makebox[\linewidth]{\rule{\textwidth}{0.4pt}}
\vspace{1cm}


\section{Software for a Clothing Store}

\subsection{}

Write a \texttt{Warehouse} class that represents the warehouse of a clothing
store.

The class must have an empty constructor, an \texttt{add} method to store
garments in the warehouse and a \texttt{toString} method that returns a string
representation of all the stored garments and their total price.

Test your classes with the following program:

\codeinput{../java/WarehouseTest.java}{WarehouseTest.java}

The output from this program must be exactly like this:

\begin{cmd}
$ java WarehouseTest
Jacket(color=java.awt.Color[r=255,g=0,b=0], price=10)
Jacket(color=java.awt.Color[r=0,g=255,b=0], price=10)
Pants(color=java.awt.Color[r=0,g=0,b=0], size=small, price=7)
Pants(color=java.awt.Color[r=0,g=0,b=0], size=big, price=14)
Total price = 41
\end{cmd}

Additional details:

\begin{itemize}

  \item The shop only works with 2 types of garments: jackets and pants.

  \item Jackets come in many colors and their price is always 10 Euros.

  \item Pants come only in black, and have two sizes (small and big).

    \begin{itemize}

      \item The constructor of this class receives a boolean to know if the new
        pants must be small or not.

      \item The price of the small pants is 7 Euros, the price for the big
        pants is the double of the small ones.

    \end{itemize}

\end{itemize}

\solutioninput{../java/Garment.java}
\solutioninput{../java/Jacket.java}
\solutioninput{../java/Pants.java}
\solutioninput{../java/Warehouse.java}


\subsection{}

Write an interface called \texttt{Reversible} that allows classes to have a
\texttt{void reverse()} method.

\solutioninput{../java/Reversible.java}

\subsection{}

Write a class called \texttt{ReversibleJacket} that represents a reversible
Jacket.

A reversible jacket will have two colors, the ordinary one on the outside and
another one on the inside.

Calling the \texttt{reverse} method on a reversible jacket will interchange
those colors.

\begin{solution}
TODO: write the solution to this exercise.
\end{solution}

\subsection{}

Write a class called \texttt{ReversiblePoint} that represents a point that can
be reversed.

When you reverse a point, you are swapping its x and y coordinates.

\begin{solution}
TODO: write the solution to this exercise.
\end{solution}

\subsection{}

Try to write a class called \texttt{ReversibleString} that represents a Java
\texttt{String} that can be reversed.

When you reverse an string, you will get a string with the caracters in reverse
order, for example:

\begin{blackboard}
ReversibleString s = new ReversibleString("hello");
s.reverse(); // "olleh"
\end{blackboard}

This problem is impossible to solve, explain why.

\begin{solution}
String is a final class.
\end{solution}


\section{Comparable Things}

\subsection{}

Write a class \texttt{ComparablePoint}, that represent \texttt{Point}s that can
be compared.

A point $\alpha$ will be bigger than a point $\beta$ if the distance between
$\alpha$ and the origin is bigger than the distance between $\beta$ and the
origin.

\begin{solution}
TODO: write the solution to this exercise.
\end{solution}


\subsection{}

Write a class \texttt{ComparableReversiblePoint}, that represent
\texttt{Point}s that can be compared and reversed.

Test your class with a program that checks that a point is as big as its reverse.

\begin{solution}
TODO: write the solution to this exercise.
\end{solution}


\subsection{}

Modify your answer to the first exercise about the \texttt{Warehouse} so now
garments are comparable, based on their price.

The \texttt{toString} method of \texttt{Warehouse} must now print the garments
sorted from the cheapest one to the most expensive one.

\begin{solution}
TODO: write the solution to this exercise.
\end{solution}

\end{document}
