\documentclass[a4paper, 9pt]{extarticle}

\usepackage[notes]{style}

\newcommand{\realtitle}{Session 08 - Interfaces}

\begin{document}

\makebox[\linewidth]{\rule{\textwidth}{0.4pt}}
UC3M \hfill Alberto Cortés Martín\\
Systems Programming, 2014-2015 \hfill version: \today\\
\makebox[\linewidth]{\rule{\textwidth}{0.4pt}}
\begin{center}
  \Large{\realtitle}\\Lecture Notes
\end{center}
\makebox[\linewidth]{\rule{\textwidth}{0.4pt}}
\vspace{1cm}


\section{Today's Topics}
\begin{blackboard}
Abstract Classes
Interfaces
--
Access Modifires Table
Packages
Object.equals(Object)
\end{blackboard}










\section{Abstract Classes}

%The sole purpose of an Abstract Class is to be extended by other classes.

%This means, you cannot create objects of abstract classes.

%You will have to create a derived class, and then construct objects of that
%derived class.

Allow me to exaplain what an abstract class is, using an example:

Do you remember the ``biggest'' method from our last class?

It takes an array of rectangles and returns the biggest rectangle in the array.

\begin{blackboard}
  public static Rectangle biggest(Rectangle[] inputs) {
    ...
  }

  public static Rectangle biggest(ArrayList<Rectangle> inputs) {
    ...
  }
\end{blackboard}

MOSTRAR \verb+uml_square.eps+
\begin{center}
  \includegraphics[width=5cm]{./img/uml_square/uml_square.eps}
\end{center}

But what if we want to insert Triangles and Circles also.

Well, you already know how to solve that:

You can create a Shape class and make Rectangles, Circles... extend from it:

MOSTRAR \verb+uml_shape_1.eps+
\begin{center}
  \includegraphics[width=10cm]{./img/uml_shape_1/uml_shape_1.eps}
\end{center}

We will have to change the method definition form Rectangles to Shapes of
course:

\codeinput{../java/v1/ShapeTest.java}{v1/ShapeTest.java}

And Circles, Triangles and Rectangles have to extend Shape.

We don't have to extend Square though, as it is already extending Shape from Rectangle.

But there is a problem here, can you see it?

The \texttt{Shape} class has no \texttt{area()} method.

Even if know that all Circles, Rectangles... have an area, Java only see a
Shape, and a Shape has no area method.

OK, no problem, let's fix it:

MOSTRAR \verb+uml_shape_2.eps+
\begin{center}
  \includegraphics[width=10cm]{./img/uml_shape_2/uml_shape_2.eps}
\end{center}

Now, this works.

DEMO: \verb+cd v1 ; java ShapeTest+

Let's see how to write the Shape class:

\begin{blackboard}
class Shape {
  // no attributes

  // no need to declare a ctor,
  // as it has no attributes
  // the default ctor will do.

  public double area() {
    // now, what?
  }
}
\end{blackboard}

What is the area of a shape?

We are not sure, we know whats the area of a circle, and of a rectangle, but of a shape?

We will not know until we instantiate a more concrete shape (circle, rectangle...).

So we cannot implement this method.

This is called an abstract method, because it is not implemented, it has no
code and you can not really execute it, because Java will not know what to do.


Let's say we create a new Shape object, what will happend if we call the area method over it:

\begin{blackboard}
  Shape s = new Shape();
  System.out.println(s.area());
\end{blackboard}

Java cannot do this, that is way this class is an abstract class, you are not
allowed to instantiate it.

But you are allowed to subclass it, and implemente the area method on all their subclasses, so you can write code like this:

\begin{blackboard}
  Shape s = new Rectangle(2D, 3D);
  System.out.println(s.area());
\end{blackboard}

\codeinput{../java/v1/Shape.java}{v1/Shape.java}

\codeinput{../java/v1/ShapeTest.java}{v1/ShapeTest.java}

This is the simplest example about abstract classes you can write.

Let's complicate things a little bit:

In the future, we would like to draw all this shapes in the screen, so we need
to improve our shapes a little by adding to them a center (point) and a color.

MOSTRAR \verb+uml_graphicalshape.eps+
\begin{center}
  \includegraphics[width=10cm]{./img/uml_graphicalshape/uml_graphicalshape.eps}
\end{center}

\begin{itemize}

  \item Do GraphicalShape extends Shape? YES.

  \item Do you think GraphicalShape is an abstract class? YES. IT HAS AN ABSTRACT METHOD (area)

  \item Can GraphicalShape has constructors? YES, IT HAS ATTRIBUTES.

  \item Can you instantiate it? NO, IT IS ABSTRACT.

  \item Why it has ctors then? SUBCLASSES WILL CALL THEM USING THE SUPER CTOR.

\end{itemize}

\begin{blackboard}
MOSTRAR CÓDIGO:
  v2/GraphicalShape.java
  v2/Rectangle.java
\end{blackboard}


\subsection{Abstract rules}

MOSTRAR \verb+abstract_rules.eps+
\begin{center}
  \includegraphics[width=10cm]{./img/abstract_rules/abstract_rules.eps}
\end{center}

The point of abstract classes is to define a user interface to their subclasses.

Sometimes and abstract class is also called an ADT (Abstract Data Type): it
defines HOW to use some classes, but not HOW they are implemented.

EXAMPLE: You can write the biggest method just by knowing the Shape ADT, you
don't need to know about Rectangles or Circles.

You can even create more shapes later, and all of them will work just fine with
your method.




\subsection{More examples of Abstract Classes}

MOSTRAR \verb+number.eps+
\begin{center}
  \includegraphics[width=10cm]{./img/number/number.eps}
\end{center}

MOSTRAR \verb+abstractlist.eps+
\begin{center}
  \includegraphics[width=10cm]{./img/abstractlist/abstractlist.eps}
\end{center}


\section{Java Interfaces}

An interface is the public methods and attributes of a class: this is, how to
use that class.

A Java Interface is something different, altough related:

A Java Interface is a type of abstract class, with the following rules:

MOSTRAR \verb+interfaces_rules.eps+
\begin{center}
  \includegraphics[width=10cm]{./img/interfaces_rules/interfaces_rules.eps}
\end{center}

EXAMPLE:

MOSTRAR \verb+drawable.eps+
\begin{center}
  \includegraphics[width=14cm]{./img/drawable/drawable.eps}
\end{center}

\begin{blackboard}
// HasColor.java
import java.awt.Color;

interface HasColor {
  setColor(Color color);
  Color getColor();
}

// HasCenter.java
interface HasCenter {
  setCenter(Point center);
  Point getCenter(center);
}

// Drawable.java
interface Drawable extends HasCenter, HasColor {
  draw();
}

// Shape.java
abstract class Shape implements Drawable {
  ...
}
\end{blackboard}


\subsection{Extends and Implements}

MOSTRAR \verb+extends_implements.eps+
\begin{center}
  \includegraphics[width=10cm]{./img/extends_implements/extends_implements.eps}
\end{center}

\subsection{Examples of Java Interfaces}

MOSTRAR \verb+important_interfaces.eps+
\begin{center}
  \includegraphics[width=10cm]{./img/important_interfaces/important_interfaces.eps}
\end{center}

MOSTRAR \verb+abstractlist_interface.eps+
\begin{center}
  \includegraphics[width=10cm]{./img/abstractlist_interface/abstractlist_interface.eps}
\end{center}















\section{Access Modifiers}

NO SE SI DARÁ TIEMPO A ESTO.

\begin{center}
  \begin{tabular}{|p{3cm}|p{4cm}|p{4cm}|p{4cm}|}
\hline
\multicolumn{1}{|c|}{access mod.} & \multicolumn{1}{|c|}{class} & \multicolumn{1}{|c|}{method} & \multicolumn{1}{|c|}{attribute} \\
\hline
\texttt{public}    & \multicolumn{3}{|c|}{Accesible to all classes.} \\
\hline
\textsl{(empty) AKA friendly AKA package-protected} & \multicolumn{3}{|c|}{Accesible only to classes in the same package.} \\
\hline
\texttt{protected}   & Only for inner classes. & \multicolumn{2}{|c|}{Acecsible only to subclasses.} \\
\hline
\texttt{private}     & Only for inner classes. & \multicolumn{2}{|c|}{Accessible only inside the class.} \\
\hline
\hline
\texttt{final}       & Cannot be subclassed & Cannot be overriden & Cannot be modified once initialized \\
\hline
\texttt{static}      & Turns an inner class into a statically nested class. & Class method (instead of an object method) & Class attribute (instead of an object attribute) \\
\hline
\texttt{abstract}    & Cannot be instantiated & Has no code & --- \\
\hline
\end{tabular}
\end{center}
















\section{Packages}

ESTO YA SE HA EXPLICADO EN LA SESIóN 7 DE LAB.

A package is a collection of classes, and the package is part of the full
qualified name of the class.

For example:

There is a class in Java to represent colors, it is called
\texttt{java.awt.Color}.

\begin{blackboard}
  main:
    java.awt.Color red = new java.awt.Color(1.0, 0, 0);
\end{blackboard}

The java.awt.Color class is called like that because it is defined as follows:

\begin{blackboard}
package java.awt;

class Color {
...
}
\end{blackboard}

So, this is a Java class called \texttt{Color} that belongs to the
\texttt{java.awt} package. The \textsl{full qualified name} of the class is
\texttt{java.awt.Color}.

This menas you can have many Color classes in your programs, as long as they
not in the same package:

\begin{blackboard}
java.awt.Color // the official java Color class
alcortes.Color // my own color class
uc3m.web.Color // the UC3M color class for creating the UC3M official website
...
\end{blackboard}

Compiling classes in packages can be tricky, you have an explanation on how to
do it on the bibliographic references.





\subsection{Imports}

The \texttt{import} keyword is a trick to avoid writting long Java class names:

Instead of writting this:

\codeinput{../java/ColorTest.java}{ColorTest.java}

You can write this:

\begin{blackboard}
              |
              V
import java.awt.Color;
class ColorTest {
    public static void main(String[] args) {
        Color c1 = new Color();
        alcortes.Color c2 = new alcortes.Color();
    }
}
              |
              V
import alcortes.Color;
class ColorTest {
    public static void main(String[] args) {
        java.awt.Color c1 = new java.awt.Color();
        Color c2 = new Color();
    }
}
              |
              V
import alcortes.Color;
import java.awt.Color;
// this don't work!
\end{blackboard}

So, \texttt{import} don't actually imports anything to your program, it just
let you use short class names instead of long names.

Java automatically imports java.lang.* (all classes in the java.lang package)
so you can write:

\begin{blackboard}
String instead of java.lang.String
Object instead of java.lang.Object
...
\end{blackboard}







\subsection{Default package}

If you do not define a package for your classes, they will go into the
"Default" package, that has no name, therefore:

\begin{blackboard}
class Rectangle {
  ...
}
\end{blackboard}

is a class called Rectangle in the default package, and its short name and full
qualified name are both "Rectangle".

But:

\begin{blackboard}
package alcortes;

class Rectangle {
...
}
\end{blackboard}

would be a DIFFERENT java class, called "alcortes.Rectangle".




















\section{\texttt{equals} and \texttt{instanceof}}

NO SE SI DARÁ TIEMPO A ESTO.


You already know the \texttt{toString} method.

It is a very important method because as it is defined in \texttt{Object}, all
classes inherits it.

It is also very important because \texttt{println} call that method to get the
string representation of objects.

This means you can print objects in a meaningfull way by overriding their
\texttt{toString} method.

\texttt{equals} is also a very important method for similar reasons.

It is also defined in the \texttt{Object} class and can be used to compare
objects.

Here is its prototype and implementation:

\begin{blackboard}
  public boolean equals(Object o) {
    return (this == o);
  }

  Object a = new Object();
  Object b = new Object();
  Object c = a;

  System.out.println(a == b);      // false
  System.out.println(a.equals(b)); // false               DIBUJAR MEMORIA

  System.out.println(a == c);      // true
  System.out.println(a.equals(c)); // true
\end{blackboard}

So the \texttt{Object} version of this method just return if two references are
pointing to the same memory location.

If you want to compare objects using this method you should override it,
for example, you already know that \texttt{String}s override it to compare the
characters inside the string.

Let's see how to do that for a Rectangle, for example:

\begin{center}
  \includegraphics[width=5cm]{./img/uml_square/uml_square.eps}
\end{center}

\begin{blackboard}
class Rectangle {
  ...
  public boolean equals(Object o) {
    if (o == null) {
      return false;
    }
    if ( ! (o instanceof Rectangle)) {
      return false;
    }
    Rectangle r = (Rectangle) o;
    return (this.width == r.width && this.height == r.height);
  }
}

  Rectangle r1 = new Rectangle(1D);
  Rectangle r2 = new Rectangle(2D);
  Square s1 = new Square(1D);

  System.out.println(r1.equals(r1)); // true
  System.out.println(r1.equals(r2)); // false
  System.out.println(r1.equals(s1)); // true
  System.out.println(s1.equals(r1)); // true
  // carefull if you override the equals method in Square
\end{blackboard}





















\end{document}
