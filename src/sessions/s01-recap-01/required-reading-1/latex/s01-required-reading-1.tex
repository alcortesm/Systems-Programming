\documentclass[a4paper, 12pt]{article}

\usepackage[handout]{style}

\newcommand{\realtitle}{Session 01 - Required Reading 1}

\begin{document}

\makebox[\linewidth]{\rule{\textwidth}{0.4pt}}
UC3M \hfill Alberto Cortés Martín\\
Systems Programming, 2014-2015 \hfill version: \today\\
\makebox[\linewidth]{\rule{\textwidth}{0.4pt}}
\begin{center}
  \Large{\realtitle}\\Working with Linux in the Labs
\end{center}
\makebox[\linewidth]{\rule{\textwidth}{0.4pt}}
\vspace{0.5cm}

\tableofcontents
\vspace{1cm}

We will be using Linux in this course as the main programming environment.
Most students already use Linux daily, in their phones and tablets, but few
have prior experience on using Linux as a programming environment.

This document contains a few short explanations and some links to external
resources that will help the student to learn the basics about using Linux as a
programming environment.

Disclaimer: The authors of the resources linked in this document are not
involved with this course in any way. Please do not bother them with your
questions; ask the course teaching staff instead.

\section{Connecting to your Linux account}

The first thing you need to be able to connect to your UC3M Linux account is
your Linux account itself. Your Linux account will be automatically created for
you the first time log into a Linux computer at the University labs (Sadly, you
have to be phisically there, sitting at a computer in the labs, you cannot
create your account remotely).

Once your Linux account has been created, you can log into it remotely from
anywhere: from home, the library, the cafeteria\ldots

The authentication credentials to log into your Linux account (locally or
remotely) are the same as for your Windows account.

\subsection{Connecting to your Linux account at the labs (local connection)}

  If you are sitting at a computer in the labs, you can log into your Linux
  account using the following procedure:

  \begin{enumerate}

    \item Boot into Linux

      All computers in the labs are configured to
      dual-boot into Linux or Windows. If the computer is currently running
      Windows, just reboot it and choose Linux at the boot menu.

    \item Log into your account

      After the boot process has completed, you
      will be asked for a login name and a password. Enter the same login and
      password as for your student Windows account.

    \item Logging out

      Once you have finished working, remember to log out from
      your account using a graphical menu you will find in your desktop manager
      (details on how to do this depend on the actual desktop manager you are
      running).

      There is no need to reboot the computer back into Windows
      before you leave, as most computers in the lab will be happier if you
      leave them running Linux (this is a personal appreciation).

  \end{enumerate}

  \subsection{Connecting to your Linux account from outside the labs (remote
  connection)}

  You can work on your Linux account remotely, connecting from home, a
  cafeteria, a library, etc.  You will find remote connections useful while
  working on this course on weekends, outside of class hours, etc.

  To connect to your Linux account remotely, follow this procedure:

  \begin{enumerate}

    \item Install an SSH client for your Operating System.

      You can find more information about SSH
      \href{http://en.wikipedia.org/wiki/Secure_Shell}{here}.

      If you are running Windows, I recommend using
      \href{http://www.chiark.greenend.org.uk/~sgtatham/putty/download.html}{PuTTY}.

      If you are using Linux or Mac, there is a big chance you
      already have an SSH client installed in your computer, the program is
      called ``\texttt{ssh}.'' If this program is not already installed, just
      install it.

    \item Connect to the SSH server of the UC3M

      Use your SSH client of choice to connect to the SSH server of the UC3M at
      ``\texttt{linux.aig.uc3m.es}'' (port is 22).

      Make sure the fingerprint of the server you are connecting to is:

      \begin{center}
        \texttt{e9:98:4d:37:18:07:53:eb:33:3b:06:8d:c5:40:c2:f7}
      \end{center}

      Otherwise, drop your connection immediately, as you are probably
      suffering a
      \href{http://en.wikipedia.org/wiki/Man-in-the-middle_attackman-in-the-middle}
      man-in-the-middle attack.

      You can personally verify the fingerprint of the UC3M's SSH server at
      room 1.1.J07 during working hours, if you think this document may has
      been compromised as part of the attack.

      On a successful connection attempt, you will be welcomed with a message
      from the Linux admins, like in this example:

\begin{blackboard}
alcortes@bat:~$ ssh linux.aig.uc3m.es
alcortes@linux.aig.uc3m.es's password: 

       Bienvenido a Aulas Informáticas Generales
          Universidad Carlos III de Madrid

          Puede contactar con nosotros en:
     
                gsai@aig.uc3m.es

Last login: Tue Jan 13 13:53:33 2015 from bat.it.uc3m.es
alcortes@linux:~$
\end{blackboard}

    \item Logging out

      Once you have finished working, you can log out from
      your account by exiting from your login shell with the ``\texttt{exit}''
      command.

  \end{enumerate}

You can find more information about your Linux account at
\href{http://www.aig.uc3m.es/index.php?option=com_content&task=view&id=28&Itemid=49}{Aulas
Informáticas - UC3M - Linux} (sorry, only in Spanish).


\section{Configuring your Linux Account}

Thankfully pretty much everything is already configured and ready to go,
except for one very important detail: UC3M's Linux accounts are configured
so that most programs will use Spanish instead of English.

Even if you can read/write Spanish just fine, \textbf{this is still a mayor
issue}, because when a program outputs an error or some other important
information in Spanish it can be very hard to google it.

Therefore, to be able to debug your Java code quickly or fix any other issues
with your Linux account you should configure it to talk English instead of
Spanish. Here is how to do it:

\subsection{Configuring Locales to English}

To configure your Linux account to use English in all its programs, input the
following commands into a terminal:

\begin{blackboard}
alcortes@linux:~$ export LANG=en_US.utf8
alcortes@linux:~$ echo 'LANG=en_US.utf8' >> ~/.profile
\end{blackboard}

The command at line 1 tells your current ongoing session to use English. The
command at line 2 modifies a configuration file to make the same change
automatically every time you login.

Run those two commands once, preferably, the first time you log into your
account, and you will never have to worry again about programs using Spanish
instead of English.

The teachers will not answer student's questions if they do not have their
account correctly configured to use English.

\section{Linux as a Programming Environment}

The hardest part of learning to use Linux as a programming environment is
learning how to use the Command Line.

You will find a nice introduction to the Linux command line
\href{http://linuxcommand.org/lc3_learning_the_shell.php}{here}.

There is also a very good book about this topic: ``The Unix Programming
Environment,'' by Brian W. Kernighan and Rob Pike, published by Prentice-Hall.
Unix and Linux are two different operating systems, although so similar that
you can this book to learn both.


\section{Working with Java in Linux}

To work with Java in Linux you will need at least three kinds of programs:

\begin{enumerate}

  \item \textbf{A good text editor}: You will spend most of your time in this
    course writing programs, so you will need a good (programming) text
    editor.

    I recommend you to learn either Vim or Emacs. They are both very powerful
    programming text editors, well documented and widely used by programmers
    everywhere. They are also already installed in the Linux UC3M's
    servers.

    You can learn how to use the Vim editor by reading this
    \href{ftp://ftp.vim.org/pub/vim/doc/book/vimbook-OPL.pdf}{guide} or by
    using the command \texttt{vimtutor}.  You can learn to use the Emacs editor
    by reading this \href{http://www.gnu.org/software/emacs/tour/}{guide}.

    In case you do not want to learn Vim or Emacs, you can use any other text
    editor, at your own risk.

    We will not be using IDEs (like Eclipse, for example) in this course as
    they will slow your learning process and has been responsible for 50\% of
    student fails in the past (this is another personal appreciation).

    The teachers will not answer student's questions if they use Eclipse or
    other IDEs, as often, the student problem turn out to be with the IDE, not
    with Java.

  \item \textbf{A Java Compiler}: There is already a Java compiler installed in
    the Linux servers called \texttt{javac}.

    \begin{blackboard}
alcortes@linux:~$ javac -version
javac 1.7.0_13
\end{blackboard}

  \item \textbf{A Java Virtual Machine}: There is a Java Virtual Machine
    installed in the Linux servers called \texttt{java}.

    \begin{blackboard}
alcortes@linux:~$ java -version
java version "1.7.0_13"
Java(TM) SE Runtime Environment (build 1.7.0_13-b20)
Java HotSpot(TM) Client VM (build 23.7-b01, mixed mode)
\end{blackboard}


\end{enumerate}

\section{Appendix: Working Remotely with Graphical Applications}

In the previous sections we have seen how to connect to the labs and work
remotely using text-based applications.

There are times when you need to launch a graphical application in the remote
server, for instance, a drawing tool. To run this kind of aplications
you will have to export the server's windowing system to your local host.

The UC3M's SSH server is running the ``X Window System'' as its windowing
system.  Here follows how to export it to your local host:

\begin{itemize}

  \item If your local host is also running Linux with an X Window System, just
    connect to the server passing the ``\texttt{-X}'' argument to your SSH
    client, for instance:

    \begin{blackboard}
$ ssh -X linux.aig.uc3m.es
alcortes@linux.aig.uc3m.es's password: 

       Bienvenido a Aulas Informáticas Generales
          Universidad Carlos III de Madrid

          Puede contactar con nosotros en:
     
                gsai@aig.uc3m.es

Last login: Mon Jan 19 12:05:31 2015 from bat.it.uc3m.es
alcortes@linux:~$ xeyes 
\end{blackboard}

  \item If your local host is running Windows, you will need to run an X server
    for Windows and connect it to the UC3M's SSH server through your SSH client
    of choice.

    I have no personal experience installing and using X servers for Windows,
    but the ``\href{http://www.straightrunning.com/XmingNotes/}{Xming X
    Server}'' seems to be the most popular choice.

    There are plenty of online resources explaining how to install an Xming X
    Server in Windows and connect it to an SSH server using PuTTY.

\end{itemize}

\end{document}
