\documentclass[8pt, a4paper, oneside]{extarticle}
\usepackage[english]{babel}
\usepackage[utf8]{inputenc}
\usepackage{graphicx}
\usepackage{hyperref}
\usepackage{amsmath}
\usepackage{nicefrac}
\usepackage[includehead, includefoot, left=1.5cm, right=1.5cm, top=2cm, bottom=2cm]{geometry}

\usepackage{eurosym}
\usepackage{CJKutf8}

\usepackage{fancyhdr}
\pagestyle{fancy}
\fancyhf[HL]{}
\fancyhf[HC]{\realtitle}
\fancyhf[HR]{}
\fancyhf[FL]{}
\fancyhf[FC]{\thepage}
\fancyhf[FR]{}

\usepackage{multicol}
\usepackage{fancyvrb}
\RecustomVerbatimEnvironment{Verbatim}{Verbatim}
{frame=single, commandchars=\\\{\}}

\DefineVerbatimEnvironment{blackboard}{Verbatim}
{frame=single}

\usepackage{enumitem}
\setitemize{noitemsep,topsep=0pt,parsep=0pt,partopsep=0pt}

\newcommand{\realtitle}{Session 01 - Recap 1}

\begin{document}

\title{Systems Programming\\\realtitle\\Lecture Notes}
\author{Alberto Cortés}
\date{\today}
\maketitle

\section{The World according to Java}

Java sees the world as a big empty space.

A program place data (numbers) in it, and asks Java to perform computations on
them.

That big empty space is the computer memory, and looks like this:

\begin{figure}[!ht]
  \centering
  \includegraphics{img/computer-memory/computer-memory.eps}
\end{figure}

Each small cell is a bit, and can hold a ``1'' or a ``0'', so the memory is not
really empty, it always holds some value.

8 bits are called a byte, and the bytes are numbered from 0 to some big number.

This are called memory addresses.

This is what java does when you execute a program:

\begin{figure}[!ht]
  \centering
  \includegraphics{img/world-data/world-data.eps}
\end{figure}

\verb+int i = 5+ will do 3 things:

\begin{itemize}

  \item allocate enough space in memory to store a int (32 bits) in some
    \emph{random} possition (e.g. from address 60 to 63)

  \item call that memory chunk \verb+a+ from now on (instead of 60).

  \item overwrite that memory with \verb+00...0101+ (32 bits)

\end{itemize}

\subsection{A simple real world problem}

\begin{Verbatim}
Print a conversion table from Celsius to Fahrenheit, from 0 to 50 in steps of 10.

Tip: the mathematical function to convert Celsius degrees to Fahrenheit degrees
is:

f(x) = x * 9/5 + 32  (dibuja la gráfica xy)

Example:

$ java CelsiusToFahrenheit
0.0 32.0
10.0 50.0
20.0 68.0
30.0 86.0
40.0 104.0
50.0 122.0
\end{Verbatim}

\begin{multicols}{2}
\VerbatimInput[numbers=left, numbersep=2pt, frame=single]{../java/CelsiusToFahrenheit.java}
\columnbreak

\begin{itemize}

  \item Lines 1 to 13 are the declaration of the class.

  \item Lines 2 to 7 are the definition of the main method, the entry point of
    the program.

  \item Lines 9 to 13 are the definition of an auxiliary method.

  \item Line 5 shows the output to the user.

  \item Line 4 is the one that actually does the work (translating degrees).

\end{itemize}

\end{multicols}

\subsection{The Memory while the program is running}

\begin{figure}[!ht]
  \centering
  \includegraphics{img/world-ctof-1/world-ctof-1.eps}
\end{figure}
\begin{figure}[!ht]
  \centering
  \includegraphics{img/world-ctof-2/world-ctof-2.eps}
\end{figure}
\begin{figure}[!ht]
  \centering
  \includegraphics{img/world-ctof-3/world-ctof-3.eps}
\end{figure}

\section{Data Types in Java}

As you have seen, programming is all about creating and operating with data in
the memory of the computer.

What types of data there are in Java then?

There are 2 types of data:

\begin{itemize}

  \item Primitive Types: int, float, char, boolean\ldots

  \item Reference Types: (Arrays and Objects) String, Person, int[]\ldots

\end{itemize}

When you work with primitive types, you store their values in ``variables'' and
operate on the data with ``operators''. For isntance:

\begin{blackboard}
int a = 4;
int b = 6;
int c = a + b;
// a, b, and c are variables
// = and + are operators
\end{blackboard}

When you work with composite types, you store their addresses in ``references''
and operate on them by calling ``methods''.

\begin{blackboard}
Object a = new Object();
Object b = new Object();
a.toString();
a.equals(b);
// a and b are references to Objects
// .toString and .equals are methods to operate on Objects
\end{blackboard}

We will go back to this later, now lets review the Java Primitive Types:

\subsection{Java Booleans}

A boolean in Java is a primitive data type that can store only 2 values:

\verb+true or false+

To declare a boolean (allocate memory for it and give it a name), you do
something like this:

\begin{blackboard}
                 ___
boolean a;     a |?|
                 ---
\end{blackboard}

\verb+boolean+ here is a reserved word, that tells Java that you want to
allocate memory for a boolean and call that chunk of memory \verb+a+ from now
on.

Values: only 2 possible values.

Literals: \verb+true+ \& \verb+false+.

You can operate on booleans using operators:

\begin{itemize}

\item \verb+=+: (assignment opertor), stores a value inside the variable, valid
  values are only \verb+true+ or \verb+false+. Example: \verb+a = true+.

  It returns its \emph{lvalue}, so chaining is possible:

\begin{blackboard}
boolean a;
boolean b;
boolean c;
c = b = a = true;
// a = true;
// b = a;
// c = b;
\end{blackboard}

\item \verb+==+: (Equal to operator), tells you if the value stores in a
  boolean variable is equal to the boolean value on the other variable. Returns
  a boolean. The oposite is the ``Not equal to'' operator (\verb+!=+). Example:

\begin{blackboard}
boolean a = true;
boolean b = false;
boolean c = a == b; // c is false
boolean d = a != b; // d is true
\end{blackboard}

\item \verb+!+: (Logical complement operator), inverts the value of a boolean.

\begin{blackboard}
boolean a = true;
boolean b = !a; // a is false
b = !b; // a is true again
\end{blackboard}

\item \verb+|| and &&+: (Conditionals operators), logical OR and logical AND:

\begin{blackboard}
boolean a = true;
boolean b = false;
boolean or = a || b; // or is true
boolean and = a && b; // and is false
\end{blackboard}

\end{itemize}

Printing: using the \verb+java.io.PrintStream.println(boolean b)+ method.

\subsection{Java Ints}

\begin{itemize}

  \item Declaration (allocates 32 bits) \verb+int a;+

  \item Can hold integers from -2\_147\_483\_648 ($-2^{31}$) to 2\_147\_483\_647
    ($2^{31}-1$) (about 2 billions) \\ (Compilation error: \verb+error: integer number too large+)

  \item Literals: \verb+26+, \verb+0x1a+ or \verb+0b11010+. Underscores are
    ignored for readability (1\_000 is same as 1000).

  \item Operators:

    \begin{itemize}

      \item Arithmetic: \verb^+, -, *, /, %^

\begin{blackboard}
int a = 7;
int b = 3;
int c = a + b; // 10
c = b - a; // -4
c = a * b; // 21
c = a / b; // 2
c = a % b; // 1
\end{blackboard}

      \item Unary: \verb^+a, -a, ++a, a++, --a, a--^

\begin{blackboard}
int a = 7;
int b = +a; // 7 (not useful)
b = -a; // -7 (changes the sign)
b = ++a; // a=8 b=8
b = a++; // a=9 b=8
b = --a; // a=8 b=8
b = a--; // a=7 b=8
\end{blackboard}

      \item Relational: \verb^==, !=, <, <=, >, >=^

\begin{blackboard}
int a = 7;
int b = 3;
boolean c = a == b; // false
c = a != b // true
c = a < b // false
c = a <= b // false
c = a > b // true
c = a >= b // true
\end{blackboard}

      \item Bitwise: \verb+~, <<, >>, >>>, &, |, ^+ Don't worry about them in
this course.

    \end{itemize}

  \item Printing: using the \verb+java.io.PrintStream.println(int b)+ method.

\end{itemize}

\subsection{Java small integers (byte, short)}

They are just like int, but smaller:

\begin{itemize}

  \item Declaration \verb+byte a;+ or \verb+short a;+
  \item \verb+byte+ is 8 bits (-128 ($-2^{7}$) - 127 ($2^{7} -1$)).
  \item \verb+short+ is 16 bits (-32\_768 ($-2^{15}$) - 32\_767 ($2^{15}-1$)), less than 33 thousands.
  \item Literals: the same as for ints.

\begin{blackboard}
byte b1 = 15;
byte b2 = 200; // compilation error: possible loss of precision.
short s1 = 15_000;
short s2 = 33_000; // compilation error: possible loss of precision.
\end{blackboard}

  \item Printing: using the \verb+java.io.PrintStream.println(byte b);+ or
    \verb+java.io.PrintStream.println(short b)+ methods.

\end {itemize}

\subsection{Java big integers (long)}

\begin{itemize}

  \item Declaration \verb+long a;+

  \item 64 bits long (-lots ($-2^{63}$) - +lots ($2^{63}-1$) (about a 9 and 18 zeros).

  \item Literals: int literals can be used, but for big numbers the sufix
    \verb+L+ or \verb+l+ has to be used

\begin{blackboard}
long l1 = 15L;
long l2 = 15;
long l3 = 3_000_000_000; // compilation error, but some compilers are ok with it
long l4 = 3_000_000_000L;
\end{blackboard}

  \item Printing: using the \verb+java.io.PrintStream.println(long b)+ method.

\end{itemize}

\subsection{Java characters (char)}

Represent characters in Unicode (don't need to know).

They are really shorts, but gets converted to chars using conversions tables,
also know as \emph{Charset}s.

\begin{itemize}

  \item Declaration \verb+char a;+

  \item 16 bits long, there are 65,536 possible chars.

  \item Literals:

    \begin{itemize}

      \item \verb+'a'+, \verb+'7'+, \verb+','+, \verb+'ñ'+, \verb+'+$\beta$\verb+'+,
        \verb+'+$\infty$\verb+'+, \verb+'+\euro\verb+'+,
      \verb+'+$\forall$\verb+'+, \verb+'+\begin{CJK*}{UTF8}{min}茶\end{CJK*}\verb+'+.

      \item invisible charactes: tab \verb+'\t'+, space \verb+' '+, linefeed
        \verb+'\n'+, carriage return \verb+'\r'+, single quote \verb+'\''+,
        double quote \verb+'\"'+, backslash \verb+'\\'+\ldots

      \item by unicode code: \verb+'\u0041'+ (character A)

    \end{itemize}

  \item Operators: 

    \begin{itemize}

      \item Arithmetic: Don't use them unless you know alot about Charsets.

      \item Unary: Don't use them unless you know alot about Charsets.

      \item Relational: \verb^==, !=, <, <=, >, >=^ are OK.

    \end{itemize}

  \item Printing: using the \verb+java.io.PrintStream.println(char b)+ method.

\end{itemize}

\subsection{Floating point numbers (float and double)}

Floats and doubles are fast but inexact representations of real numbers.

\begin{itemize}

  \item Declaration: \verb+float f;+ or \verb+double d;+

  \item float: 32-bit IEEE 754 floating point number (ignore this for this course).

  \item double: 64-bit IEEE 754 floating point number (ignore this for this course).

  \item Literals:

    \begin{itemize}

      \item floats end in \verb'f' or \verb'F'.

      \item doubles end in \verb'd' o \verb'D' or nothing.

    \end{itemize}

  \item Operators: the same as ints, but without the remainder.

  \item They are inexact:

\begin{blackboard}
double a = 0.1d;
boolean b = (a + a) == 0.2d; // true
boolean c = (a + a + a) == 0.3d; // false
System.out.println(a + a + a); // 0.30000000000000004
\end{blackboard}

    This is way floats and doubles are not used for important operations like
    currency, the class BigDecimal is used instead (session 03).

  \item Printing: using the \verb+java.io.PrintStream.println(float f)+ or the
    \verb+java.io.PrintStream.println(double d)+ method.

\end{itemize}

\subsection{Review}

\begin{itemize}

  \item boolean - true or false

  \item byte (8 bits) -128 to 127

  \item short (16 bits) -32 thousand to 32 thousand

  \item int (32 bits) -2 billions to 2 billions

  \item long (64 bits) -lots to lots

  \item float (32 bits) not very precise real numbers

  \item double (64 bits) double precission real number

  \item char (16 bits) really numbers in disguise using the Unicode Charset table

\end{itemize}

\subsection{Constant Variables}

This looks like an oximoron, but it is not.

To declare a variable so that once initiliazed, its value can not be changed,
use the \verb+final+ keyword.

\begin{blackboard}
final int DAYS_IN_A_WEEK = 7;
final boolean YES = true;
final char ALPHA = '\u03B1';

DAYS_IN_A_WEEK = 8; // compilation error: cannot assign a value to final variable

final int NUM_SESSIONS;
NUM_SESSIONS = 29;
NUM_SESSIONS = 34; // compilation error: cannot assign a value to final variable
\end{blackboard}

\subsection{Castings}

A casting is interpreting a variable as if it has a different type, example:
turn an int into a long.

An implicit castings is performed when casting a small variable into a big one.
Are done automatically by Java.

\begin{blackboard}
int i = 3;
long l = i; // implicit casting, l is 3L

long l = 3L;
int i = l; // compilation error: possible loss of precision
\end{blackboard}

Explicit castings are performed when casting a big variable into a smaller one.
As you can loose some data, they are not performed automatically by Java, you
must request the conversion explicitly by using the casting operator
\verb+(type)+, which is a way of telling Java, it's OK, I know what I'm doing.

\begin{blackboard}
long l = 3L;
int i = (int) l; // explicit casting, works, i is 3

long l = 3_000_000_000L;
int i = (int) l; // explicit casting, works, i is -1294967296, is this what you want?
\end{blackboard}

You can cast all primitive types into any other (implicitly or explicitly),
except booleans.

\end{document}
