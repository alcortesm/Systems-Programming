\documentclass[8pt, a4paper, oneside]{extarticle}

\usepackage[utf8]{inputenc}
\usepackage[english]{babel}
\usepackage{graphicx}
\usepackage{hyperref}
\usepackage{amsmath}
\usepackage{nicefrac}
\usepackage[includehead, includefoot, left=1.5cm, right=1.5cm, top=2cm, bottom=2cm]{geometry}

\usepackage{fancyhdr}
\pagestyle{fancy}
\fancyhf[HL]{}
\fancyhf[HC]{\realtitle}
\fancyhf[HR]{}
\fancyhf[FL]{}
\fancyhf[FC]{\thepage}
\fancyhf[FR]{}

\usepackage{lipsum}
\usepackage{multicol}
\usepackage{fancyvrb}
\RecustomVerbatimEnvironment{Verbatim}{Verbatim}
{frame=single, commandchars=\\\{\}}

\usepackage{enumitem}
\setitemize{noitemsep,topsep=0pt,parsep=0pt,partopsep=0pt}

\newcommand{\realtitle}{Session 01 - Recap 1}

\begin{document}

\title{Systems Programming\\\realtitle\\Lecture Notes}
\author{Alberto Cortés}
\date{\today}
\maketitle

\section{What is ``Programming''}

\begin{multicols}{2}
  \begin{Verbatim}
"The process of \underline{writting} computer programs."

 - \underline{Think} what you want to write about
 - Use \underline{proper} English/Java
 - \underline{Proofread} what you wrote
  \end{Verbatim}
  \columnbreak

  So, programming is basically writting,
  like when you write a book, a novel or an essay.

  When you write a text you need to:
  \begin{itemize}
    \item Think what you want to write
    \item Write it on proper English/Java
    \item Proof read what you wrote
  \end{itemize}

  Knowing proper English (or Java) is important for writting, but the thinking
  part and the proofreading is also essential.

  A 10 year old child already know how to write in proper English, but would
  not write very good books, because he don't know yet how to think or how to
  proofread.

  You already know how to write proper Java from the first term, in this second
  term you will learn how to think about what you want to programm and how to
  test it.
\end{multicols}

\subsection{Programming in high level languages}

\begin{multicols}{2}
\begin{Verbatim}
"Find a solution to a real world problem by \underline{simulating} it
 inside the computer memory."
\end{Verbatim}
  \columnbreak

  A common definition for programming in high level computer languages (like
  Java) is: Find a solution to a real world problem by simulating it inside
  your computer memory.

  In this definition, the ``simulating'' part is the important part. Also you
  will need to know a little bit about the computer memory for programming and
  how the world looks like to Java.

\end{multicols}

\subsection{Representing Data Inside the Computer Memory}

\begin{figure}[!ht]
  \centering
  \includegraphics{img/world-data/world-data.pdf}
\end{figure}

\subsection{A simple real world problem}

\begin{Verbatim}
Find the maximum of two integers.  -->   $ java Max 3 7
                                         7
                                         $ java Max -2 -6
                                         -2
                                         $ java Max 237 237
                                         237
\end{Verbatim}
\newpage
\begin{multicols}{2}
\VerbatimInput[numbers=left, numbersep=2pt, frame=single]{../java/Max.java}
\columnbreak

\begin{itemize}

  \item Line 1 is a comment (this programm is incomplete because there is no
    error checking code in it.

  \item Lines 2 to 19 is the declaration of the class.

  \item Lines 3 to 10 are the definition of the main method, the entry point of
    the program.

  \item Lines 12 to 18 are the definition of another method.

  \item Lines 4 and 5 are for processing the input arguments into something
    useful (Strings to ints)

  \item Line 9 shows the output to the user.

  \item Line 7 is the one that actually does the work (finding the max).

\end{itemize}

\end{multicols}

\section{The World According to Java}

\begin{figure}[!ht]
  \centering
  \includegraphics{img/world-max-1/world-max-1.pdf}
\end{figure}
\begin{figure}[!ht]
  \centering
  \begin{minipage}{.5\textwidth}
    \centering
    \includegraphics{img/world-max-2/world-max-2.pdf}
  \end{minipage}%
  \begin{minipage}{.5\textwidth}
    \centering
    \includegraphics{img/world-max-3/world-max-3.pdf}
  \end{minipage}
\end{figure}


\section{Primitive Data Types}

\end{document}
