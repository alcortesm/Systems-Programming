\documentclass[a4paper, 9pt]{extarticle}

\usepackage[notes]{style}

\newcommand{\realtitle}{Session 12 - GUI 02}

\begin{document}

\makebox[\linewidth]{\rule{\textwidth}{0.4pt}}
UC3M \hfill Alberto Cortés Martín\\
Systems Programming, 2014-2015 \hfill version: \today\\
\makebox[\linewidth]{\rule{\textwidth}{0.4pt}}
\begin{center}
  \Large{\realtitle}\\Lecture Notes
\end{center}
\makebox[\linewidth]{\rule{\textwidth}{0.4pt}}
\vspace{1cm}


\section{Today's Topics}
\begin{blackboard}
Events
\end{blackboard}


\newpage

\section{Review of last lecture}

\begin{itemize}

  \item In Swing, a window is called a \texttt{JFrame}.

  \item A \texttt{JFrame} has a \texttt{JPanel} called the \texttt{contentPanea}.

  \item \texttt{JPanel}s are like corkboards, where you can pin other graphical objects, using its method \texttt{add()}.

  \item You can add any \texttt{JComponents} (\texttt{JLabel}, \texttt{JButton}...) to the \texttt{contentPane}.

  \item All \texttt{JComponents} have some common properties: they have a background
    color, a foreground color, and things you can configure like a preferred
    size, a minumum size, a maximum size...

  \item Note that as \texttt{JPanel} is also a \texttt{JComponent}, so you can add \texttt{JPanels} inside other \texttt{JPanels}, to create complex windows.

  \item \texttt{JComponent}s inside a \texttt{JPanel} will be drawn according to the \texttt{LayoutManager} of that \texttt{JPanel}.

  \item There are several types of \texttt{LayoutManager}s: \texttt{GridLayout}, \texttt{BorderLayout}, \texttt{FlowLayout}...

  \item All graphical operations must be performed inside the Event Dispacher Thread (EDT).

  \item You can tell Java to execute code inside the EDT by passing a FUNCTOR to the \texttt{SwingUtilities.invokeLater()} method.

  \item Most of the times is not worth it to create FUNCTORs as ordinary classes, better use NESTED classes or ANONYMOUS classes.

\end{itemize}

\vspace{1cm}
\begin{center}
\includegraphics[width=18cm]{./img/jframe/jframe.eps}
\end{center}

\newpage







\section{How to make a swing program react to the users inputs}

Some JComponents are interactive, like buttons for example.

We would like to tell our program: when the user press this button, please
execute this method. If the user press this other button, then execute this
other method. Meanwhile, just do nothing.

Of course, you already know that you can not pass methods in Java, but you can pass funtors.

So you would like to tell Java: when the user press this button, then execute
this funtor, if he presses this other button, then execute that other functor,
and so on and so forth.


\section{The Swing Event Dispacher Thread}

The EDT is constantly monitoring the user mouse and keyboard, waiting for user inputs.

When the user interacts with any JComponent (for example, when he press a button), the EDT will generate an Event.

Depending on the user action, JComponents will create different kinds of Events:

\begin{itemize}

  \item \texttt{MouseEvent}:

    \begin{itemize}

      \item mouse button is pressed

      \item mouse button is released

      \item the mouse cursor enters Jcomponent area

      \item the mouse cursos leaves JComponent area.

      \item the mouse is moved

      \item the mouse is dragged

    \end{itemize}

  \item \texttt{MouseWheelEvent}

    \begin{itemize}

    \item Mouse Wheel Up

    \item Mouse Wheel Down

    \end{itemize}

  \item \texttt{KeyEvent}

    \begin{itemize}

    \item Key pressed

    \item Key released

    \end{itemize}

\end{itemize}

These are all pretty low level events, and your very simple graphical
application should not care about them, just let Swing take care of them in the
standard way.

Instead of handling these kind of low level events, you will like to handle
more high level events:

Instead of detecting a MOUSE\_CURSOR\_ENTER\_JCOMPONENT, then MOUSE\_BUTTON\_PRESSED and
then a MOUSE\_BUTTON\_RELEASED, you will like to detect a single BUTTON\_CLICKED events.

Those kind of high level events are called \texttt{ActionEvents}:

\section{ActionEvent}

The class ActionEvent is a class that represents high level events from AbstractButtons.

All AbstractButtons will create ActionEvents when a particular combination of the
low level events is detected:

For example, when you click on a button, the MOUSE\_BUTTON\_PRESSED and then the
MOUSE\_BUTTON\_RELEASED event are created, that is a hint that the user has click
on the button.

The JButton is listening to those events and asking Swing to redraw itself so
you can see that the button is pressed and then it is redraw again to show that
the button was released. There is a high chance you don't want to change that
in a simple application.

But when that particular combination of low level events happend, the button
also create a new ActionEvent for the clicking of the button. But this time,
the JButton is not handling that event and is not doing anything with it. You
as the programmer of the application must handle that event and do something
interisting in you application.

But you can also click on a button without using the mouse, you can press the
TAB key until the graphical focus is set on the button you want to press and
then press the SPACEBAR key, that will be like clicking the button.

In this case the graphical application will have created some KEY PRESSED
events and some KEY RELEASED events for the clicking of the TAB key and for the
SPACEBAR key. The JButton will also be listening to those events and redraw itself
so it looks like you are pressing and releasing the button.

Again, after doing all that, the JButton is creating the same ActionEvent as
before, when you clicked the button with the mouse, so your application can
handle that event and do what it has to do when the button is pressed.

The class ActionEvent looks like this:

\begin{center}
  \includegraphics[width=8cm]{./img/actionevent/actionevent.eps}
\end{center}

When a button creates an ActionEvent, it fills in the source and command for the event.

The source of the event is the JComponent creating the event, in our example the button.

You can get the source of an ActionEvent by calling the \texttt{getSource()}
method, it will return the JComponent that create that event.

For example: if you have several buttons in your application, the events
created by pressing each button will have a different source (each of the
JButtons).

When a button is pressed, you will receive an ActionEvent created by that
button. You can use the \texttt{getSource()} methodto know which of your
buttons was pressed, for example.

The \texttt{command} of an ActionEvent is additional information in the event
about the button that was pressed. You can ask for the command inside an event
with the method \texttt{getActionCommand}.

This allows you to create MODAL buttons, example: you can have a button to show
or hide a portion of the window. When you press the button once, an ActionEvent
will be created with that button as the source and the command \verb+"SHOW"+,
when you press the button again, it will create an ActionEvent with the same
button as its source but with the command \verb+"HIDE"+.

So, you use the source of an event to identify what button was pressed and you
use the command of an event to identify what action has to be done if the same
button can do several things at different moments.

Null commands are allowed if you don't want to use modal buttons.






\section{Handling ActionEvents}

By default, all the ActionEvents created by the JComponents of your window will be ignored.

That is why the windows you have been doing untill now don't do anything when a button is pressed.

If you want to handle an ActionEvent, you have to tell your AbstractButtons
what to do with the ActionEvent, this is done by adding a funtor to your button
that receives the ActionEvent as a parameter.

These funtors are called Listeners in Swing.

All AbstractButtons have an \texttt{addActionListener()} method that you can use to
add an ActionListener (this is a funtor) to your graphical components.

The UML of this relationship looks as follows:

\begin{center}
  \includegraphics[width=12cm]{./img/actionlistener/actionlistener.eps}
\end{center}

So to do something when a button is pressed, you have to:

\begin{enumerate}

  \item Create a button.

  \item Set the actionCommand of that button if you want to used actionCommands.

  \item Create an ActionListener for your button with the code of what you want to do when the button is pressed.

  \item Add the ActionListener to your button.

\end{enumerate}

If you do that, then when the button is pressed:

\begin{enumerate}

  \item The button will create an ActionEvent with the source as itself and the actionCommand of the button.

  \item Then it will ask the EDT to execute the actioPerformed method of the actionListener that was added to the button.

  \item The actionPerformed method will receive the ActionEvent as an argument.

\end{enumerate}

\end{document}
