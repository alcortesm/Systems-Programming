\documentclass[a4paper, 9pt]{extarticle}

\usepackage[notes]{style}

\newcommand{\realtitle}{Session 26 - Heaps}

\newcommand{\separator}{\begin{center}%
\noindent\makebox[\linewidth]{\rule{0.75\paperwidth}{0.4pt}}%
\end{center}}

\begin{document}

\makebox[\linewidth]{\rule{\textwidth}{0.4pt}}
UC3M \hfill Alberto Cortés Martín\\
Systems Programming, 2014-2015 \hfill version: \today\\
\makebox[\linewidth]{\rule{\textwidth}{0.4pt}}
\begin{center}
  \Large{\realtitle}\\Lecture Notes
\end{center}
\makebox[\linewidth]{\rule{\textwidth}{0.4pt}}
\vspace{1cm}


\section{Priority Queues}

You already know about queues: queues are a standard way to process objects on
a \emph{first-come, first-served} scheduling, for example, clients in a store.

However, for many problems, some ``clients'' are more important than others:

\begin{itemize}

  \item Passengers waitting to board a plane, often first class passengers are
    going to board before Business class passengers, no matter how late they
    get into the airport.

  \item Your agenda is a collection of tasks that usually come in in a random
    order, but you usually start working on the most urgent task (this is, the
    task with the highest priority), no matter how late this task got into your
    agenda.

  \item An Internet router is constantly receiving data packages from all
    around the world, but it serves the packets with the highest priority
    first.

  \item Any real time simulation program, including most games: the game is
    constantly creating new events that will happend in the close future, they
    are being stored in a priority queue, so the most urgent event is simulated
    first. In this case the priority of an event is how close does it have to
    happen.

    Example: in a multiplayer game: several people are giving commands to the
    game, in which order are this commands going to happen? (begins casting a
    spell, another throw an arrow, does the arrow hit before the spell is
    cast?)

\end{itemize}

So the main operations on a priority queue are:

\codeinput{../java/PriorityQueue.java}{PriorityQueue.java}

\subsection{Operations on a Priority Queue}

\codeinput{pq_usage.txt}{Handout}

\subsection{Implementations}

\begin{center}
  \includegraphics[width=14cm]{./img/pq_implementations/pq_implementations-16.eps}
\end{center}


\subsection{The Heap Data Structure}

A binary heap is a binary tree with two properties:

\begin{enumerate}

  \item The \emph{shape} property: The tree is complete.

    This means the tree has the minimum possible height for its number of
    elements ($h = \lfloor log_{2} n \rfloor$).

    The excellent performance of this data structure comes from this property:
    contrary to binary search trees, there are not ``unbalanced'' heaps.

    This is also the reason why we use array based trees for heaps instead of
    trees based in linked nodes (more on this later).

  \item The \emph{heap order} property: For every node, its priority is $\ge$ than the
    priority of its children, if any.

    This means the node with the highest priority is always the root of the tree.

\end{enumerate}

\begin{center}
  \includegraphics[width=7cm]{./img/is_heap/is_heap-01.eps}
  \includegraphics[width=7cm]{./img/is_heap/is_heap-02.eps}
  \includegraphics[width=7cm]{./img/is_heap/is_heap-03.eps}
  \includegraphics[width=7cm]{./img/is_heap/is_heap-04.eps}
\end{center}

\begin{blackboard}
  is this tree a binary heap?

  yes, no
  no, yes
\end{blackboard}
\end{document}
