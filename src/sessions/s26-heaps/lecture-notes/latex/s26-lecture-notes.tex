\documentclass[a4paper, 9pt]{extarticle}

\usepackage[notes]{style}

\newcommand{\realtitle}{Session 26 - Heaps}

\newcommand{\separator}{\begin{center}%
\noindent\makebox[\linewidth]{\rule{0.75\paperwidth}{0.4pt}}%
\end{center}}

\begin{document}

\makebox[\linewidth]{\rule{\textwidth}{0.4pt}}
UC3M \hfill Alberto Cortés Martín\\
Systems Programming, 2014-2015 \hfill version: \today\\
\makebox[\linewidth]{\rule{\textwidth}{0.4pt}}
\begin{center}
  \Large{\realtitle}\\Lecture Notes
\end{center}
\makebox[\linewidth]{\rule{\textwidth}{0.4pt}}
\vspace{1cm}


\section{Priority Queues}

You already know about queues: queues are a standard way to process objects on
a \emph{first-come, first-served} scheduling, for example, clients in a store.

However, for many problems, some ``clients'' are more important than others:

\begin{itemize}

  \item Passengers waitting to board a plane, often first class passengers are
    going to board before Business class passengers, no matter how late they
    get into the airport.

  \item Your agenda is a collection of tasks that usually come in in a random
    order, but you usually start working on the most urgent task (this is, the
    task with the highest priority), no matter how late this task got into your
    agenda.

  \item An Internet router is constantly receiving data packages from all
    around the world, but it serves the packets with the highest priority
    first.

  \item Any real time simulation program, including most games: the game is
    constantly creating new events that will happend in the close future, they
    are being stored in a priority queue, so the most urgent event is simulated
    first. In this case the priority of an event is how close does it have to
    happen.

    Example: in a multiplayer game: several people are giving commands to the
    game, in which order are this commands going to happen? (begins casting a
    spell, another throw an arrow, does the arrow hit before the spell is
    cast?)

\end{itemize}

So the main operations on a priority queue are:

\codeinput{../java/PriorityQueue.java}{PriorityQueue.java}

\subsection{Operations on a Priority Queue}

\codeinput{pq_usage.txt}{Handout}

\subsection{Priority Queues Implementations}

\begin{center}
  \includegraphics[width=14cm]{./img/pq_implementations/pq_implementations-16.eps}
\end{center}


\subsection{The Heap Data Structure}

A binary heap is a binary tree with two properties:

\begin{enumerate}

  \item The \emph{shape} property: The tree is complete.

    This means the tree has the minimum possible height for its number of
    elements ($h = \lfloor log_{2} n \rfloor$).

    The excellent performance of this data structure comes from this property:
    contrary to binary search trees, there are not ``unbalanced'' heaps.

    This is also the reason why we use array based trees for heaps instead of
    trees based in linked nodes (more on this later).

  \item The \emph{heap order} property: For every node, its priority is $\ge$ than the
    priority of its children, if any.

    This means the node with the highest priority is always the root of the tree.

\end{enumerate}

\begin{center}
  \includegraphics[width=7cm]{./img/is_heap/is_heap-01.eps}
  \includegraphics[width=7cm]{./img/is_heap/is_heap-02.eps}
  \includegraphics[width=7cm]{./img/is_heap/is_heap-03.eps}
  \includegraphics[width=7cm]{./img/is_heap/is_heap-04.eps}
\end{center}

\begin{blackboard}
  is this tree a binary heap?

  yes, no
  no, yes
\end{blackboard}

\subsection{Insert Algorithm}

The insert algorithm has two parts:

\begin{multicols}{2}
\begin{enumerate}

  \item Place the new node at the end of the complete tree.

    That way the tree already satisfies the \emph{shape} property, but probably
    not the \emph{heap order} property.

  \item If the priority of your parent is smaller than yours, swap possitions
    with your parent. Repeat until the heap property is satisfied.

    This will fix the \emph{heap order} property eventually, without breaking
    the \emph{shape} property, as swapping does not modify the shape of the
    tree.

\end{enumerate}
\columnbreak
\begin{center}
  \includegraphics[width=8cm]{./img/insert/insert.eps}
\end{center}
\end{multicols}

\newpage
\subsection{RemoveMax Algorithm}

\begin{multicols}{2}

The root element is always the element with the maximum priority, therefore, we
do not have to search for it.

But once we extract it, the tree is broken, as it has not root.

To recover the tree shape, we have to choose one other node to substitute the root.

The best way to do that while mantaining the \emph{shape} property is to choose
the last node in the tree, and move it to the root.

This way the shape property is mantained.

Now, the heap order property is probably broken, but we can easily fix that by swapping:

If the priority of the node is smaller than any of its children, then swap with
the child with the biggest priority. Repeat until the heap property is satisfied.

\begin{enumerate}

  \item Return root, move las node to root.

  \item If priority of the node is smaller than any of its children,
    swap with the biggest child. Repeat until the heap property is satisfied.

\end{enumerate}
\columnbreak
\begin{center}
  \includegraphics[width=8cm]{./img/removeMax/removeMax.eps}
\end{center}
\end{multicols}


\subsection{Binary Heap Implementation}

As a binary heap is a binary tree, we can reuse the last week implementation of
a binary tree, with node that have parent, left and right references.

But since binary heaps are complete binary trees, it is much more simpler to
just use an array imlementation of a binary tree.

Do you remember the array imlementation of a binary tree from two weeks ago?

It looked like this:

\begin{center}
  \includegraphics[width=12cm]{../../../s22-trees/lecture-notes/latex/img/arraybtree/arraybtree.eps}
\end{center}

So let us implement a binary heap using an array:

\codeinput{../java/PriorityQueue.java}{PriorityQueue.java}

\begin{multicols}{2}
The first thing you have to realize is that even if we have been drawing heaps
as nodes with numbers (this is, the priority), a priority queue holds both the
priority of an entry and the value.

This is similar as what we had with disctionaries, where each entry in the
dictionary was a key and a value.

So, to implement a heap using an array, we will need a \texttt{Pair} class that
holds the priority and the value, or use two arrays in parallel, one of
priorities and another one of values:

This pair class does not officially need to implement the
\texttt{Comparable<Pair>} interface. But, as we will want to compare the
priorities of a pair quite often, instead of getting a pair, then getting its
priority, then comparing it, we can write cleaner code by just comparing pairs.

For this to work, one pair is going to be greater, equal or smaller than other
pair, just if its priority is greater, equal or smaller than the priority of
the other pair. This will save us a lot of typing.
\columnbreak

\codelinesinput{../java/Heap.java}{The pair internal class (entry will also be a good name)}{5}{25}
\end{multicols}

\separator

Now, let us see what attributes do we need for our heap:

\codelinesinput{../java/Heap.java}{Heap attributes and ctor}{27}{54}

\newpage

Now, let us write the easy methods: \texttt{isEmpty()} and \texttt{size()}.

\codelinesinput{../java/Heap.java}{Heap attributes and ctor}{56}{62}

Let us also write the \texttt{getMax} method, as it is pretty simple:

\codelinesinput{../java/Heap.java}{Heap attributes and ctor}{125}{130}

Before going with the complicated methods, let us define some helper functions:

\codelinesinput{../java/Heap.java}{Heap attributes and ctor}{64}{87}

Now let us write the insert method. Do you remember the algorithm?

\codelinesinput{../java/Heap.java}{Heap attributes and ctor}{89}{123}

And now let us finish with the removeMax():

\codelinesinput{../java/Heap.java}{Heap attributes and ctor}{132}{194}


\subsection{Min Heaps vs. Max Heaps}

What we have implement is called a max heap, because the heap order property
sort elements from higher priority to lower priority.

That is what you would expect from a priority queue and the real life meaning
of the term ``priority''.

But, there are also \emph{min heaps} where elements are store the other way
around, and the element with the lowest priority is always at the top.

Min heaps are just as max heaps but changing all the comparisons form $<$ to
$>$ and the other way around.

Contrary to what you whould expect, most programmers use min heaps instead of
max heaps, because in computer science, the things with the lowest priority are
the most important.

This is not a play on words, this is for real, for example in an operating
system, processess have priorities, your computer will execute the most
important ones before the least important ones, for example, the process doing
some calculations in an spread sheet will be more important than the SWING
process drawing its graphical interface.

Well, most operating systems uses low values for the most important processess
and high values for the least important, meaning that the execution queue of
processess will be a min heap, not a max heap (it is not really a heap, for
reasons you will learn in a few years).

Computer scienties are just that weird, what whould you expect from people that
grows trees upside down?


\end{document}

