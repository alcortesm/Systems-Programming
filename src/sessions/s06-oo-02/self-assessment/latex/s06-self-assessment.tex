\documentclass[a4paper, 11pt]{article}

\usepackage[assessment]{style}

\newcommand{\realtitle}{Session 06 - Object-Oriented Programming 2}

\begin{document}

\makebox[\linewidth]{\rule{\textwidth}{0.4pt}}
UC3M \hfill Alberto Cortés Martín\\
Systems Programming, 2014-2015 \hfill version: \today\\
\makebox[\linewidth]{\rule{\textwidth}{0.4pt}}
\begin{center}
  \Large{\realtitle}\\Self-Assessment
\end{center}
\makebox[\linewidth]{\rule{\textwidth}{0.4pt}}
\vspace{1cm}

From now on and to the end of the course, follow these simple rules when
writing code:

\begin{enumerate}

  \item \textbf{Write your constructors so that it is imposible to create
    an invalid object}.

    For example, do not allow a \verb+Circle+ constructor to
    create circles with negative radius.

    This means you will have to write some extra code in your constructors to
    check their parameters. But it is worth it, because no longer will you be
    needing to check the validity of your attributes in the code of your
    methods.

  \item \textbf{Do not add more methods to your class than the ones you really
    need} to solve the problem. Remember: more code means more bugs.

    For instance, do not automatically create getters and setters methods for
    all your attributes unless you really need them; the user do not need to
    know, and do not want to know the internals of your class, and you do not
    need to spend extra thinking time checking inputs for methods that are not
    really needed.

  \item \textbf{If your class has methods that modify the values of your
    attributes, use the same checks you use with your constructors}, so you can
    be sure that a valid objects remains valid no mather what the user do to
    it.

    For example, if you add a \verb+public void setRadius(int radius)+ to your
    circle class, the first thing you should do in that method is check that
    the radius is not negative. Then you may want to call this method from your
    constructor so you do not repeat code and checks inside your class.

  \item \textbf{Use the most restrictive access modifiers you can afford for
    your attributes, methods and constructors}: whenever you create one of
    them, ask yourself if you can get along with using \verb+private+. If you
    cannot, think about \verb+protected+ and only if there is no other option,
    use \verb+public+.

    That way, your class will be as safe as it can be, but will still be able
    to solve the problem.

  \item \textbf{Never trust the user input}. Always check the input parameters
    to your methods, as you do not know what kind of garbage the user is going
    to throw at you.

    For example, if your method receives an array of circles, check that the
    array is not null, check that it is not a zero-length array and if it is
    not, then be ready to cope with empty arrays or with arrays with holes in
    them.

\end{enumerate}









\section{Causality dilemmas: pirate or parrot?}

\subsection{}

Write a \verb+Person+ class that represents people.

A person will have a name, a nationality and a number of eyes between 0 and 2,
being 2 the default number of eyes for a person (this is, the number of eyes a
person will have if the number of eyes is not specified).

The \verb+toString()+ method must return a string with the following
format:

\verb+    [Person] name (numEyes eye)+

For example:

\verb+    [Person] Alberto (2 eyes)+

Remember to use the plural (``eyes'') if the person has 0 or 2 eyes, but use
the singular if the person has only 1 eye.

\solutioninput{../java/Person.java}








\subsection{}

Write a program called \verb+PersonTest+ that creates 3 people and prints them.

Use the following data for your people:

\begin{itemize}

  \item Alberto, Spanish, 2 eyes.

  \item Alicia, Spanish, 2 eyes.

  \item Ragetti, English, 1 eye.

\end{itemize}

Its output must be exactly like this:

\begin{cmd}
$ java PersonTest
[Person] Alberto (Spanish, 2 eyes)
[Person] Alicia (Spanish, 2 eyes)
[Person] Ragetti (English, 1 eye)
\end{cmd}

\solutioninput{../java/PersonTest.java}





\subsection{}

Write an \verb+Animal+ class that represents animals.

An animal will have a name and a number of wings, from 0 to 4, being 0 the
default number of wings in an animal.

The \verb+toString()+ method must return a string with the following
format:

\verb+    [Animal] name (numWings)+

\textsl{Practice with the refactoring tools of your editor so you can write
this class by copying the person class and modifying a few things in a
moment.}

\solutioninput{../java/Animal.java}






\subsection{}

Write a program called \verb+AnimalTest+ that creates 3 animals and prints
them.

Use the following data for your animals:

\begin{itemize}

  \item Pepa Pig, 0 wings.

  \item Maya the bee, 4 wings.

  \item Durin's Bane, 0 or 2 wings, depending on your geekiness.

\end{itemize}

\begin{cmd}
$ java PersonTest
[Animal] Pepa Pig (0)
[Animal] Maya the bee (4)
[Animal] Durin's Bane (2)
\end{cmd}

\solutioninput{../java/AnimalTest.java}










\subsection{}

Write a class called \verb+Pet+. Pets are animals with an owner (owners will be
objects of the class \verb+Person+).

The \verb+toString()+ method of pets will return a string with the following
format:

\verb+/animal/, owned by /owner/+

Where \verb+/animal/+ is the string representation of the animal part of the
pet and \verb+/owner/+ is the string representation of the owner. For example:

\begin{cmd}
[Animal] Gardfield (0 wings), owned by [Person] Jon Arbuckle (USA, 2 eyes)
\end{cmd}

\solutioninput{../java/Pet.java}

\subsection{}

Write a program called \verb+PetTest+ that creates 2 pets and prints
them.

Use the following data for your animals:

\begin{itemize}

  \item Gardfield, 0 wings, owned by Jon Artbucke (North American).

  \item Odie, 0 wings, owned by the same owner as Gardfield.

\end{itemize}

\begin{cmd}
$ java PetTest
[Animal] Garfield (0 wings), owned by [Person] Jon Arbuckle (USA, 2 eyes)
[Animal] Odie (0 wings), owned by [Person] Jon Arbuckle (USA, 2 eyes)
\end{cmd}

\solutioninput{../java/PetTest.java}









\subsection{}

Write a class called \verb+Parrot+. Parrots are pets with 2 wings and can talk.

The \verb+talk()+ method of parrots will return the \verb+name+ of its owner,
twice.

The \verb+toString()+ method of parrots will return a string with the
following format:

\verb+[Parrot] name, owned by /owner/+

Where \verb+/owner/+ is the string representation of its owner. For example:

\begin{cmd}
[Parrot] Alex, owned by [Person] Irene Pepperberg (USA, 2 eyes)
\end{cmd}

\solutioninput{../java/Parrot.java}


\subsection{}

Write a program called \verb+ParrotTest+ that creates a parrot and prints
the parrot and also makes it talk.

Use the following data for your parrot:

\begin{itemize}

  \item Alex, owned by Irene Pepperberg.

\end{itemize}

\begin{cmd}
$ java ParrotTest
[Parrot] Alex, owned by [Person] Irene Pepperberg (USA, 2 eyes)
Irene Pepperberg, Irene Pepperberg!
\end{cmd}

\solutioninput{../java/ParrotTest.java}








\subsection{}

Write a class called \verb+Pirate+ that represent pirates.

Pirates are people, with one eye and a parrot.

The \verb+toString()+ method of pirates return a string with the following
format:

\verb+    \person\, owner of \parrot\+

Also write a test program for your class where you create a pirate, print it
and them asks his parrot to talk.

You can use the following data for your pirate:

\begin{itemize}

  \item Cotton, english, owns a parrot called ``Cotton's parrot'' (nobody knows the real name of the parrot, as Cotton is dumb).

\end{itemize}

\textsl{Tip: A program like this will never work. Explain the error you get and
why are you getting it.  Then fix the bug so you can print as much information
about the pirate and his parrot, in a safe way.}

\solutioninput{../java/Pirate.java}
\solutioninput{../java/PirateTest.java}









\subsection{}

Write a class called \verb+EsprocendasPirate+. Everybody knows Espronceda's
pirates do not have a nationality.

Fix your code so you can create Espronceda's pirates and create a test program
for them.

\solutioninput{../java/EsproncedasPirate.java}
\solutioninput{../java/EsproncedasPirateTest.java}

\textsl{Tip: the simpler way to solve this problem is to allow for null
nationalities in the class \texttt{Person}.}









\subsection{}

Draw the UML diagram for the classes above (Person, Pirate, EsproncedasPirates,
Animal, Pet and Parrot).

\begin{solution}
  \centering
    \includegraphics[width=16cm]{img/pirates_uml/pirates_uml.eps}
\end{solution}







\subsection{}

Answer the following questions:

\begin{enumerate}

  \item Will it be a problem for the previous exercises to declare the
    \verb+Parrot+ class as a final class? Why?

  \item Will it be a problem for the previous exercises to declare the
    \verb+Person+ class as a final class? Why?

  \item Will it be a problem for the previous exercises to declare the
    \verb+toString+ method of class \verb+Pirate+ as final?, Why?

  \item Will it be a problem for the previous exercises to declare the
    \verb+toString+ method form class \verb+Pet+ as final?, Why?

  \item Will it be a problem for the previous exercises to declare the
    \verb+numWings+ attribute of the \verb+Animal+ class as final? Why?

\end{enumerate}

\begin{solution}
  1. No, because no other class is extinding it.
  2. Yes, because we will not be able to extend from it in Pirate.
  3. No, because we are not overriding that method in any subclass.
  4. Yes, because we are overriding that method in the Parrot subclass.
  5. Yes, but that has nothing to do with inheritance.
\end{solution}













\section{Classes with a fixed sets of instances.}

Another way to solve the problem associated with null nationalities is by
creating a \verb+Nationality+ class like this one:

\codeinput{../java/Nationality.java}{Nationality.java}

Write a NEW version of the \verb+Person+ class, called \verb+Person2+, that
uses this \verb+Nationality+ class.

Also write a \verb+Person2Test+ program that creates the following people and
prints them:

\begin{itemize}

  \item Alberto, Spanish

  \item John, North American

  \item Andrea, Italian

  \item Pierre, French

  \item Jack Sparrow, no nationality

  \item Batman, unknown

\end{itemize}

The output of the program should look like this:

\begin{cmd}
$ java Person2Test
[Person2] Alberto (Spanish, 2 eyes)
[Person2] John (North American, 2 eyes)
[Person2] Andrea (Italian, 2 eyes)
[Person2] Pierre (unsupported, 2 eyes)
[Person2] Jack Sparrow (no nationality, 2 eyes)
[Person2] Batman (unknown, 2 eyes)
\end{cmd}

\textsl{Pro-tip: there is another way in Java to create this kind of classes,
called \texttt{enum}s. Enums are not part of this course, but you can learn
more about them
\href{http://docs.oracle.com/javase/tutorial/java/javaOO/enum.html}{here}.}

\solutioninput{../java/Person2.java}
\solutioninput{../java/Person2Test.java}
















\section{Parents, grand-parents and bloodlines}

\subsection{}

Write a class called \verb+Child+. A child is a person2 with a parent, that is
another person2.

The \verb+toString+ method of this class must return a string with the
following format:

\verb+    name, son of /person2/+

Also, write a program to test your child class, using the following bloodline:

\begin{itemize}

  \item Aragorn, Dúnedain, son of Arathorn

  \item Arathorn, Dúnedain, son of Arador

  \item Arador, Dúnedain, son of Argonui

  \item Argonui, Dúnedain

\end{itemize}

\begin{cmd}
$ java ChildTest
Aragorn, son of Arathorn, son of Arador, son of [Person2] Argonui (Dúnedain, 2 eyes)
\end{cmd}

\solutioninput{../java/Child.java}
\solutioninput{../java/ChildTest.java}











\subsection{}

Draw the objects in the heap at the end of the previous program.

\begin{solution}
  \includegraphics[width=14cm]{./img/aragorn/aragorn.eps}
\end{solution}













\end{document}
