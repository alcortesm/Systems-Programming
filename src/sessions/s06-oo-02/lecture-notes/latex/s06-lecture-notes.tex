\documentclass[a4paper, 9pt]{extarticle}

\usepackage[notes]{style}

\newcommand{\realtitle}{Session 06 - Object Orientation 2}

\begin{document}

\makebox[\linewidth]{\rule{\textwidth}{0.4pt}}
UC3M \hfill Alberto Cortés Martín\\
Systems Programming, 2014-2015 \hfill version: \today\\
\makebox[\linewidth]{\rule{\textwidth}{0.4pt}}
\begin{center}
  \Large{\realtitle}\\Lecture Notes
\end{center}
\makebox[\linewidth]{\rule{\textwidth}{0.4pt}}
\vspace{1cm}

\section{Last Session Review}

\begin{blackboard}
Programming Styles
Object-Oriented Programming
  Advantages
    information hiding
    type safety
    encapsulation
  Disadvantages
Difference between Classes and Programs
State and Behavior
Attributes
  Default initialization
  Private
  Public
  Final
  Static
    Counting objects
Constructors
  Public
  Overloading
  This
  Default
  Private
    Utility classes
Methods
  Pubic
  Private
  Static
Shadowing
\end{blackboard}









\section{Today's Topics}
\begin{blackboard}
Castings
Inheritance
\end{blackboard}









\section{Castings}

Take a look at this Java code, do you see something weird?

\begin{blackboard}
int a = 3;
long b = a;
\end{blackboard}

This code should not even compile, because \verb+a+ and \verb+b+ are variables
of different types.

But it does compile, and what you end up having in \verb+b+ is \verb+3L+.

Another example of the same thing:

\begin{blackboard}
long c = 3;
\end{blackboard}

This is called an \textbf{implicit} casting: turning a small variable into a
bigger one.

This transformation is performed automatically by Java, it is a kind of
SYNTACTIC SUGAR: it is a small violation of the Java syntanxis, but makes the
life of the programmer sweeter.

Please note, there is no way this transformation can go wrong because small
values will always fit into bigger variables, just by filling the remaining
space with zeros, like this:

\begin{center}
  \includegraphics{./img/implicit_casting/implicit_casting.eps}
\end{center}

Now, the opposite can be quite dangerous.

It is OK with small numbers, but with big numbers, you never know what you will
get.

Infact, you would get a compiler error if you try to this. Which is nice,
because it is dangerous.

To be able to do this kind of transformation in Java, you have to convince Java
that you know what you are doing, by using a casting operator, like in the
following 2 examples:

\begin{blackboard}
long d = 3L;
int e = (int) d; // explicit casting, works, e is 3

long f = 3_000_000_000L;
int g = (int) f; // explicit casting, works, g is -1_294_967_296, is this what you want?
\end{blackboard}

This casting operator is a way to tell Java, \textsl{hey!, I know what I am
doing, so please just strip the left part of the number for me}.

This is called an \textbf{explicit} casting and it is a form of SYNTACTIC SALT:
it makes the life of the programmer harder, by forcing him to write something
extra, just to be sure he really knows he is doing a dangerous operation.

You can explicitly cast all primitive types into any other, except booleans.












\section{Inheritance}

You already know how to create new classes.

Today you will learn ANOTHER way of creating classes, called INHERITANCE.

\subsection{Version 1: print the biggest rectangle}

Let's say your boss asks you to write a method to return the rectangle with the
biggest area among a set of several rectangles.

For instance, which is the biggest rectangle?

\begin{itemize}
  \item A rectangle of 3x2
  \item A rectangle of 2x2
  \item A rectangle of 1x7
\end{itemize}

\begin{multicols}{2}
  \codeinput{../java/Max1.java}{Max1.java}
\columnbreak

  Let's list what kind of rectangle we need to solve the problem, using an UML
  diagram (Unified Modeling Language):

  \begin{center}
    \includegraphics[width=6cm]{./img/Rectangle1/Rectangle1.eps}
  \end{center}
  \codeinput{../java/Rectangle1.java}{Rectangle1.java}
\end{multicols}


\begin{blackboard}
$ java Max1
Rectangle(1.0, 7.0)
\end{blackboard}

\subsection{Version 2: Squares}

Now your boss asks you to change your code so you can also insert squares to
your method.

The simpler solution:

\verb+     Just create squares as rectangles, and we are done. I even made a constructor for that.+

\textsl{No, no, no. Your method will be part of a bigger program, written using
OOP. We need to differentiate between Squares and Rectangles, they are not the
same for the rest of your team.}

\textsl{Look, I will give you an array of Rectangles and Squares and you should
return the Square or Rectangle with the bigger area.}

For instance which of the following shapes has the biggest area?

\begin{itemize}
  \item A rectangle of 3x2
  \item A rectangle of 1x7
  \item An square of side 2
  \item An square of side 3
\end{itemize}

OK, let's write version 2 of our program:

\begin{multicols}{2}
  \codeinput{../java/Max2.java}{Max2.java}
\columnbreak
  \begin{center}
    \includegraphics[width=5cm]{./img/Square2/Square2.eps}
  \end{center}
  \codeinput{../java/Square2.java}{Square2.java}
\end{multicols}

As you can see, there is no way to insert Squares into our method, and even if
we could, we will be still returning Rectangles, not squares.

\newpage

\subsection{Version 3: Squares with Rectangles inside}

Let's try a clever trick, by creating a Square that holds a Rectangle inside.

We will be able to create a Square, but add the rectangle inside it to the
array.

Now, Square3 will be WRAPPER CLASS, it will just be a box holding a rectangle
inside. It will be a square on the outside, but a rectangle on the inside.

\begin{multicols}{2}
  \codeinput{../java/Max3.java}{Max3.java}
\columnbreak
  \begin{center}
    \includegraphics[width=5cm]{./img/Square3/Square3.eps}
  \end{center}
  \codeinput{../java/Square3.java}{Square3.java}
\end{multicols}

\begin{blackboard}
$ javac *3.java
$ java Max3
Rectangle(3.0, 3.0)
\end{blackboard}

Well, this kind of work.

But this is just a WORKAROUND, you are forcing your users to remember to call
the \verb+getRectangle()+ method before adding into the array, and what is
worst, the method returns a Rectangle, when it should return an Square.



\newpage
\subsection{Version 4: Squares are Rectangles}

What we really need is a class that behaves both as an Square and as a
Rectangle: we need a POLYMORPHIC class.

We want a class that sometimes behaves as an Square: when we create objects
and when we print them.

But other times, we want that class to behave as a Rectangle: when we add
objects to an array of Rectangles, for example.

In a similar way as when you turn an \verb+int+ into a \verb+long+ using
castings.

The way to do this is using INHERITANCE.

\begin{multicols}{2}
  \begin{center}
    \includegraphics[width=8cm]{./img/Square4/Square4.eps}
  \end{center}
  \codeinput{../java/Square4.java}{Square4.java}
\columnbreak
  \codeinput{../java/Max4.java}{Max4.java}
\end{multicols}

\begin{blackboard}
$ javac *4.java
$ java Max4
Square(3.0)
\end{blackboard}

Now the program works.

\subsection{Protected attributes}

As the width and height attributes of Rectangles are private we can not use
them from our Square.

That is way we needed the \verb+getWidth+ method.

But we can change those attributes from private to protected, which means
public for the classes that inherit from me, but private for the rest.

\begin{multicols}{2}
  \codeinput{../java/Rectangle5.java}{Rectangle5.java}
\columnbreak
  \begin{center}
    \includegraphics[width=5cm]{./img/Square5/Square5.eps}
  \end{center}
  \codeinput{../java/Square5.java}{Square5.java}
\end{multicols}
\codeinput{../java/Max5.java}{Max5.java}

\begin{blackboard}
$ javac *5.java
$ java Max5
Square(3.0)
\end{blackboard}





\section{Inheritance Lingo}

Vocabulary:

\begin{itemize}

  \item A Square \textbf{INHERITS} all attributes and methods from Rectangle.

  \item A Square \textbf{EXTENDS} a Rectangle.

  \item A Rectangle is a \textbf{SUPERCLASS} of Square.

    \begin{itemize}

      \item or is a \textbf{BASE} class of Sqaure.

      \item or is a \textbf{PARENT} class of Square.

    \end{itemize}

  \item Square is a \textbf{SUBCLASS} of Rectangle.

    \begin{itemize}

      \item or is a \textbf{DERIVED} class of Rectangle.

      \item or is a \textbf{CHILD} class of Rectangle.

    \end{itemize}

  \item Inheritance implements an \textbf{IS-A} relationship: An Square is a
    Rectangle.  As opposed to version 3, where an Square \textbf{HAS-A}
    Rectangle.

    \begin{center}
      \includegraphics[width=8cm]{./img/is_a/is_a.eps}
    \end{center}

\end{itemize}







\section{Another Example: Vehicles}

\begin{center}
  \includegraphics[width=14cm]{./img/vehicle/vehicle.eps}
\end{center}









\section{Another Example: Exceptions}

\begin{center}
  \includegraphics[width=14cm]{./img/exceptions/exceptions.eps}
\end{center}













\section{Inheritance Rules}

\begin{enumerate}

  \item A subclass inherits all the attributes and all the methods from its
    superclass, but none of its constructors.

  \item You can only inherit from one class. Multiple inheritance is not
    allowed in Java.

  \item Java will implicitly cast subclass objects to superclass references.

    \verb+Rectangle r = new Square(2);+

  \item You can explicitly cast superclass objects to subclass references, but
    it is dangerous. The \verb+instanceof+ operator might help.

\begin{blackboard}
Rectangle5 max = biggest(array);
if (max instanceof Square5) {
  Square5 s = (Square5) r;
  // do something Square5 specific
} else {
  // do something Rectangle5 specific
}
// If your program is correctly designed
// you will never have to do this or use the instanceof operator
\end{blackboard}

  \item If you re-define an attribute from your superclass in your subclass,
    the superclass version get shadowed. This is called \verb+ATTRIBUTE HIDING+.

    You are hiding the attribute of the
    superclass behind the new attribute in the subclass.

  \item If you re-define a method from your superclass in your subclass, the
    superclass version get shadowed. This is called \verb+METHOD OVERRIDING+.

    You are hiding the implementation of the methods in the superclass with a
    new implementation in the subclass.

  \item Do not confuse \verb+METHOD OVERRIDING+ with \verb+METHOD OVERLOADING+.

  \item \verb+this+ is a reference to the object itself, \verb+super+ is a
    reference to the ``superclass'' version of the object itself.

  \item Even if you re-define attributes or methods from your superclass in
    your subclass, you can still access the superclass version using the
    \verb+super+ reference.

\begin{blackboard}
class Square5 extends Rectangle5 {
  private double width;

  public getSquareWidth() {
    return this.width;
  }

  public getRectangleWidth() {
    return super.width;
  }
}
\end{blackboard}

  \item A \verb+final+ class can not be a subclassed.

  \item A \verb+final+ method in a superclass can not be overrided on
    subclasses.

  \item Constructors for subclasses:

    \begin{itemize}

      \item you must initialize the SUPER part of the new object before
        initializing the SUB part. This is done using the \verb+super+
        constructor.

      \item the call to the super constructor must be the first line in your
        constructor.

      \item If you don't call the super constructor, Java will silently insert
        a call to an empty super constructor for you.

    \end{itemize}

  \item If a class does not inherit \verb+EXPLICITLY+ from another class, then
    it does inherit \verb+IMPLICITLY+ from the \verb+Object+ class.

  \item This defines an class hirarchy.

    \begin{center}
      \includegraphics[width=10cm]{./img/object/object.eps}
    \end{center}

  \item This means all Java classes inherits from \verb+Object+ (except
    \verb+Object+ itself), and therefore all Java Objects have all the methods and
    attributes of an Object:

      \begin{itemize}

        \item String toString();

        \item 10 more you don't need to know for this course.

      \end{itemize}

\end{enumerate}








\newpage

\section{Access Modifiers}

ESTO NO DA TIEMPO, LO DEJO PARA LA SIGUIENTE CLASE.

\begin{center}
  \begin{tabular}{|p{3cm}|p{4cm}|p{4cm}|p{4cm}|}
\hline
\multicolumn{1}{|c|}{access mod.} & \multicolumn{1}{|c|}{class} & \multicolumn{1}{|c|}{method} & \multicolumn{1}{|c|}{attribute} \\
\hline
\texttt{public}    & \multicolumn{3}{|c|}{Accesible to all classes.} \\
\hline
\textsl{(empty) AKA friendly AKA package-protected} & \multicolumn{3}{|c|}{Accesible only to classes in the same package.} \\
\hline
\texttt{protected}   & Only for inner classes. & \multicolumn{2}{|c|}{Acecsible only to subclasses.} \\
\hline
\texttt{private}     & Only for inner classes. & \multicolumn{2}{|c|}{Accessible only inside the class.} \\
\hline
\hline
\texttt{final}       & Cannot be subclassed & Cannot be overriden & Cannot be modified once initialized \\
\hline
\texttt{static}      & Turns an inner class into a statically nested class. & Class method (instead of an object method) & Class attribute (instead of an object attribute) \\
\hline
\texttt{abstract}    & Cannot be instantiated & Has no code & --- \\
\hline
\end{tabular}
\end{center}








\section{Review}

\begin{blackboard}
Inplicit castings
Explicit castings
Syntactic salt
Has-a vs. is-a
Inheritance
UML diagrams
Protected access modifier
Inheritance rules
instanceof operator
Final classes
Final methods
Super automatic reference
Super() constructors
Attribute hiding
Method overriding
Exception use inheritance
Object is the father of all of objects
\end{blackboard}


\end{document}
