\documentclass[a4paper, 9pt]{extarticle}

\usepackage[notes]{style}

\newcommand{\realtitle}{Session 06 - Object Orientation 2}

\begin{document}

\makebox[\linewidth]{\rule{\textwidth}{0.4pt}}
UC3M \hfill Alberto Cortés Martín\\
Systems Programming, 2014-2015 \hfill version: \today\\
\makebox[\linewidth]{\rule{\textwidth}{0.4pt}}
\begin{center}
  \Large{\realtitle}\\Appendix 01
\end{center}
\makebox[\linewidth]{\rule{\textwidth}{0.4pt}}
\vspace{1cm}


\section{Inheritance Rules}

\begin{enumerate}

  \item A subclass inherits all the attributes and all the methods from its
    superclass, but none of its constructors.

  \item You can only inherit from one class. Multiple inheritance is not
    allowed in Java.

  \item Java will implicitly cast subclass objects to superclass references.

\verb+Rectangle r = new Square(2); // implicit casting+

  \item You can explicitly cast superclass objects to subclass references, but
    it is dangerous. The \verb+instanceof+ operator might help.

\begin{blackboard}
Rectangle4 max = biggest(array);
if (max instanceof Square4) {
  Square4 s = (Square4) r;
  // do something Square4 specific to s.
} else {
  // do something Rectangle4 specific to r.
}
// If your program is correctly designed
// you will never have to do this or use the instanceof operator
\end{blackboard}

  \item If you re-define an attribute from your superclass in your subclass,
    the superclass version gets shadowed. This is called \textbf{attribute hiding}.

    You are hiding the attribute of the
    superclass behind the new attribute in the subclass.

  \item If you re-define a method from your superclass in your subclass, the
    superclass version get shadowed. This is called \textbf{method overriding}.

    You are hiding the implementation of the method in the superclass with a
    new implementation in the subclass.

  \item Do not confuse method overriding with method overloading.

  \item \verb+this+ is a reference to the object itself, \verb+super+ is a
    reference to the ``superclass'' version of the object itself.

  \item Even if you re-define attributes or methods from your superclass in
    your subclass, you can still access their superclass version using the
    \verb+super+ reference.

  \item A \verb+final+ class cannot be a subclassed.

  \item A \verb+final+ method cannot be overridden.

  \item Constructors for subclasses:

    \begin{itemize}

      \item You must initialize the super part of the new object before
        initializing the sub part. The super part of the object is initialized
        using the \verb+super(...)+ constructors.

      \item The call to the super constructor must be the first line in your
        constructor.

      \item If you do not call the super constructor, Java will silently insert
        a call to an empty super constructor for you.

    \end{itemize}

  \item If a class does not inherit \textbf{explicitly} from another class,
    then it does inherit \textbf{implicitly} from the \verb+Object+ class.

    This defines a class hierarchy:

    \begin{center}
      \includegraphics[width=14cm]{./img/object/object.eps}
    \end{center}

  \item This means all Java classes inherit from \verb+Object+ (except
    \verb+Object+ itself), and therefore all Java objects have all the methods
    and attributes of the \verb+Object+ class, this is:

      \begin{itemize}

        \item \verb+public String toString()+

        \item and 10 more you do not need to know for this course.

      \end{itemize}

\end{enumerate}




\end{document}
