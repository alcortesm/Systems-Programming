\documentclass[a4paper, 9pt]{extarticle}

\usepackage[notes]{style}

\newcommand{\realtitle}{Session 14 - Lists}

\begin{document}

\makebox[\linewidth]{\rule{\textwidth}{0.4pt}}
UC3M \hfill Alberto Cortés Martín\\
Systems Programming, 2014-2015 \hfill version: \today\\
\makebox[\linewidth]{\rule{\textwidth}{0.4pt}}
\begin{center}
  \Large{\realtitle}\\Lecture Notes
\end{center}
\makebox[\linewidth]{\rule{\textwidth}{0.4pt}}
\vspace{1cm}


\section{Today's Topics}
\begin{blackboard}
Lists
\end{blackboard}


\section{What is a Collection?}

A collection is an object that groups multiple similar elements.

They are also called \textbf{containers}.

For example:

\begin{itemize}

  \item In your phone, your list of contacts \textbf{IS A} collection of friends and family members.

  \item Your email inbox IS A collection of unread emails.

  \item In Whatsapp, a conversation IS A collection of read and unread messages between you and a group of contacts (which is another collection of Whatsapp users).

  \item In a computer video game, the inventory of your character IS A collection of the objects your character has collects through the game.

  \item In a computer video game, your player achievements IS A collection of feats and challenges you have earned by playing the game.

  \item A bag of apples IS A collection of apples.

  \item A university course \textbf{HAS A} collection of students.

    In aula global, you can list all the students in a group by clicking on the ``participants'' link. (demo).

    Please note that a University course has also other things: a calendar, a collection of materials, a collection of teachers...

    Please, note that an Student is NOT a collection of Strings, event if you define an Student by its name, surname, and NIA, all of them being Strings.

    Those strings are the attributes of the Student, but you can not generally
    think of an Student as a bag of Strings, because you will not be adding
    Strings to a particular student along his life.

  \item In Swing a JPanel HAS A collection of JComponents (buttons, labels, other JPanels...).

  \item In Swing a JComponent (like a button for example) HAS A collection of
    ActrionListeners (usually only one, but you can add several listeners to
    the same button).

  \item An \verb+String[]+ is collection of Strings.

  \item An \verb+ArrayList<String>+ is a collection of Strings.

  \item The memory Stack of your computer is a collection of StackFrames.

\end{itemize}

Collections are extremely important, you will spend about 50\% of your time as a programer writing and using collections.

Actually, you will spend the rest of this course learning about collections.






\subsection{Operations on collections}

The minimum set of operations you will like to have on your collections are:

\begin{itemize}

  \item a \textbf{constructor}, that creates an empty collection

  \item an \textbf{add} method that inserts an element into your collection

  \item a \textbf{remove} method that extract an element from your collection

  \item a \textbf{size} method that tells you the number of elements in your collection

\end{itemize}

Then, there are a lot of other operations you might need, depending on your problem:

\begin{itemize}

  \item an \textbf{isEmpty} method that tells you if the collection has any element

  \item an \textbf{addAll} method that adds all the elements in a collection into another collection

  \item a \textbf{clear} method that removes all the elements in the collection

  \item a \textbf{contains} method that tells you if an element is already stored in the collection

  \item a \textbf{containsAll} method that tells you if all the elements in a collection are in another collection

  \item a \textbf{toArray} method that returns an array with all the elements in the collection

  \item \ldots the list is endless, and depends on your particular problem

\end{itemize}


\subsection{Types of collections}

\begin{itemize}

  \item There are \textbf{infinite} collections and collections \textbf{with a size limit}.

    \begin{itemize}

      \item your contacts list in your phone can have infinite friends (not really,
        the memory on your phone will impose a limit, the more photos you store in
        your phone, the less friends you can add your contact list, but there is
        no hard limit)

      \item but there cannot be more than 40 students in a university small group.

    \end{itemize}

    The add method in an infinte list will only return an exception when you
    run out of memory, but the add method on a small group will return an
    exception everytime you try to insert a student once you have 40.

  \item There are collections where you can add the same element several times,
    this is \textbf{duplicated elements are allowed} and collections that
    \textbf{doesn't allow duplicated elements}:

    \begin{itemize}

      \item You and your friends go to a restaurant, your order can have the same meal several times.

      \item But there cannot be suplicated students in a university course.

    \end{itemize}

    The add method in an order object should insert meal even if they are
    already there, but the add method in a university course should not insert
    a student if she is already there, it may even throw an exception.

  \item There are \textbf{ordered} collections where you store and retrieve elements by its index and \textbf{unordered} collections.

    \begin{itemize}

      \item People waiting in a queue to get into the cinema are sorted in the same
        order in which the get there. It is pretty clear how is the first one, how
        is the second one, and how is the last one.

      \item Your wardrobe is a collection of clothes, but they are not
        numbered, it makes no sense to get the 5th thing in your wardrobe, you
        will have to look for what you want to wear everytime you open your
        wardrobe.

    \end{itemize}

    You might have a get(int i) method in the cinema queue, but it makes no
    sense to address clothes in your wardrobe by a number.

  \item There are collections in which you \textbf{can insert nulls} and collections in which \textbf{you cannot}.

    \begin{itemize}

      \item The collection of NIAs of the students in a course can not have nulls, because all students must have a NIA

      \item But the collection of surnames of students in a course can have
        nulls, for example, Russian students do not have a second surname.
        Probably it will be better to add empty strings as their surnames,
        instead of nulls. Using nulls or empty strings is a very personal
        decission for the grammer.

    \end{itemize}

    The add method to the collection of NIA should throw an exception if the
    NIA string is null. But the add method to the collection of surnames might
    be OK with nulls.

\end{itemize}






\section{Lists}

A list is a especial type of collection.

Lists are also called sequences.

A lists is an ordered collection: items are added and retrieved by their index.

Usually, lists are infinite. You can add as many elements to a lists as you want.

Usually, lists can have duplicated elements.

Usually, lists don't allow nulls.

A nice and small interface for a list can be the following one:

\codeinput{../java/List.java}{List.java}

Please, don't confuse this List interface with the \verb+java.util.List+
interface, which is an interface already defined in Java to represents lists.

We are not going to use the Java one, because it has somethings too advanced
for you.

You can think of this interface as the little brother of the Java one. This one
is much more simple, yet still useful.

\subsection{Simple operations on a list}

\blackboardinput{list_usage.txt}


REPARTIR HANDOUT














\section{Array implementation of a list}

We need a class that implements our List interface, using an array.

We will call that class \verb+ArrayList+, because it is a List implemented with an array.

Please, do not confuse this \verb+ArrayList+ class with \verb+java.util.ArrayList+ that you already now.

Again this class is going to be like the litle brother of the real one.

\codelinesinput{../java/ArrayList.java}{ArrayList.java}{10}{19}

Our class implements the \verb+List+ interface.

We will need an array and a counter of elements for the size, to keep track of
how many elements are stored in the array.

We will also need some constants for the maximum length of the array, the
minumum and the initial default length.

Now, let's write the constructor:

\codelinesinput{../java/ArrayList.java}{ArrayList.java}{21}{30}

Now, let's write some simple methods:

\codelinesinput{../java/ArrayList.java}{ArrayList.java}{32}{38}

\codelinesinput{../java/ArrayList.java}{ArrayList.java}{85}{100}

\codelinesinput{../java/ArrayList.java}{ArrayList.java}{150}{166}

Now let's try something a little more cool, let's try to program the add method:

Remember, adding will insert a new element, shifting the already existing elements after it, to its right.

Also, adding elements will force the array to grow when needed.

\codelinesinput{../java/ArrayList.java}{ArrayList.java}{63}{83}

\codelinesinput{../java/ArrayList.java}{ArrayList.java}{40}{61}


Now let's write the remove method.

Remember that the remove methods, extract an element, shiftting all the elements to its right, one position to the left.

Also, removing lots of elements will force the array to shrink.

\codelinesinput{../java/ArrayList.java}{ArrayList.java}{132}{148}

\codelinesinput{../java/ArrayList.java}{ArrayList.java}{113}{130}







\section{Linked List implementation of a List}

In a linked lists, each element of the list is stored in a node.

Nodes are not contiguous in memory, they are individually allocated when nedded.

All nodes in a list are chained to its previous one and to its next one, forming a long chain of nodes.

\begin{center}
  \includegraphics[width=14cm]{./img/nodes/nodes.eps}
\end{center}

Node.
\codelinesinput{../java/LinkedList.java}{LinkedList.java}{9}{25}

Attributes and constructor.

\codelinesinput{../java/LinkedList.java}{LinkedList.java}{27}{33}

Size and isEmpty.

\codelinesinput{../java/LinkedList.java}{LinkedList.java}{35}{41}

Get and sets.

\codelinesinput{../java/LinkedList.java}{LinkedList.java}{43}{55}
\codelinesinput{../java/LinkedList.java}{LinkedList.java}{76}{85}

Index and contains.

\codelinesinput{../java/LinkedList.java}{LinkedList.java}{109}{134}

Add.

\codelinesinput{../java/LinkedList.java}{LinkedList.java}{57}{74}

Remove.
\codelinesinput{../java/LinkedList.java}{LinkedList.java}{87}{102}







\section{Array vs. Linked list implementations}

\begin{itemize}

  \item Arrays:

    \begin{itemize}

      \item[GOOD] has random access: any position in the array can be accessed in constant time.

      \item[GOOD] has low memory fragmentation: all elements are stored in a big chunk of memory, instead of using a lot of small chunks.

      \item[BAD] need continuous memory: it may not be available due to memory fragmentation.

      \item[BAD] present an innefficient use of memory when they are not full (when the length of the array is greater than the list size).

      \item[BAD]  add and remove require to move all subsequent elements, this is extremely slow.

    \end{itemize}

  \item Linked Lists:

    \begin{itemize}

      \item[GOOD] add and remove can be performed on constant time if the previous node is already known.

      \item[GODD] There is no need for contigous memory.

      \item[GOOD] efficient use of memory, as there are not more nodes than nedded.

      \item[BAD] accessing the ith element takes time proportional to i. This is a very common operation and is extremelly slow.

      \item[BAD] Each node is an overhead in memory occupation.

    \end{itemize}

\end{itemize}


\end{document}
