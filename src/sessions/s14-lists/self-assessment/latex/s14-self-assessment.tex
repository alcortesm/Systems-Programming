\documentclass[a4paper, 11pt]{article}

\usepackage[assessment]{style}

\newcommand{\realtitle}{Session 14 - Lists}

\begin{document}

\makebox[\linewidth]{\rule{\textwidth}{0.4pt}}
UC3M \hfill Alberto Cortés Martín\\
Systems Programming, 2014-2015 \hfill version: \today\\
\makebox[\linewidth]{\rule{\textwidth}{0.4pt}}
\begin{center}
  \Large{\realtitle}\\Self-Assessment
\end{center}
\makebox[\linewidth]{\rule{\textwidth}{0.4pt}}
\vspace{1cm}

\section{List implementations}

\subsection{ArrayList}

Implement the following interface using an array in a class named \verb+ArrayList+ like the one you have seen in class.

\codeinput{../../lecture-notes/java/List.java}{List.java}

The \verb+toString+ method must return a string as shown in the example bellow:

\begin{cmd}
$ java ArrayListTest
{}
{A}
{A, B}
{A, B, C}
\end{cmd}

\subsection{LinkedList}

Implement the \verb+List+ interface with a linked list. The \verb+toString+ method must use the same notation as the previous exercise.


\section{ensureCapacity}

\subsection{}

Do some research on the \verb+ensureCapacity+ method from class \verb+java.util.ArrayList+.

Why it is a good idea to have this method?

\subsection{}

Explain why this method is not part of the \verb+java.util.List+ interface.


\subsection{}

Add an \verb+ensureCapacity+ method to your \verb+ArrayList+ class.


\section{Adding more methods to your lists}

\subsection{Add at the end}

Add a \verb+void add(E e)+ method to the \verb+List+ interface, that adds elements at the end of the list.

Modify your \verb+ArrayList+ and \verb+LinkedList+ to implement this method in
an efficient way (this is, both implementations must execute this method in a
constant time, no matter the size of the list).


\subsection{Add a list to your list}

Add a \verb+void add(int i, List<? extends E> list)+ to your \verb+List+ interface.

This method will add all the elements in the list \verb+list+ to the list on which you are calling that method.

The elements are added in the same order as they appear in the \verb+list+ list.

The original \verb+list+ list must not be modified in this process.

The generic argument \verb+<? extends E>+ is a generic argument called
\verb+<?>+, that can be any class that extends \verb+<E>+ or \verb+<E>+ itself.
This allows to add a \verb+List<Person>+ to another \verb+List<Person>+ but
also allows to add a \verb+List<Student>+ to a \verb+List<Person>+, as you
should expect.

Implement this method in the \verb+ArrayList+ and in the \verb+LinkedList+ in an efficient way.


\subsection{}

Can you add a \verb+LinkedList<E>+ to an \verb+ArrayList<E>+?

Why?

\subsection{Clone constructor}

Add a constructor to \verb+ArrayList+ and to \verb+LinkedList+ that receives a
list. The new list must have the same elements and in the same order as the
list passed as an input parameter.

Be carefull to do an efficient implementation of both methods.

\subsection{equals}

Add an \verb+equals+ method to your \verb+List+ interface and implement it in \verb+ArrayList+ and \verb+LinkedList+.

The method must return \verb+true+ if both lists have \verb+equals+ elements and in the same order.

This method should support comparing lists with different implementations and of types that extends from each other.


\subsection{Find the index of the last element}

Implement an \verb+int lastIndexOf(E e)+ that returns the last index of an element \verb+equals+ to \verb+e+.

Implement this method on both \verb+ArrayList+ and \verb+LinkedList+.

Explain which implementation is faster.


\section{Duplicates}

\subsection{Remove duplicates}

Implement a method \verb+removeDuplicates+ that removes all the duplicated elements from your list.

Implement this method on both \verb+ArrayList+ and \verb+LinkedList+.


\subsection{Lists that do not allow duplicates}

Extends your \verb+ArrayList+ and your \verb+LinkedList+ in two new classes that do not allow duplicates.

Explain the performance implications of this change.

\subsection{Overriding?}

Would it be a good idea to override the \verb+removeDuplicates+ method in the version of the lists that do not allow duplicates?

Why?

How would you override it?


\end{document}
