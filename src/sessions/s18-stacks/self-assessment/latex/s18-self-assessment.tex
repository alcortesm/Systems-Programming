\documentclass[a4paper, 11pt]{article}

\usepackage[assessment]{style}

\newcommand{\realtitle}{Session 18 - Stacks and Queues}

\begin{document}

\makebox[\linewidth]{\rule{\textwidth}{0.4pt}}
UC3M \hfill Alberto Cortés Martín\\
Systems Programming, 2014-2015 \hfill version: \today\\
\makebox[\linewidth]{\rule{\textwidth}{0.4pt}}
\begin{center}
  \Large{\realtitle}\\Self-Assessment
\end{center}
\makebox[\linewidth]{\rule{\textwidth}{0.4pt}}
\vspace{1cm}


\section{Stacks}

\subsection{}

Write a dummy implementation of the \texttt{ArrayStack} class with the minimum
methods and code for it to compile (this class must implement the
\texttt{Stack} interface seen during the lecture).

Also write a dummy constructor that receives its maximum capacity and a dummy
\verb+toString+ method.

\subsection{}

Repeat the last exercise for the \texttt{LinkedStack} class.

This class has infinite capacity.

\subsection{}

Finish the class \texttt{ArrayStack} until it passes all the tests in the
\verb+StackTest+ program privided with your lecture.

Your implementation must be size bounded: it must have a constructor that
receives the maximum capacity.

Write the \verb+toString+ method so that it prints stacks as follows:

\begin{verbatim}
    Stack {(top)}
    Stack {A, B, C (top)}
\end{verbatim}

You can use the \texttt{StackTest} program as follows:

\begin{cmd}
$ java StackTest A
\end{cmd}

The program will finish silently if your implementation is correct and will
output some error numbers if your implementation is wrong.

Read the source code of the test program to learn what the errors mean
and how to fix them.

Do not copy the \texttt{ArrayStack} class provided with the lecture, it will
spoil all the fun. Try to write your own class from scrach and only use the
provided one in case of extreme pain.

\subsection{}

Repeat the last exercise with the \texttt{LinkedStack} class.

You can use the \texttt{StackTest} program as follows:

\begin{cmd}
$ java StackTest L
\end{cmd}



\section{Queues}

\subsection{}

Write a dummy implementation of the \texttt{ArrayQueue} class with the minimum
methods and code for it to compile (this class must implement the
\texttt{Queue} interface seen during the lecture).

Also write a dummy constructor that receives its maximum capacity and a dummy
\verb+toString+ method.

\subsection{}

Repeat the last exercise for the \texttt{LinkedQueue} class.

This class has infinite capacity.

\subsection{}

Finish the class \texttt{ArrayQueue} until it passes all the tests in the
\verb+QueueTest+ program privided with your lecture.

Your implementation must be size bounded: it must have a constructor that
receives the maximum capacity.

Write the \verb+toString+ method so that it prints stacks as follows:

\begin{verbatim}
    Queue {(front)}
    Queue {A, B, C (front)}
\end{verbatim}

You can use the \texttt{QueueTest} program as follows:

\begin{cmd}
$ java QueueTest A
\end{cmd}

The program will finish silently if your implementation is correct and will
output some error numbers if your implementation is wrong.

Read the source code of the test program to learn what the errors mean
and how to fix them.

Do not copy the \texttt{ArrayQueue} class provided with the lecture, it will
spoil all the fun. Try to write your own class from scrach and only use the
provided one in case of extreme pain.

\subsection{}

Repeat the last exercise with the \texttt{LinkedQueue} class.

You can use the \texttt{QueueTest} program as follows:

\begin{cmd}
$ java QueueTest L
\end{cmd}



\section{Deques}

\subsection{}

Write a dummy implementation of the \texttt{ArrayDeque} class with the minimum
methods and code for it to compile (this class must implement the
\texttt{Deque} interface seen during the lecture).

Also write a dummy constructor that receives its maximum capacity and a dummy
\verb+toString+ method.

\subsection{}

Repeat the last exercise for the \texttt{LinkedDeque} class.

This class has infinite capacity.

\subsection{}

Finish the class \texttt{ArrayDeque} until it passes all the tests in the
\verb+DequeTest+ program privided with your lecture.

Your implementation must be size bounded: it must have a constructor that
receives the maximum capacity.

Write the \verb+toString+ method so that it prints stacks as follows:

\begin{verbatim}
    Deque {(head) (tail)}
    Deque {(head) A, B, C (tail)}
\end{verbatim}

You can use the \texttt{DequeTest} program as follows:

\begin{cmd}
$ java DequeTest A
\end{cmd}

The program will finish silently if your implementation is correct and will
output some error numbers if your implementation is wrong.

Read the source code of the test program to learn what the errors mean
and how to fix them.

Do not copy the \texttt{ArrayDeque} class provided with the lecture, it will
spoil all the fun. Try to write your own class from scrach and only use the
provided one in case of extreme pain.

\subsection{}

Repeat the last exercise with the \texttt{LinkedDeque} class.

You can use the \texttt{DequeTest} program as follows:

\begin{cmd}
$ java DequeTest L
\end{cmd}




\section{Counting elements}

\subsection{}

Add a \texttt{size} method to all the above interfaces (\texttt{Stack},
\texttt{Queue} and \texttt{Deque}) that returns the number of elements in each
collection.

Implement the \texttt{size} method in all your implementations in the most
efficient way for each one.





\section{Removing duplicates}

\subsection{}

Write a \texttt{RemoveDuplicates()} method for the:

\begin{itemize}

  \item \texttt{ArrayStack} class.
  \item \texttt{LinkedStack} class.
  \item \texttt{ArrayQueue} class.
  \item \texttt{LinkedDeque} class.

\end{itemize}





\section{Splitting the collections}

\subsection{}

Write a \texttt{split} method that removes the lower half of all the classes
above and return an instance of the opposite class (array if linked or the
other way around) with the missing elements.


\subsection{}

Write a \texttt{split} method for your \texttt{ArrayDeque} class that removes
and returns the lower half of the collection in a \texttt{LinkedStack} object.

\end{document}
