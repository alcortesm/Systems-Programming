\documentclass[a4paper, 9pt]{extarticle}

\usepackage[notes]{style}

\newcommand{\realtitle}{Session 18 - Stacks and Queues}

\begin{document}

\makebox[\linewidth]{\rule{\textwidth}{0.4pt}}
UC3M \hfill Alberto Cortés Martín\\
Systems Programming, 2014-2015 \hfill version: \today\\
\makebox[\linewidth]{\rule{\textwidth}{0.4pt}}
\begin{center}
  \Large{\realtitle}\\Lecture Notes
\end{center}
\makebox[\linewidth]{\rule{\textwidth}{0.4pt}}
\vspace{1cm}


\section{Today's Topics}
\begin{blackboard}
Stacks
  Array
  Linked
Queues
  Array
  Linked
Deques
  Array
  Linked
\end{blackboard}

REPARTIR HANDOUT.

\section{Stacks}

A stack is a type of collection with two main operations:

\begin{itemize}

  \item \verb+void push(E e)+ to add an element to the collection

  \item \verb+E pop()+ to remove an element from the collection

\end{itemize}

Adding and removing from a Stack is always performed from the same end, as
follows:

\begin{center}
  \includegraphics[width=8cm]{./img/stack/stack.eps}
\end{center}

That is why they are also called \verb+LIFO+s: last in, first out.

Stacks are ordered collections, usually bounded. They usually allow duplicates
and do not allow for nulls.

\codeinput{../java/Stack.java}{Stack.java}





\subsection{Operations on a stack}

\codeinput{stack_usage.txt}{Handout}





\subsection{Examples of stacks}

\begin{itemize}

  \item a stack of plates in your kitchen

  \item a stack of books on your table

  \item emplyees in a company, when reducing staff, the last ones to get in,
    are the first ones to get fired.

  \item the list of visited webpages in your browser, visiting a new webpage
    push() the previous webpage into the stack, pressing the "back button"
    opens the webpage returned by pop().

  \item the "undo" functionality of most programs, performing an action pushes
    the "undo" equivalent action into the stack. Pressing the undo button
    execute the action returned by pop.

  \item Balancing parenthesis in an expression: \verb+(()(()()))+

  \item Reverse polish notation in a calculator: \verb;2 * (((4 * 3) / 2) + 5) --> 2 4 3 * 2 / 5 + *;

  \item method calls in the Stack.

\end{itemize}





\subsection{ArrayStack}

When implementing a stack using an array you have to ask yourself two questions:

\begin{itemize}

  \item In what order are you going to store the elements of the array? (is the top going to be index 0 of the index of the last inserted element)

  \item How are you going to keep track of the "top" of the stack?

\end{itemize}

\begin{center}
  \includegraphics[width=8cm]{./img/arraystack/arraystack.eps}
\end{center}

\codelinesinput{../java/ArrayStack.java}{ArrayStack.java}{10}{13}

The constructor and \verb+isEmpty()+ are pretty simple:

\codelinesinput{../java/ArrayStack.java}{ArrayStack.java}{15}{32}

The \verb+peek()+ method is also very simple:

\codelinesinput{../java/ArrayStack.java}{ArrayStack.java}{53}{58}

The \verb+push()+ method is just a line plus some exceptions:

\codelinesinput{../java/ArrayStack.java}{ArrayStack.java}{34}{42}

The \verb+pop()+ method should be just a line, \verb+return array[top--]+,
except that we need another line to help the garbage collector.

\codelinesinput{../java/ArrayStack.java}{ArrayStack.java}{44}{51}









\subsection{LinkedStack}

When implementing a stack using a linked list, you have to ask yourself this question:

\begin{itemize}

  \item In which order is most efficient (is top = to first or the last)

\end{itemize}

\begin{center}
  \includegraphics[width=8cm]{./img/linkedstack/linkedstack.eps}
\end{center}

The first thing we need is a node class, and the attributes of the list:

\codelinesinput{../java/LinkedStack.java}{LinkedStack.java}{9}{19}

The default constructor is ok and \verb+isEmpty()+ is pretty simple:

\codelinesinput{../java/LinkedStack.java}{LinkedStack.java}{21}{23}

The \verb+peek()+ method is also very simple:

\codelinesinput{../java/LinkedStack.java}{LinkedStack.java}{41}{46}

The \verb+push()+ method is just a line plus some exceptions:

\codelinesinput{../java/LinkedStack.java}{LinkedStack.java}{25}{30}

The \verb+pop()+ is just removing the first node and returning its datum.

\codelinesinput{../java/LinkedStack.java}{LinkedStack.java}{32}{39}

\end{document}
