\documentclass[a4paper, 9pt]{extarticle}

\usepackage[notes]{style}

\newcommand{\realtitle}{Session 18 - Stacks and Queues}

\begin{document}

\makebox[\linewidth]{\rule{\textwidth}{0.4pt}}
UC3M \hfill Alberto Cortés Martín\\
Systems Programming, 2014-2015 \hfill version: \today\\
\makebox[\linewidth]{\rule{\textwidth}{0.4pt}}
\begin{center}
  \Large{\realtitle}\\Lecture Notes
\end{center}
\makebox[\linewidth]{\rule{\textwidth}{0.4pt}}
\vspace{1cm}


\section{Today's Topics}
\begin{blackboard}
Stacks
  Array
  Linked
Queues
  Array
  Linked
Deques
  Array
  Linked
\end{blackboard}

REPARTIR HANDOUT.

\section{Stacks}

A stack is a type of collection with two main operations:

\begin{itemize}

  \item \verb+void push(E e)+ to add an element to the collection

  \item \verb+E pop()+ to remove an element from the collection

\end{itemize}

Adding and removing from a Stack is always performed from the same end, as
follows:

\begin{center}
  \includegraphics[width=8cm]{./img/stack/stack.eps}
\end{center}

That is why they are also called \verb+LIFO+s: last in, first out.

Stacks are ordered collections, usually bounded. They usually allow duplicates
and do not allow for nulls.

\codeinput{../java/Stack.java}{Stack.java}





\subsection{Operations on a stack}

\codeinput{stack_usage.txt}{Handout}





\subsection{Examples of stacks}

\begin{itemize}

  \item a stack of plates in your kitchen

  \item a stack of books on your table

  \item emplyees in a company, when reducing staff, the last ones to get in,
    are the first ones to get fired.

  \item the list of visited webpages in your browser, visiting a new webpage
    push() the previous webpage into the stack, pressing the "back button"
    opens the webpage returned by pop().

  \item the "undo" functionality of most programs, performing an action pushes
    the "undo" equivalent action into the stack. Pressing the undo button
    execute the action returned by pop.

  \item Balancing parenthesis in an expression: \verb+(()(()()))+

  \item Reverse polish notation in a calculator: \verb;2 * (((4 * 3) / 2) + 5) --> 2 4 3 * 2 / 5 + *;

  \item method calls in the Stack.

\end{itemize}





\subsection{ArrayStack}

When implementing a stack using an array you have to ask yourself two questions:

\begin{itemize}

  \item In what order are you going to store the elements of the array? (is the top going to be index 0 of the index of the last inserted element)

  \item How are you going to keep track of the "top" of the stack?

\end{itemize}

\begin{center}
  \includegraphics[width=8cm]{./img/arraystack/arraystack.eps}
\end{center}

\codelinesinput{../java/ArrayStack.java}{ArrayStack.java}{10}{13}

The constructor and \verb+isEmpty()+ are pretty simple:

\codelinesinput{../java/ArrayStack.java}{ArrayStack.java}{15}{32}

The \verb+peek()+ method is also very simple:

\codelinesinput{../java/ArrayStack.java}{ArrayStack.java}{53}{58}

The \verb+push()+ method is just a line plus some exceptions:

\codelinesinput{../java/ArrayStack.java}{ArrayStack.java}{34}{42}

The \verb+pop()+ method should be just a line, \verb+return array[top--]+,
except that we need another line to help the garbage collector.

\codelinesinput{../java/ArrayStack.java}{ArrayStack.java}{44}{51}









\subsection{LinkedStack}

When implementing a stack using a linked list, you have to ask yourself this question:

\begin{itemize}

  \item In which order is most efficient (is top = to first or the last)

\end{itemize}

\begin{center}
  \includegraphics[width=8cm]{./img/linkedstack/linkedstack.eps}
\end{center}

The first thing we need is a node class, and the attributes of the list:

\codelinesinput{../java/LinkedStack.java}{LinkedStack.java}{9}{19}

The default constructor is ok and \verb+isEmpty()+ is pretty simple:

\codelinesinput{../java/LinkedStack.java}{LinkedStack.java}{21}{23}

The \verb+peek()+ method is also very simple:

\codelinesinput{../java/LinkedStack.java}{LinkedStack.java}{41}{46}

The \verb+push()+ method is just a line plus some exceptions:

\codelinesinput{../java/LinkedStack.java}{LinkedStack.java}{25}{30}

The \verb+pop()+ is just removing the first node and returning its datum.

\codelinesinput{../java/LinkedStack.java}{LinkedStack.java}{32}{39}









\section{Queues}

A queue is a collection where you store items before processing them, in the
same order as they appear.

It has two main methods:

\begin{itemize}

  \item \verb+void enque(E e)+: add an element at the end of the queue.

  \item \verb+E dequeue()+: remove the first element from the queue.

\end{itemize}

\begin{center}
  \includegraphics[width=8cm]{./img/queue/queue.eps}
\end{center}

That is why they are also called \verb+FIFO+s: first in, first out.

Queues are ordered collections, usually bounded. They usually allow duplicates
and do not allow for nulls.

\codeinput{../java/Queue.java}{Queue.java}







\subsection{Operations on a queue}

\codeinput{queue_usage.txt}{Handout}





\subsection{Examples of stacks}

\begin{itemize}

  \item A queue of people waiting to get into the cinema

  \item Waiting on some queue at the supermarket

  \item A production or an assembly line, where products are processed by each worker in a line.

  \item Computer buffers in general: example System.out is a buffered output.
    When you println to System.out your program is not really writting to the
    computer screen, it is adding string to a queue (a buffer) of strings that
    the OS will print later, when it has time.

  \item The Swing EDT: whenever an event is produced by the user (clicking a
    button), the event is enqueued in a queue, to be processed by the
    ActionListener when the EDT has time to do it.

  \item Any time you want to connect a "producer" of things with a consumer of things.

\end{itemize}



\subsection{ArrayQueue}

They are commonly know as BUFFERS.

When implementing a queue with an array you must consider:

\begin{itemize}

  \item Where is going to be the \verb+first+ and the \verb+last+.

  \item How to reuse the space left by the dequeue operations for new enqueues.

\end{itemize}

The answer is using a circular buffer.

\begin{center}
  \includegraphics[width=6cm]{./img/arrayqueue/arrayqueue.eps}
\end{center}

So, to imlement this, you will need an array, something to keep track of the
first element and something to keep track of the last element.

\codelinesinput{../java/ArrayQueue.java}{ArrayQueue.java}{13}{26}

To calculate the next element, you no longer can use \verb,i++,. In this case a
private \verb+next()+ method will do the trick:

\codelinesinput{../java/ArrayQueue.java}{ArrayQueue.java}{36}{38}

The \verb+isEmpty()+ is very simple. We will also need an \verb+isFull()+
method later:

\codelinesinput{../java/ArrayQueue.java}{ArrayQueue.java}{28}{34}

The \verb+front()+ method is very simple:

\codelinesinput{../java/ArrayQueue.java}{ArrayQueue.java}{70}{75}

The \verb+enqueue()+ method is just inserting at the tail, but we must remember
to initialize the first if the queue was empty:

\codelinesinput{../java/ArrayQueue.java}{ArrayQueue.java}{40}{53}

The \verb+dequeue()+ method has to take into account that when the queue is
empty, we must reset the \verb+first+ and \verb+last+ to \verb+-1+:

\codelinesinput{../java/ArrayQueue.java}{ArrayQueue.java}{55}{68}







\subsection{LinkedQueue}

When implementing a queue using a linked list, we need to consider the problem
that inserting or deleting from the head is very fast, but inserting or
deleting from the last element is very expensive.

You can fix that by keeping track od the last element in the list.

So you will need a Node, a first and a last:

\codelinesinput{../java/LinkedQueue.java}{LinkedQueue.java}{10}{20}

The \verb+isEmpty()+ is very simple:

\codelinesinput{../java/LinkedQueue.java}{LinkedQueue.java}{22}{24}

The \verb+front()+ method is very simple:

\codelinesinput{../java/LinkedQueue.java}{LinkedQueue.java}{52}{57}

The \verb+enqueue()+ method is just inserting at the tail, and remember to set
the first element if the queue was empty before the insertion:

\codelinesinput{../java/LinkedQueue.java}{LinkedQueue.java}{26}{38}

As before, the \verb+dequeue()+ method has to take into account that when the
queue is empty, we must reset the \verb+first+ and \verb+last+ to \verb+null+:

\codelinesinput{../java/LinkedQueue.java}{LinkedQueue.java}{40}{50}









\section{Deque}

A \verb+Deque+ is an acronym for "Double ended queue" and it is pronounced
"Deck".

A Deque is just a queue where you can add and remove from both ends.

For example, like a deck of cards.

\codeinput{../java/Queue.java}{Queue.java}



\subsection{ArrayDeque}

An Deque implemented as an array is extremely simple, it is just an ordinary
ArrayQueue, but with an additional \verb+addHead()+ and \verb+removeTail()+
methods, that are exactly the same as the queue ones, but with all the
movement mirrowed.

We have rename first and last to head and tail and add an extra \verb+prev+
method to calculate the previous index.

\codeinput{../java/ArrayDeque.java}{ArrayDeque.java}





\subsection{LinkedDeque}

A Deque implemented as a linked list is extremely simple, if you know some
tricks.

\begin{itemize}

  \item The trick of using a last reference solve the problem of adding at the
    tail, but removing from the last is still expensive because we need to
    transverse the collection to look for the previous one.

    The way to solve this is to use a "Doubly linked list", where each node
    holds a reference to its next AND ALSO TO ITS PREVIOUS.

  \item This complicates a lot the methods for adding and removing, as we must
    not only consider the special case when the list is empty, but when the
    list is small, as we will be adding and extracting near the other border
    pretty often.

    The best way to reduce the complexity of this methods is to add two
    additional nodes, at the beginning and at the end, called sentinels, that
    will prevent us from adding or removing near the borders.

\end{itemize}

\begin{center}
  \includegraphics[width=10cm]{./img/linkeddeque/linkeddeque.eps}
\end{center}

Using sentinels is a waste of memory, but makes your coding very simple:

You will need a Node class with prev and next, a head and a tail and the two
sentinels:

\codelinesinput{../java/LinkedDeque.java}{LinkedDeque.java}{7}{27}

The \verb+isEmpty()+ method is very simple:

\codelinesinput{../java/LinkedDeque.java}{LinkedDeque.java}{29}{31}

When adding we have to create a new node and set the 4 references:

\codelinesinput{../java/LinkedDeque.java}{LinkedDeque.java}{33}{49}

When removing we have to store the return value and fix the surrounding nodes:

\codelinesinput{../java/LinkedDeque.java}{LinkedDeque.java}{51}{69}



\end{document}

