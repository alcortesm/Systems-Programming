\documentclass[a4paper, 11pt]{article}

\usepackage[assessment]{style}

\newcommand{\realtitle}{Session 24 - Trees 2}

\begin{document}

\makebox[\linewidth]{\rule{\textwidth}{0.4pt}}
UC3M \hfill Alberto Cortés Martín\\
Systems Programming, 2014-2015 \hfill version: \today\\
\makebox[\linewidth]{\rule{\textwidth}{0.4pt}}
\begin{center}
  \Large{\realtitle}\\Self-Assessment
\end{center}
\makebox[\linewidth]{\rule{\textwidth}{0.4pt}}
\vspace{1cm}

\section{}

\subsection{}

Write 5 dummy implementations of the \texttt{OrderedDictionary} interface provided with this session source code.

Name these 5 implementations as follows: \texttt{ODArrayList}, \texttt{ODLinkedList}, \texttt{ODSortedLinkedList}, \\
\texttt{ODSortedArrayList} and \texttt{Bst}.

Do not bother implementing the methods yet, just write one-liners, throwing an exception or returning some made-up value.

For the \texttt{Bst} add also a \texttt{toStringLevelOrder} method.

\subsection{}

Work on the \texttt{ODArrayList} class until it pass all the checks made by the program \texttt{ODTest} provided with this session source code.

This dictionary should be implemented using an ordinary array list (not sorted).

To test this version you should invoke the test program using the \texttt{AL} argument, as follows:

\begin{cmd}
$ java ODTest AL

=========== Testing ARRAY LIST

=========== OK

\end{cmd}

When done, compare your version of \texttt{ArrayList} with the one provided with this session source code.

\subsection{}

Work on the \texttt{ODLinkedList} class until it pass all the checks made by the program \texttt{ODTest} provided with this session source code.

This dictionary should be implemented using an ordinary linked list (not sorted).

To test this version you should invoke the test program using the \texttt{LL} argument, as follows:

\begin{cmd}
$ java ODTest LL

=========== Testing LINKED LIST

=========== OK

\end{cmd}

When done, compare your version of \texttt{LinkedList} with the one provided with this session source code.

\subsection{}

Work on the \texttt{ODSortedLinkedList} class until it pass all the checks made by the program \texttt{ODTest} provided with this session source code.

This dictionary should be implemented using a sorted linked list.

To test this version you should invoke the test program using the \texttt{SLL} argument, as follows:

\begin{cmd}
$ java ODTest SLL

=========== Testing SORTED LINKED LIST

=========== OK

\end{cmd}

When done, compare your version of \texttt{SortedLinkedList} with the one provided with this session source code.

\subsection{}

Work on the \texttt{ODSortedArrayList} class until it pass all the checks made by the program \texttt{ODTest} provided with this session source code.

This dictionary should be implemented using a sorted array list and using binary search.

To test this version you should invoke the test program using the \texttt{SAL} argument, as follows:

\begin{cmd}
$ java ODTest SAL

=========== Testing SORTED ARRAY LIST

=========== OK

\end{cmd}

When done, compare your version of \texttt{SortedArrayList} with the one provided with this session source code.


\subsection{}

Work on the \texttt{Bst} class until it pass all the checks made by the program \texttt{ODTest} provided with this session source code.

This dictionary should be implemented using a binary search tree.

To test this version you should invoke the test program using the \texttt{BST} argument, as follows:

\begin{cmd}
$ java ODTest BST

=========== Testing BINARY SEARCH TREE

=========== OK

\end{cmd}

When done, compare your version of \texttt{Bst} with the one provided with this session source code.

\textsl{Tip: the remove method from a BST can be tricky. Consider working first in some helper methods so the main \texttt{remove} method can be written in only a few lines of code.}


\subsection{}

Write a program that counts the frequency of characters in a sentence and prints it out, example:

\begin{cmd}
$ java Frequency "a"
(a,1)

$ java Frequency "aa"
(a,2)

$ java Frequency "aaaaa"
(a,5)

$ java Frequency "a b c"
( ,2), (a,1), (b,1), (c,1)

$ java Frequency "a b c bb bb"
( ,4), (a,1), (b,5), (c,1)

$ java Frequency "Frequency is the number of occurrences of a repeating event per 
unit time.[1] It is also referred to as temporal frequency, which emphasizes the 
contrast to spatial frequency and angular frequency. The period is the duration 
of time of one cycle in a repeating event, so the period is the reciprocal of the
 frequency"
( ,53), (,,2), (.,2), (1,1), (F,1), (I,1), (T,1), ([,1), (],1), (a,16), (b,1), 
(c,14), (d,5), (e,42), (f,10), (g,3), (h,10), (i,17), (l,6), (m,5), (n,18), 
(o,17), (p,9), (q,5), (r,22), (s,12), (t,21), (u,10), (v,2), (w,1), (y,6), 
(z,1)
\end{cmd}

\textsl{Tip: remember there are 65.536 different characters in Java so using an
array to hold their frequencies can be an overkill, an ordered dictionary will
be much better.}

\solutioninput{../java/Frequency.java}
\end{document}
