\documentclass[a4paper, 9pt]{extarticle}

\usepackage[notes]{style}

\newcommand{\realtitle}{Session 24 - Trees 2}

\newcommand{\separator}{\begin{center}%
\noindent\makebox[\linewidth]{\rule{0.75\paperwidth}{0.4pt}}%
\end{center}}

\begin{document}

\makebox[\linewidth]{\rule{\textwidth}{0.4pt}}
UC3M \hfill Alberto Cortés Martín\\
Systems Programming, 2014-2015 \hfill version: \today\\
\makebox[\linewidth]{\rule{\textwidth}{0.4pt}}
\begin{center}
  \Large{\realtitle}\\Lecture Notes
\end{center}
\makebox[\linewidth]{\rule{\textwidth}{0.4pt}}
\vspace{1cm}

\begin{blackboard}
Interface Comparable
Modeling real life dictionaries
Dictionaries in computer science
Implementations:
 - Unsorted ArrayList
 - Unsorted LinkedList
 - Sorted Linked List
 - Sorted Array List
 - Binary Search Tree implementation
   + Definition
   + find
   + insert
   + remove
   + degenerated trees
 - Hash Table
\end{blackboard}


\section{The Comparable Interface}

Do you remember our old friend the Circle class?

\codeinput{../java/Circle1.java}{Circle1.java}

PROBLEM: add a method to the Circle class that returns if a Circle is bigger, equal or smaller than another circle.

What is the input arguments to that method?

What is the return type of that method?

What would you call such method?

\codelinesinput{../java/Circle2.java}{Circle2.java}{20}{38}

What is this program going to print when executed?

\separator

Do you remember our old friend the Point class?

\begin{center}
  \includegraphics[width=10cm]{./img/point/point.eps}
\end{center}

Can you think of a reasonable way to compare points?

It depends on the problem, but probably the only thing that makes sense for
comparing points is comparing their distance to the origin.

\separator

Do you remember the Person class?

\begin{center}
  \includegraphics[width=10cm]{./img/person/person.eps}
\end{center}

Can you think of a reasonalbe way to compare people?

Again it depends on the problem, but an alphabetical comparisson of their names
maight makes sense, or for other programs, you may compare the ages, so someone
older is ``bigger''.

Can you see the pattern here?

\separator

PROBLEM: write a program that alphabetically compare two strings and prints if
the first is smaller, equal or bigger than the second one.

Example: alcortes is SMALLER than alicia.

To compare two strings alphabetically you will have to compare the first
character of both string, then if they are equal, continue with the second and
so on an so forth.

Thankfully, the string class already has a compareTo method that does exactly
that, so the program is very simple to write:

\codeinput{../java/StringComparator.java}{StringComparator.java}

The String class has this method because it implements the \verb+Comparable<String>+ interface.

This means that strings has a \verb+compareTo(String other)+ that returns a
negative int, zero or positive depending on if this string is alphabetically
smaller, equal or bigger than the ``other'' string.

Behold the Comparable interface:

\begin{center}
  \includegraphics[width=4cm]{./img/comparable/comparable.eps}
\end{center}

A class that implements this interface means that if you have several objects
from that class, you can sort them from smaller to bigger.

For example, if you have a bunch of circles in a list, you can sort them from smaller to bigger.

Or if you have a bunch of persons in a class, you can sort them by names in alphabetical order.

So, the Circle2 class before allows to compare Circle2 objects with Circle2
objects, so let us state that property clearly to Java:

\codelinesinput{../java/Circle3.java}{Circle3.java}{1}{1}

Being able to sort objects is exceptionally important as we will see today during the class.

Sorting is one of the more important things a computer can do: ``Ordenador'' : que ordena.

Sorting is a fun and awesome topic, sadly we will not see sorting in this course, but
we are going to do extensive usage of the comparable interface for the rest of
the course.

\newpage

\section{Modeling real life dictionaries}

\begin{center}
  \includegraphics[width=12cm]{./img/realdict/realdict5.eps}
\end{center}

An class for a real life dictionary would be something like this:

\begin{center}
  \includegraphics[width=12cm]{./img/realdictinterface/realdictinterface.eps}
\end{center}







\section{Dictionaries in Computer Science}

Dictioaries in Computer Science are a generalization of real life dictionaries.

\begin{center}
  \includegraphics[width=10cm]{./img/ordereddictionary/ordereddictionary.eps}
\end{center}

To simplify things, let us consider only dictionaries that do not allow duplicated keys.

This means that keys are unique, if you try to insert a new pair key-value when
that key is already in the dictionary, the old value will be updated or an
exception can be throw.

Please note that keys are of type K, that implements Comparable<K>, meaning
keys can be ordered from smallest to the biggest.

For example the real life dictionary from before (that does not accept duplicated keys) would be:

\begin{center}
  \includegraphics[width=14cm]{./img/demodict/demodict.eps}
\end{center}

\subsection{More examples of dictionaries}

Counting cars in a garage:

\begin{blackboard}
OrderedDictionary<String, Integer>

"Mercedes", 5
"BMW", 3
"Ford", 7
...
\end{blackboard}

\separator

Students in Aual Global:

\begin{blackboard}
OrderedDictionary<Integer, Student>

10001111, "Alberto Cortes, age 22, NIA 10001111"
10002222, "Alicia Rodriguez, age 22, NIA 10002222"
10003333, "Alberto Cortes, age 35, NIA 10003333"
...
\end{blackboard}

\separator

Appointments in a calendar:

\begin{blackboard}
OrderedDictionary<Date, String>

5 of May, "My birthday"
25 of December, "Christmas"
19th of May, "PS exam!"
...
\end{blackboard}

\separator

Storing the textures of 3D objects in a computer game:

\begin{blackboard}
OrderedDictionary<Point, Texture>
\end{blackboard}

During the loading of a level in the game you will create a dictionary with all
the textures of all the objects in the game, using the location of the objects
as the key.

Whenever the player is looking somewhere, you will find the textures at the
points he is looking at and apply the textures to the objects there.  There is
a good reason for storing the entries by alphabeticall order of their keys: to
be able to find values quickly. We will come back to this later.



\newpage

\section{Implementation}

\subsection{Array List}

\begin{center}
  \includegraphics[width=14cm]{./img/unsortedarraylist/unsortedarraylist.eps}
\end{center}

\begin{blackboard}
insert(key, value) when key is not in the dictionary
  search for key                              n       (proportional to n)
  insert                                      c       (constant time)
                                             -----
                                              n

remove(key) when key is in the dictionary
  search for key                              n/2     (proportional to n/2)
  extract                                     c
  shift left                                  n/2
                                             ------
                                              n
\end{blackboard}

\begin{multicols}{2}
\begin{center}
  \includegraphics[width=7cm]{./img/efficiency/efficiency1.eps}
\end{center}
\columnbreak

Let say we are working with a dictionary with the people in Madrid: 3 million.

Let say that 100 of them are trying to access a web page.

This means that just trying to log in will require 1 minute and a half to each of them.

Can we make this faster?

Well, the are two main sources of delay are:

\begin{itemize}
  \item having to shift all the elements to the left when extracting
  \item having to search throw the whole list when inserting
\end{itemize}

Let us get rid of the first source of delay, how can we avoid having to shift
the remaining elements to the left when extracting?

\end{multicols}

\newpage

\subsection{Linked List}

\begin{center}
  \includegraphics[width=14cm]{./img/unsortedlinkedlist/unsortedlinkedlist.eps}
\end{center}

\begin{blackboard}
insert(key, value) when key is not in the dictionary
  search for key                              n
  insert                                      c
                                             ------
                                              n

remove(key) when key is in the dictionary
  search for key                              n/2
  extract                                     c
                                             ------
                                              n/2
\end{blackboard}

\begin{multicols}{2}
\begin{center}
  \includegraphics[width=7cm]{./img/efficiency/efficiency2.eps}
\end{center}
\columnbreak

It takes the same time to search for the key, but now we do not have to shift the
rest of the elements to the left when extracting, which means we are faster.

Can we do better?

Of course, let us take care of the remaining problem: having to search through the
whole dictionary when inserting or when extracting.

How can we do that?

Well, real life dictionaries have their entries sorted by key for a reason: it
is faster to search for keys if they are ordered, because you can stop
searching once you get to the portion of the dictionary that has bigger keys
than the one you are looking for.

\end{multicols}


\newpage

\subsection{Linked List - Sorted}

\begin{center}
  \includegraphics[width=14cm]{./img/sortedlinkedlist/sortedlinkedlist.eps}
\end{center}

\begin{blackboard}
insert(key, value) when key is not in the dictionary
  search for key                              n/2
  insert                                      c
                                             ------
                                              n/2

remove(key):
  search for key                              n/2
  extract                                     c
                                             ------
                                              n/2
\end{blackboard}

\begin{multicols}{2}
\begin{center}
  \includegraphics[width=7cm]{./img/efficiency/efficiency3.eps}
\end{center}
\columnbreak

By storing the elements in sorted order we are skipping a lot of checks when
searching because we now can stop looking for entries once we have reach the
portion of the dictionary that is bigger than our key.

The performance of this version should be much better, because of this early
escape. In reality, it is not so good as the extra check to see if the key we
are looking for is smaller than the rest of the dictionary is a heavy burden
that lowers the performance.

Can we do this even faster?

Yes, in a real dictionary, we you search for the definition of ``table'', you
do not start by the first page, and keep passing pages until you get to the
``T''.

You open the dictionary at its half, you see that you are looking at words that
begins by M, then you know ``table'' will be on the right half of the
dictionary. You have just skip $n/2$ of the entries in only one step.

Then you continue jumping to the half of the remaining, to skip anoter $n/4$.

And so on and so forth until you get to ``table'' in very few jumps.

This way of searching in ordered collections is called \emph{Binary Search}.

In a 1000 pages dictionary, searching from the beginning will require 1000
steps in the worst case, but searching using BS will require only 10 steps. The
performance gain is huge.

So, how do we do something like that in our implementation?
\end{multicols}



\newpage

\subsection{Array List - Sorted}

\begin{center}
  \includegraphics[width=14cm]{./img/sortedarraylist/sortedarraylist.eps}
\end{center}

\codelinesinput{../java/ODSortedArrayList.java}{ODSortedArrayList.java}{157}{177}
\begin{blackboard}
insert(key, value):
  search for key                              lg n
  shift to right                              n/2
  insert                                      1
                                             ------
                                              n/2

remove(key):
  search for key                              lg n
  extract                                     1
  shift to left                               n/2
                                             ------
                                              n/2
\end{blackboard}

\begin{multicols}{2}
\begin{center}
  \includegraphics[width=7cm]{./img/efficiency/efficiency4.eps}
\end{center}
\columnbreak

Nowadays, computers are optimized for moving data to adjacent positions and it
can be achieved must faster than jumping from node to node in a linked list, so
actually, the $n/2$ performance of this version has a very marginal
contribution, leading to an awesome performance.

But can we do better?

Yes, by not having to shift the elements of the dictionary when inserting or
extracting.

You already know that can do that by using a linked list, but then, you can not
access the middle element.

You need some kind of mix between the good things of an array (jumping at the
middle of some data range) and the good things of a linked list (not having to
shift remaining elements when inserting or extracting).

Is this even possible, how can we do that?

\end{multicols}



\newpage

\section{Binary Search Tree}

We can do that by using a Binary Search Tree, that will give us this amazing
performance:

\begin{center}
  \includegraphics[width=7cm]{./img/efficiency/efficiency5.eps}
\end{center}

A BST is a binary tree with one important property, called the binary search tree property:

\begin{blackboard}
  For all nodes,
        all the keys in its LEFT  subtree are SMALLER than the key of the node
    and all the keys of its RIGHT subtree are BIGGER  than the key of the node.
\end{blackboard}

\begin{center}
  \includegraphics[width=6cm]{./img/bstquiz/bstquiz1.eps}
  \includegraphics[width=6cm]{./img/bstquiz/bstquiz2.eps}
  \includegraphics[width=6cm]{./img/bstquiz/bstquiz3.eps}
  \includegraphics[width=6cm]{./img/bstquiz/bstquiz4.eps}
\end{center}

\begin{blackboard}
  no, yes
  yes, yes
\end{blackboard}

PROBLEM: print the last BST in IN-order.

PROBLEM: search by hand the key 11 in the last BST.

\subsection{Implementation}

\begin{center}
  \includegraphics[width=12cm]{./img/bstimpl/bstimpl.eps}
\end{center}

\subsection{Find}

\begin{center}
  \includegraphics[width=12cm]{./img/algofind/algofind.eps}
\end{center}

\begin{multicols}{2}
  \codelinesinput{../java/Bst.java}{Bst Class}{259}{268}
  \codelinesinput{../java/Bst.java}{Bst Node}{97}{99}
\columnbreak
  \codelinesinput{../java/Bst.java}{Bst Node}{79}{95}
\end{multicols}


\subsection{Insert}

\subsection{Extract}

\subsection{Unbalanced BST}






\end{document}
