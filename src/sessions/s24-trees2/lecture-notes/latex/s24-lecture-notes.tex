\documentclass[a4paper, 9pt]{extarticle}

\usepackage[notes]{style}

\newcommand{\realtitle}{Session 24 - Trees 2}

\newcommand{\separator}{\begin{center}%
\noindent\makebox[\linewidth]{\rule{0.75\paperwidth}{0.4pt}}%
\end{center}}

\begin{document}

\makebox[\linewidth]{\rule{\textwidth}{0.4pt}}
UC3M \hfill Alberto Cortés Martín\\
Systems Programming, 2014-2015 \hfill version: \today\\
\makebox[\linewidth]{\rule{\textwidth}{0.4pt}}
\begin{center}
  \Large{\realtitle}\\Lecture Notes
\end{center}
\makebox[\linewidth]{\rule{\textwidth}{0.4pt}}
\vspace{1cm}

\begin{blackboard}
Interface Comparable
Modeling real life dictionaries
Dictionaries in computer science
Implementations:
 - Unsorted ArrayList
 - Unsorted LinkedList
 - Sorted Linked List
 - Sorted Array List
 - Binary Search Tree implementation
   + Definition
   + find
   + insert
   + remove
   + degenerated trees
 - Hash Table
\end{blackboard}


\section{The Comparable Interface}

Do you remember our old friend the Circle class?

\codeinput{../java/Circle1.java}{Circle1.java}

PROBLEM: add a method to the Circle class that returns if a Circle is bigger, equal or smaller than another circle.

What is the input arguments to that method?

What is the return type of that method?

What would you call such method?

\codelinesinput{../java/Circle2.java}{Circle2.java}{20}{38}

What is this program going to print when executed?

\separator

Do you remember our old friend the Point class?

\begin{center}
  \includegraphics[width=10cm]{./img/point/point.eps}
\end{center}

Can you think of a reasonable way to compare points?

It depends on the problem, but probably the only thing that makes sense for
comparing points is comparing their distance to the origin.

\separator

Do you remember the Person class?

\begin{center}
  \includegraphics[width=10cm]{./img/person/person.eps}
\end{center}

Can you think of a reasonalbe way to compare people?

Again it depends on the problem, but an alphabetical comparisson of their names
maight makes sense, or for other programs, you may compare the ages, so someone
older is ``bigger''.

Can you see the pattern here?

\separator

PROBLEM: write a program that alphabetically compare two strings and prints if
the first is smaller, equal or bigger than the second one.

Example: alcortes is SMALLER than alicia.

To compare two strings alphabetically you will have to compare the first
character of both string, then if they are equal, continue with the second and
so on an so forth.

Thankfully, the string class already has a compareTo method that does exactly
that, so the program is very simple to write:

\codeinput{../java/StringComparator.java}{StringComparator.java}

The String class has this method because it implements the \verb+Comparable<String>+ interface.

This means that strings has a \verb+compareTo(String other)+ that returns a
negative int, zero or positive depending on if this string is alphabetically
smaller, equal or bigger than the ``other'' string.

Behold the Comparable interface:

\begin{center}
  \includegraphics[width=4cm]{./img/comparable/comparable.eps}
\end{center}

A class that implements this interface means that if you have several objects
from that class, you can sort them from smaller to bigger.

For example, if you have a bunch of circles in a list, you can sort them from smaller to bigger.

Or if you have a bunch of persons in a class, you can sort them by names in alphabetical order.

So, the Circle2 class before allows to compare Circle2 objects with Circle2
objects, so let us state that property clearly to Java:

\codelinesinput{../java/Circle3.java}{Circle3.java}{1}{1}

Being able to sort objects is exceptionally important as we will see today during the class.

Sorting is one of the more important things a computer can do: ``Ordenador'' : que ordena.

Sorting is a fun and awesome topic, sadly we will not see sorting in this course, but
we are going to do extensive usage of the comparable interface for the rest of
the course.

\newpage

\section{Modeling real life dictionaries}

\begin{center}
  \includegraphics[width=12cm]{./img/realdict/realdict5.eps}
\end{center}

An class for a real life dictionary would be something like this:

\begin{center}
  \includegraphics[width=12cm]{./img/realdictinterface/realdictinterface.eps}
\end{center}







\section{Dictionaries in Computer Science}

Dictioaries in Computer Science are a generalization of real life dictionaries.

\begin{center}
  \includegraphics[width=10cm]{./img/ordereddictionary/ordereddictionary.eps}
\end{center}

To simplify things, let us consider only dictionaries that do not allow duplicated keys.

This means that keys are unique, if you try to insert a new pair key-value when
that key is already in the dictionary, the old value will be updated or an
exception can be throw.

Please note that keys are of type K, that implements Comparable<K>, meaning
keys can be ordered from smallest to the biggest.

For example the real life dictionary from before (that does not accept duplicated keys) would be:

\begin{center}
  \includegraphics[width=14cm]{./img/demodict/demodict.eps}
\end{center}

\subsection{More examples of dictionaries}

Counting cars in a garage:

\begin{blackboard}
OrderedDictionary<String, Integer>

"Mercedes", 5
"BMW", 3
"Ford", 7
...
\end{blackboard}

\separator

Students in Aual Global:

\begin{blackboard}
OrderedDictionary<Integer, Student>

10001111, "Alberto Cortes, age 22, NIA 10001111"
10002222, "Alicia Rodriguez, age 22, NIA 10002222"
10003333, "Alberto Cortes, age 35, NIA 10003333"
...
\end{blackboard}

\separator

Appointments in a calendar:

\begin{blackboard}
OrderedDictionary<Date, String>

5 of May, "My birthday"
25 of December, "Christmas"
19th of May, "PS exam!"
...
\end{blackboard}

\separator

Storing the textures of 3D objects in a computer game:

\begin{blackboard}
OrderedDictionary<Point, Texture>
\end{blackboard}

During the loading of a level in the game you will create a dictionary with all
the textures of all the objects in the game, using the location of the objects
as the key.

Whenever the player is looking somewhere, you will find the textures at the
points he is looking at and apply the textures to the objects there.  There is
a good reason for storing the entries by alphabeticall order of their keys: to
be able to find values quickly. We will come back to this later.



\newpage

\section{Implementation}

\subsection{Array List - Unsorted}

\begin{center}
  \includegraphics[width=14cm]{./img/unsortedarraylist/unsortedarraylist.eps}
\end{center}

\begin{blackboard}
insert(key, value):
  search for key                              O(n)  (proportional to n)
  shift right and insert                      O(n)

                                              O(n)

remove(key):
  search for key                              O(n)
  extract                                     O(1)  (constant time)
  shift left                                  O(n)

                                              O(n)
\end{blackboard}

\begin{center}
  \includegraphics[width=7cm]{./img/efficiency/efficiency1.eps}
\end{center}

Let say we are working with a dictionary with the people in Madrid: 3 million.

Let say that 100 of them are trying to access a web page.

This means that just trying to log in will require 1 minute and a half to each of them.




\end{document}
