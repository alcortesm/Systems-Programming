\documentclass[a4paper, 9pt]{extarticle}

\usepackage[notes]{style}

\newcommand{\realtitle}{Session 04 - Object Orientation 1}

\begin{document}

\makebox[\linewidth]{\rule{\textwidth}{0.4pt}}
UC3M \hfill Alberto Cortés Martín\\
Systems Programming, 2014-2015 \hfill version: \today\\
\makebox[\linewidth]{\rule{\textwidth}{0.4pt}}
\begin{center}
  \Large{\realtitle}\\Lecture Notes
\end{center}
\makebox[\linewidth]{\rule{\textwidth}{0.4pt}}
\vspace{1cm}

\section{Last Session Review}

\begin{blackboard}
Constants (final)
  Magic numbers

References
  Declaration
  Operators

Objects
  Methods
    stack frames
    parameters
      automatic variables
      this
    return values
    exceptions
    static
    overloading
  Constructors
  Strings
    Syntactic sugar

Control Flow statements
  Conditionals
    if-then
      if-then-else
    switch
  Loops
    while
    do while
    for
    infinite loops
  Continue, break

Arrays
Dates
Limits of numerical types
Floating point are not exact
Interactive programs (42)
Unix filters
Generating random numbers
\end{blackboard}




\section{Today's Topics}
\begin{blackboard}
Programming Styles
Object-Oriented Programming
Difference between Classes and Programs
State and Behavior
Attributes
Constructors
Methods
Shadowing
\end{blackboard}










\section{Programming Styles or Paradigms}

\begin{center}
\includegraphics{./img/martial-arts/martial-arts.eps}
\end{center}

\begin{multicols}{2}
\setlength{\columnseprule}{0.4pt}

There are many styles of martial arts.

\begin{itemize}

  \item Boxing

  \item Karate

  \item Kung-Fu

  \item \ldots

\end{itemize}

\columnbreak

There are many styles of writing programs.

\begin{itemize}

  \item Imperative

  \item Object-Oriented

  \item Functional

  \item \ldots

\end{itemize}

\end{multicols}

Some programming languages are better suited for one style or another.

For example, Java is specifically designed for Object-Oriented programming.

This means writing Object-Oriented programs in Java is EASY,
but writing Functional programs in Java is HARD.

This also means solving some problems in Java is very EASY,
but other problems are quite HARD.

\textsl{Many companies would ask you to solve a problem in Java. WRONG: just
explain me the problem and I will see which language I use to solve it.}










\section{Object-Oriented Programming}

Or OOP for short.

Allow me to explain what Object Oriented programing is by explaining first what
it is not.

\begin{multicols}{2}
  \codeinput{../java/AreaOfCircle1.java}{Imperative Programming}
  \columnbreak
  \codeinput{../java/AreaOfCircle2.java}{Object-Oriented Programming}
\end{multicols}

\paragraph{Advantages of OOP}

\begin{description}

  \item[Code clarity] Solutions are written in the same terms as the
    description of the problem.

    You write about circles and areas, not mathematical expressions.

    This is known as \textbf{information hiding}.

    You are hiding the complexity behind a name, so you can think high level,
    without caring about the details of the low level.

  \item[Passing objects] Say your program is part of a bigger program, with
    squares and circles. Squares are just doubles (for its side), the same as
    circles. How do you know what concepts are represented by each double?

    By passing Circles and Squares around the meaning is much clearer.

    This is known as \textbf{type safety}.

  \item[Objects know about what actions can they do] Each object carries its
    own methods, telling you what you can do to them.

    You no longer need to know how to calculate the area of a circle, the
    circle already knows how to do it.

    You will never calculate the area of a Square over a Circle by mistake.

    This is known as \textbf{encapsulation}. An object is responsible for its data
    and how to operate on it.

  \item[Code reusability]

    Classes are libraries, which you can use and reuse.

    Say you write another program to calculate the length of the circle's
    circumference ($2*\pi*radius$).

    If not using OOP, you will write a new mathematical expression for that
    calculation, if you want to reuse that in other program, you will have to
    rewrite the expression, maybe with different variable names.

    In OOP you will simply add a new method to the circle class, and you will
    be able to use that new method from any program.

  \item[You are extending the language] This is no longer standard Java; it is
    your own version of Java that has a new Data Type, the circle.

    Your new version of Java is much better to solve the problem than the
    standard version of Java.

\end{description}

\paragraph{Disadvantages of OOP}

You will have to create your own data types, like circles for example.









\newpage
\section{My First Java Class}

Let's create a Java class to represent Circles.

\begin{blackboard}
class Circle {
  private double radius;

  public Circle(double radius) throws Exception {
    if (radius <= 0) {
      throw new Exception("Invalid radius");
    }
    this.radius = radius;
  }

  public double area() {
    return this.radius * this.radius * Math.PI;
  }
}
\end{blackboard}

This class represents Circles, so instead of using a double, we will be
creating objects from this class.

This class has an ATTRIBUTE (radius) a CONSTRUCTOR (Circle) and a METHOD (area()).

The purpose of a class is to be the template for creating objects from that
class.

DIBUJAR MEMORIA PARA EL PROGRAMA AreaOfCircle2.







\section{Classes and Programs}

A program is a Java class with an entry point.

An entry point is where a program begins executing.

In Java, that is the \verb+public static void main(String[] args)+ method.

\begin{center}
  \includegraphics{./img/classes-and-programs/classes-and-programs.eps}
\end{center}

\begin{multicols}{2}
\begin{blackboard}
class Circle { // Template for creating
               // objects in programs
  private double radius;

  public Circle(double radius) throws Exception {
    if (radius <= 0) {
      throw new Exception("Invalid radius");
    }
    this.radius = radius;
  }

  public double area() {
    return this.radius * this.radius * Math.PI;
  }
}
\end{blackboard}
\columnbreak
\begin{blackboard}
class Bla {  // a program
  public static void main(String[] args)  {

    System.out.println("Hello World!");

-->    try {
-->      Circle c = new Circle(Double.parseDouble(args[0]));
-->    } catch (Exception ex) {
-->      System.err.println(ex.getMessage);
-->      return;
-->    }
-->    double area = c.area();
-->    System.out.println(area);
  }
}
\end{blackboard}
\end{multicols}


\begin{multicols}{2}
\begin{blackboard}

  ...

  public static void main(String[] args)  {
    Syste.out.println("Hello World!");
  }

  ...

\end{blackboard}
\columnbreak
\begin{blackboard}

  ...

  public static void main(String[] args)  {
    Circle c;
    try {
      c = new Circle(Double.parseDouble(args[0]));
    } catch (Exception ex) {
      System.err.println(ex.getMessage);
      return;
    }
    double area = c.area();
    System.out.println(area);
  }

  ...

\end{blackboard}
\end{multicols}









\section{State and Behavior}

Objects have:

\begin{description}

  \item[State] what information objects have INSIDE.

    what is different between two objects of the
    same class.

    In Java, the state of an object is its ATTRIBUTES.

    Attributes are initialized by the CONSTRUCTORS.

    Examples:

\begin{blackboard}
  Point: x and y            // write the class declaration
  Circles: radius (or diameter or circunference) + [Point center]
  Squares: side + point left-top-corner and an angle / or its two top points
  Student: name, age, student id, courses they have paid tuition for
  King: name and country [ + age]
  Cars: plate, model, color, current speed, current direction
\end{blackboard}

  \item[Behavior] what objects can do.

    What orders you can give to them.

    In Java, the set of orders they can take are its METHODS.

\begin{blackboard}
  Circles: area(), circumference()
  Square: area(), perimeter()
  Student: addToCourse(), isInCourse(), passCourse(), failCourse()
  Kings: whatever kings do
  Cars: accelerate(), breakDown(), turnRight(), turnLeft()
\end{blackboard}
\end{description}













\section{Attributes and Constructors}

What information objects have.

Variables and references.

They are usually private, as you don't want anyone to set them to invalid values
(like a negative radius).

Constructors are used to initialize attributes.

\begin{blackboard}
class Circle {
  private double radius;
  private Point center;

  public Circle(double radius, Point center) {
    if (radius <= 0 || center == null) {
      throw new Exception("Invalid arguemnts");
    }
    this.radius = radius;
    this.center = center;
  }

  public Circle() {
  }
}
                    // DIBUJAR MEMORIA
\end{blackboard}

If they are not initialized by a constructor, they get initialized by default
to:

\begin{blackboard}
boolean: false
byte, short, int: 0
long: 0L
float: 0F
double: 0D
char: '\u0000'
references: null
\end{blackboard}

Difference between variables in the stack and attributes:

\begin{blackboard}
main:

  Circle a = new Circle(2D, new Point(2D,3D));
  double b;
  Circle c;
  Circle d = new Circle();

  // DIBUJAR MEMORIA
\end{blackboard}









\section{Methods}

They operate on the objects:

\begin{blackboard}
main:

 Circle a = new Circle(1D);
 Circle b = new Circle(2D);
 System.out.println(a.area()); // 3.14...
 System.out.println(b.area()); // 12.57...

     // DIBUJAR MEMORIA
\end{blackboard}











\subsection{Overloading Constructors}

You can overload them:

\begin{blackboard}
  ...

  public Circle() {          // empty constructor
    this.radius = 1D;
  }

  public Circle(String radius) {
    this.radius = Double.parseDouble(radius);
  }

  ...

main:
  Circle c1 = new Circle(); // 0.0
  Circle c2 = new Circle(5D); // 5.0
  Circle c3 = new Circle("7"); // 7.0
  Circle c4 = new Circle("-3"); // -3.0 <-- this is wrong
\end{blackboard}










\subsection{Calling constructors from other constructors}

You can call a constructor from another constructor, using the method named
\verb+this+.

\begin{blackboard}

  ...

  public Circle() {
    this(1D);
  }

  public Circle(String radius) {
    this(Double.parseDouble(radius));
  }

  ...

\end{blackboard}

It has to be the first line of code in your constructor, due to some Java
complicated magic.









\subsection{Default constructors}

When a Java Class has no constructor, Java automatically creates a default
constructor for you.

This default constructor does not have any parameters and does not do anything.

\begin{blackboard}

  ...

  public Circle() {
  }

  ...

\end{blackboard}


But it is useful because you can at least create objects from that class.

\begin{blackboard}
main:
  Circle c = new Circle(); // 0.0 <-- wrong in this case
\end{blackboard}









\section{Attributes: Public or Private}

A public attribute can be accesses from outside, for reading or writing.

They are part of the API of the object (what users need to know about your
class to use it):

\begin{blackboard}

  ...

  public double radius;

  ...

main:

Circle a = new Circle(); // circle of radius 0.0
System.out.println(a.radius); // 0.0
a.radius = 7D;
System.out.println(a.radius); // 7.0
a.radius = -3D;
\end{blackboard}

Because most times, you want to hide the INTERNALS of the objects from the
user.

Why should the user know if you represent circles by a radius, or with a
diameter or with a circumference?

Why should the user know what are the valid values for a radius.

Let the circle expert decide what is the best way to represent a circle in
memory; users don't care about those details.










\section{Final Attributes}

Final attributes cannot be changed once initialized.

\begin{blackboard}

  ...

  final private double DEFAULT_RADIUS = 1D;

  ...

  public Circle() {
    this(DEFAULT_RADIUS);
  }

  ...

\end{blackboard}








\section{Static Attributes}

Static attributes do not belong to the objects of the class but to the class.

You can access them using the name of the class in addition to the name of the
object.

\begin{blackboard}

  ...

  final static public double DEFAULT_RADIUS = 1D;

  ...                           // DIBUJAR MEMORIA

\end{blackboard}

All objects of the Circle class will have the same DEFAULT\_RADIUS, and there is only
one reference to it stored in the static section.

\begin{blackboard}

  ...

  public static int howMany;

  ...

  public Circle(double radius) {
    if (radius <= 0) {
      throw new Exception("Invalid radius");
    }

    this.radius = radius;
    Circle.howMany++;
  }

main:
  s.o.p(Circle.howMany); // 0
  Circle a = new Circle();
  s.o.p(Circle.howMany); // 1
  Circle b = new Circle();
  Circle a = new Circle();
  s.o.p(Circle.howMany); // 3
\end{blackboard}










\section{Private Methods}

Private methods are useful for splitting long public methods into meaningful
parts.

\begin{blackboard}

  ...

  public Circle(double radius) throws Exception {
    checkInvalidRadius(radius);
    this.radius = radius;
  }

  private void checkInvalidRadius(double radius) throws Exception {
    if (radius <= 0) {
      throw new Exception("Invalid radius");
    }
  }

  ...

\end{blackboard}










\section{Static Methods}

Static methods are useful for the class, not the objects:

- write a test program inside your class.

- other advanced uses: In general, you should not use them.







\section{Private Constructors}

Private constructors are useful to prevent users from creating objects.

\begin{blackboard}
Most Java programs.

class HelloWorld {
  private HelloWorld() {
  }

  public static void main(String[] args) {
    System.out.println("Hello World");
  }
}

Most utility classes, like this one:

class Math {
  private Math() { // private default constructor
  }

  public final static PI = 3.1415952265359;
  public final static E = 2.7182818284590;

  public static int abs(int a) {
    // ...
  }

  public static int max(int a, int b) {
    // ...
  }

  // ...
}

With Singletons: a singleton is a class that can only be instantiated once.
class TheOneRing {
  private static TheOneRing instance = null;

  private TheOneRing() {
  }

  public static TheOneRing getInstance() {
    if (TheOneRing.instance == null) {
      TheOneRing.instance = new TheOneRing();
    }
    return TheOneRing.instance;
  }
}

main:
  TheOneRing ring = new TheOneRing(); // compiler error private constructor
  TheOneRing ring = TheOneRing.getInstance();
  frodo.addToInventory(ring);
\end{blackboard}

Singletons are generally a very bad idea.

Even if in your program there is to be only one instance of a class, for
testing your program you may need several, so don't use singletons.








\section{Shadowing}

\begin{multicols}{2}
\begin{blackboard}
class Circle {
  private double radius;

  public Circle(double radius) {
    this.radius = radius;
  }
}
\end{blackboard}
\columnbreak
\begin{blackboard}
// This should not work, but it DOES work
class Circle {
  private double radius;

  public Circle(double r) {
    radius = r;    // shadowing
  }
}
\end{blackboard}
\end{multicols}

\begin{blackboard}
// This doesn't work
class Circle {
  private double radius;

  public Circle(double radius) {
    radius = radius;
  }
}
\end{blackboard}














\section{Review}

\begin{blackboard}
Programming Styles
Object-Oriented Programming
  Advantages
    information hiding
    type safety
    encapsulation
  Disadvantages
Difference between Classes and Programs
State and Behavior
Private Attributes
  Default initialization
Public Constructors
Public Methods
Overloading constructors
the this constructor
Default constructors
Public attributes
Final attributes
Static attributes
  Counting objects
Private methods
Static methods
Private constructors
  Utility classes
  Singletons
Shadowing
\end{blackboard}

\end{document}
