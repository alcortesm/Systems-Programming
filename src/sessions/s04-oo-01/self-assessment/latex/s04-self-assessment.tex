\documentclass[a4paper, 11pt]{article}

\usepackage[assessment]{style}

\newcommand{\realtitle}{Session 04 - Object-Oriented Programming 1}

\begin{document}

\makebox[\linewidth]{\rule{\textwidth}{0.4pt}}
UC3M \hfill Alberto Cortés Martín\\
Systems Programming, 2014-2015 \hfill version: \today\\
\makebox[\linewidth]{\rule{\textwidth}{0.4pt}}
\begin{center}
  \Large{\realtitle}\\Self-Assessment
\end{center}
\makebox[\linewidth]{\rule{\textwidth}{0.4pt}}
\vspace{1cm}


\section{Rectangles}

\subsection{}

Write a class called \verb+Rectangle1+ that stores the width and height of a
rectangle in public attributes.

It must have a constructor that receives the width and height of the rectangle.
This constructor should throw an exception if any of these
parameters are zero or negative.

\solutioninput{../java/Rectangle1.java}


\subsection{}

Write a program called \verb+Rectangle1TestException+ to test if creating a
\verb+Rectangle1+ with an invalid width or height throws an exception.

The program must print \verb+Exception caught!+ if the exception is caught.

If the exception is not caught, it must print \verb+Exception NOT caught+.
Modify the code of the program, changing the parameters to the rectangle
constructor, compile again and run it, to test both situations.

\solutioninput{../java/Rectangle1TestException.java}




\subsection{}

Make the necessary changes to your \verb+Rectangle1+ class so that the following
program behaves as the example below.

\codeinput{../java/Rectangle1Test.java}{Rectangle1Test.java}

\begin{cmd}
$ java Rectangle1Test
Rectangle: width = 1.0 and height = 3.0
\end{cmd}

\solutioninput{../java/Rectangle1.java}






\subsection{}

Given that the width and the height should be greater than 0, do you think the
width and height attributes of the Rectangle1 class should be public?

Why not?

Write a program called \verb+BreakARectangle1+ that creates a valid
\verb+Rectangle1+, then change its width or its height to some invalid value
and then prints the rectangle.

\begin{blackboard}
$ java BreakARectangle1
Rectangle: width = -1.0 and height = 3.0
\end{blackboard}

\solutioninput{../java/BreakARectangle1.java}




\subsection{}

Fix the class \verb+Rectangle1+ by writing a new version called
\verb+Rectangle2+ that protects the attributes of the rectangle from the user
by making them private.

\solutioninput{../java/Rectangle2.java}




\subsection{}

Would you say that instances of the class \verb+Rectangle2+ are immutable
objects?

\begin{solution}
Yes.
\end{solution}




\subsection{}

Write a mutable version of the class \verb+Rectangle2+ called \verb+Rectangle3+
by adding a \verb+setWidth(double width)+ and a \verb+segHeight(double height)+
methods.

These methods must throw an exception if asked to set an invalid width or height
for the rectangle.

Test your class with the following program:

\codeinput{../java/Rectangle3Test.java}{Rectangle3Test.java}

That must behave exactly as shown below:

\begin{cmd}
$ java Rectangle2Test
Rectangle: width = 1.0 and height = 3.0
Rectangle: width = 5.0 and height = 7.0
ERROR: height must be > 0
\end{cmd}

\solutioninput{../java/Rectangle3.java}


\subsection{}

Simplify the constructor of \verb+Rectangle3+ by using the methods
\verb+setWidth+ and \verb+setHeight+ to make it look like this:

\begin{blackboard}
public Rectangle3(double width, double height) throws exception {
    setWidth(width);
    setHeight(height);
}
\end{blackboard}

And explain if the class is still safe to use.

Do you still need to declare that the constructor throws an exception, why?

Which version of the constructor do you prefer?

\begin{solution}
The class is still safe to use, even if the constructor is not catching
the exceptions from the methods it calls, it will retrhow them.

I still need to declare the throw in my constructor as it is calling methods
that throw exception but not caching them.

I will rather use the new version as it works fine, and it is simpler.
\end{solution}

\subsection{}

Create a new version of the class \verb+Rectangle3+ called \verb+Rectangle4+
that has two overloaded constructors:

\begin{itemize}

  \item \verb+Rectangle4(double side)+ that creates a square rectangle of side
    \verb+side+.

  \item an empty constructor that creates a square rectangle of side \verb+1.0+.

\end{itemize}

Each of these two new constructors must also be safe and their bodies must have
only one line of code.

\solutioninput{../java/Rectangle4.java}





\subsection{}

Create a new version of the class \verb+Rectangle4+ called \verb+Rectangle5+
that has:

\begin{itemize}

  \item a method called \verb+area+ to calculate the area of the rectangle.

  \item a method called \verb+perimeter+ to calculate its perimeter.

  \item a method called \verb+invert+ that exchanges its width with its
    height.

  \item a method called \verb+equals+ to tell if another \verb+Rectangle5+ has
    the same width and height than the one you are invoking this method on.
    Write the body of this method in only one line of code.

  \item a new overloaded constructor that receives a \verb+Rectangle5+ and
    creates a new rectangle with the same width and height. This is called a
    \emph{clone} constructor. Write the body of this constructor in only one
    line of code.

\end{itemize}

Also write a test program for this class, testing all the new methods in
imaginative ways (e.g. is the invert of the invert of a rectangle equals to a
clone of the original?).

\solutioninput{../java/Rectangle5.java}

\subsection{}

Explain if \verb+Rectangle5.invert()+ can throw an exception. Are you sure?

\begin{solution}
It can, but it shouldn't, because a dimension from an object cannot be invalid.
\end{solution}


\subsection{}

Explain why you can access the attributes of one \verb+Rectangle5+ object from
another \verb+Rectangle5+ object if they are private (like you did in the
\verb+equals+ method).

\begin{solution}
Private attributes are private to other classes, but not to objects of
that same class, so you can access the private attributes of one object
from other objects of that same class.
\end{solution}







\section{Simulating Dice}

\subsection{}

Write a class called \texttt{SixSidedDie} that simulates 6-sided die.

It will have an empty constructor and a \verb+public int roll()+ method that
returns a random integer between 1 and 6.

Do not create a new random number generator for each roll; just create a new
random number generator for each die.

\textsl{Tip: do you need to define an empty constructor for the class? or will
the default constructor suffice.}

\solutioninput{../java/SixSidedDie.java}



\subsection{}

Write a program called \texttt{SixSidedDieTest} that test the class from the
previous exercise by creating a die, rolling it 10 times and printing the
outcomes.

\begin{cmd}
$ java SixSidedDieTest
5
6
2
2
4
3
5
2
6
1
\end{cmd}

\solutioninput{../java/SixSidedDieTest.java}





\subsection{}

Write a class called \texttt{Die} that simulates dices of any number of sides.

The constructor must throw an exception if the number of sides is smaller than
2 (this number, 2, is a magic number, avoid using it directly).

Add also an empty constructor that construct dices of 6 sides.

\solutioninput{../java/Die.java}




\subsection{}

Write 3 or 4 paragraphs about how would you test the class from the previous
exercise.

\begin{solution}
Writing a program that tries to create a die of 1 or less sides
to see if the exception is thrown.

Then create a die of 2 sides and throw it some times.

Then create a die of a big number of sides and throw it some times.

Trying to test the fairness of the dice is out of the scope of this course.
\end{solution}







\subsection{}

Use the following program to test your \verb+Die+ class.

Do you understand this program?

\codeinput{../java/DieTest.java}{DieTest.java}






\subsection{}

Add a method called \texttt{howMany} to your \texttt{Die} class that returns
the number of dice created so far.

Test the class with the following program:

\codeinput{../java/CountDice.java}{CountDice.java}

The output from the program should be exactly this:

\begin{cmd}
$ java CountDice
0
3
4
\end{cmd}

\subsection{Playing with REGEXs}

Write a program called \texttt{RollDice} that adds the rolls from a set of dice
according to the following simplified \emph{Dungeons \& Dragons} notation:

\begin{itemize}

  \item \verb+1d6+ means roll one 6-sided die

  \item \verb+3d8+ means roll three 8-sided dice and sum their outcomes

\end{itemize}

\begin{cmd}
$ java RollDice 1d6
5
$ java RollDice 3d8
13
$ java RollDice 1000d100
51249
$ java RollDice
ERROR: bad number of arguments
$ java RollDice 1d6 3d6
ERROR: bad number of arguments
$ java RollDice alberto
ERROR: bad argument format
$ java RollDice 3d6a
ERROR: bad argument format
$ java RollDice 5d1
ERROR: number of sides must be >= 2
\end{cmd}

\textsl{Tip: learn about \emph{regular expressions} before trying to
solve this problem.}

\solutioninput{../java/RollDice.java}








\section{Counting students is hard}

\subsection{}

Write a \verb+Student+ class, that holds the information of a student:

\begin{itemize}

  \item His name

  \item His surname

  \item His Student ID: a number different for each student, assigned to each
    student when he/she is created, starting from 1000.

\end{itemize}

The \verb+toString()+ method for this class must print the information for the
student as shown below:

\verb+surname, name (student_id)+

Test your class with a simple program that creates three students in this order:

\begin{itemize}

  \item Alberto Cortés

  \item Beatriz González

  \item Carlos García

\end{itemize}

Then print their info, first of Alberto, then Carlos, the Beatriz.

The output from your program must look like this:

\begin{cmd}
$ java StudentTest
Cortés, Alberto (1001)
García, Carlos (1003)
González, Beatriz (1002)
\end{cmd}

\solutioninput{../java/Student.java}

\subsection{}

Check the differences between your \verb+StrudentTest+ program and the
teachers' version and try to understand the differences if any:

\codeinput{../java/StudentTest.java}{Teachers' version of StudentTest.java}

\end{document}
