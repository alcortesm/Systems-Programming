\documentclass[a4paper, 9pt]{extarticle}

\usepackage[notes]{style}

\newcommand{\realtitle}{Session 20 - Recursion}

\begin{document}

\makebox[\linewidth]{\rule{\textwidth}{0.4pt}}
UC3M \hfill Alberto Cortés Martín\\
Systems Programming, 2014-2015 \hfill version: \today\\
\makebox[\linewidth]{\rule{\textwidth}{0.4pt}}
\begin{center}
  \Large{\realtitle}\\Lecture Notes
\end{center}
\makebox[\linewidth]{\rule{\textwidth}{0.4pt}}
\vspace{1cm}


\section{Today's Topics}
\begin{blackboard}
Recursion
\end{blackboard}


\section{The Concept}

In mathematics, the factorial of a nonnegative integer is calculated as follows:

\begin{multicols}{2}
\begin{equation*}
  n! =
  \begin{cases}
    1, & \text{if } n = 0\\
    \prod_{i=1}^{n}i, & \text{otherwise}
  \end{cases}, \forall n \in \mathbb{Z}^{*}.
\end{equation*}

\columnbreak

\begin{center}
  \begin{tabular}{|l|l|r|}
    \hline
    $n$ & calculation & $n!$\\
    \hline
    0   & $1$          & 1 \\
    1   & $1$          & 1 \\
    2   & $1 \cdot 2$ & 2 \\
    3   & $1 \cdot 2 \cdot 3$ & 6 \\
    4   & $1 \cdot 2 \cdot 3 \cdot 4$ & 24 \\
    5   & $1 \cdot 2 \cdot 3 \cdot 4 \cdot 5$ & 120 \\
    \ldots & \ldots & \ldots \\
    \hline
  \end{tabular}
\end{center}
\end{multicols}

This is, $n!$ is the product of all the natural numbers from 1 to $n$ and $0!$ is 1.


ASK THE STUDENTS TO WRITE A PROGRAM: write a program that calculates the factorial of a number.

\begin{blackboard}
$ java Factorial 0
1
$ java Factorial 1
1
$ java Factorial 2
2
$ java Factorial 3
6
$ java Factorial 4
24
$ java Factorial 5
120
...
\end{blackboard}


You probably wrote an \emph{iterative} version of the program, this is, a program that \emph{iterates}, this is, a program that uses a loop.

Something like this:

\codelinesinput{../java/FactorialIterative.java}{FactorialIterative.java}{1}{36}


But there is \emph{another} definition of the factorial:
\begin{multicols}{2}
\begin{equation*}
  n! =
  \begin{cases}
    1,              & \text{if } n = 0\\
    n(n-1)!,    & \text{otherwise}
  \end{cases}, \quad \forall n \in \mathbb{Z}^{*}.
\end{equation*}

\columnbreak

\begin{center}
  \begin{tabular}{|l|l|r|}
    \hline
    $n$ & calculation & $n!$\\
    \hline
    0   & $1$          & 1 \\
    1   & $1 \cdot 0!$ & 1 \\
    2   & $2 \cdot 1!$ & 2 \\
    3   & $3 \cdot 2!$ & 6 \\
    4   & $4 \cdot 3!$ & 24 \\
    5   & $5 \cdot 4!$ & 120 \\
    \ldots & \ldots & \ldots \\
    \hline
  \end{tabular}
\end{center}
\end{multicols}

In math, this is called a \emph{recursive} definition, as the definition of the factorial uses the factorial itself.

You have already seen some recursive definitions in Java, for example a \verb+Node<E>+ has an attribute \texttt{next} of type \verb+Node<E>+.

According with this new definition of the factorial, we can write an
alternative version of the \verb+factorial+ method above:

\codelinesinput{../java/FactorialRecursive.java}{FactorialRecursive.java}{29}{35}

This method is shorter and simpler to understand than the previous one, as it
directly mirrors the mathematical definition.

In programming, this is called a \emph{recursive} method, this is a method that calls itself.





\section{Recursion and the Stack}

Let us see what is happening in the stack while the \texttt{FactorialRecursive} program is running:

\begin{blackboard}
$ java FactorialRecursive 3
6
$ java FactorialRecursiveVerbose 3
calling fact(3)
|  calling fact(2)
|  |  calling fact(1)
|  |  |  calling fact(0)
|  |  |  returning 1
|  |  returning 1
|  returning 2
returning 6
6
\end{blackboard}

DIBJAR MEMORIA DURANTE LA EJECUCIÓN




\section{Anatomy of a recursive call}

A recursive call must have two sections:

\begin{itemize}

\item one or more \emph{base cases}: returns a direct solution to the problem
  without using recursion.

\codelinesinput{../java/FactorialRecursive.java}{FactorialRecursive.java}{30}{31}

\item one or more \emph{recursive cases}: reduces the problem by doing some
  operations over a \emph{simplified} version of the problem.

\codelinesinput{../java/FactorialRecursive.java}{FactorialRecursive.java}{32}{33}

\end{itemize}

Without a base case you will have infinite recursion, as the recursive method
will call itself forever.

Without a recursive case, you will not have recursion at all.

It is important that the recursive case calls a simplified version of the
problem, this is, you have to eventually get to the base case throwout your
recursive calls, otherwise you will be making the problem bigger and bigger
instead of simplifiying it and the recursion will never reach the base case.




\section{Can you solve any problem using recursion}

Nobody knows, this is still an open problem, but \emph{it is belived} among the
scientific comunity that

\begin{blackboard}
Any problem on natural integers that can be solved using
iteration, can be solved using recursion, and the other way around.
\end{blackboard}

This is call the ``Church-Turin thesis'', due to Alonso Church and Alan Turin.

It has been proved that this thesis cannot be proved, but most programmers
believe it is true (me included).






\section{Recursion or iteration}

Even if you believe in the Church-Turin thesis, not all problems are the same.

\begin{center}
  \includegraphics[width=8cm]{./img/recur-vs-iter/recur-vs-iter.eps}
\end{center}

There are some problems that are easier to solve using an iterative approach,
and there are problems that are easier to solve using a recursive approach.

Sadly there is no way to know to what group your problem belongs, so you will
need some intuition and imagination to decide between working on an iterative
solution or a recursive one.

Then even if you get it right, maybe the other solution is more efficient in
terms of memory or execution time.

So chosing between iteration or recursion is more an art than a science.





\section{Tips on solving problems using recursion}

The trick is:

\begin{blackboard}
Realize that your problem is just some operations over a simpler version of the
same problem.
\end{blackboard}

Of course this is not always easy, but some times, it is, for example the factorial:

\begin{equation*}
  n! =
  \begin{cases}
    1,              & \text{if } n = 0\\
    n(n-1)!,    & \text{otherwise}
  \end{cases}, \quad \forall n \in \mathbb{Z}^{*}.
\end{equation*}

Instead of solving the factorial of $n$, you perform some operations on the
factorial of $(n-1)$, that is a simpler version of the problem, as you will
eventually reach the base case of $0! = 1$, which is the most simple case.




\section{More Examples of recursion}

Let us see more examples of recursive methods:


\subsection{Counting from N to 1}

\begin{multicols}{2}
  \codelinesinput{../java/CountNTo1Iterative.java}{CountNTo1Iterative.java}{24}{28}
\columnbreak
  \codelinesinput{../java/CountNTo1Recursive.java}{CountNTo1Recursive.java}{24}{29}
\end{multicols}

Counting from $N$ to 1 is just printing $N$ and then counting from $N-1$ to 1.

The base case is just when we reach 1: we do nothing (except for printing the 1).

\subsection{Counting from 1 to N}

\begin{multicols}{2}
  \codelinesinput{../java/Count1ToNIterative.java}{Count1ToNIterative.java}{24}{28}
\columnbreak
  \codelinesinput{../java/Count1ToNRecursive.java}{Count1ToNRecursive.java NOT IN THE HANDOUT}{24}{29}
\end{multicols}

This is quite interesting, just changing when we print the message we are
printing backguards.

This is because the stack frames for the methods live inside a stack, and as
you know an stack reverse the order of its elements.

You will be able to understand this better with a verbose version of the programs:

\newpage

\begin{multicols}{2}
  \begin{blackboard}
$ java CountNTo1Recursive 5
5
4
3
2
1
$ java CountNTo1RecursiveVerbose 5
calling countNTo1(5)
5
|  calling countNTo1(4)
4
|  |  calling countNTo1(3)
3
|  |  |  calling countNTo1(2)
2
|  |  |  |  calling countNTo1(1)
1
|  |  |  |  returning
|  |  |  returning
|  |  returning
|  returning
returning
\end{blackboard}
\columnbreak
  \begin{blackboard}
$ java Count1ToNRecursive 5
1
2
3
4
5
$ java Count1ToNRecursiveVerbose 5
calling countNTo1(5)
|  calling countNTo1(4)
|  |  calling countNTo1(3)
|  |  |  calling countNTo1(2)
|  |  |  |  calling countNTo1(1)
1
|  |  |  |  returning
2
|  |  |  returning
3
|  |  returning
4
|  returning
5
returning
\end{blackboard}
\end{multicols}



\subsection{Palindromes}

Write a method to tell if a string is a palindrome.

A string is a palindrome if it has the same characters when you read it from
the beginning to the end or backwards.

For example:

\begin{blackboard}
"" true
"a" true
"aa" true
"ab" false
"aba" true
"anna" true
"Anna" false
"racecar" true
\end{blackboard}

The iterative solution to this problems is a loop from the begining to the
middle of the string, comparing the \texttt{ith} word with the
\texttt{string.length()-1-ith} word:

\codelinesinput{../java/IsPalindromeIterative.java}{IsPalindromeIterative.java}{14}{21}

The recursive version is simpler:

Let us define some names for the parts of a string:

\begin{center}
\includegraphics[width=8cm]{./img/palindrome/palindrome.eps}
\end{center}

A string is a palindrome if:

\begin{blackboard}
  (first == last) && isPalindrome(middle);
\end{blackboard}

Checking if the middle is a palindrome is easier than checking if the whole
string is palindrome, so we are reducing the problem with each recursion.

The base case is when the middle is of lenght 0 or 1, because all strings of
lenght 0 or 1 are palindromes.

\codelinesinput{../java/IsPalindromeRecursive.java}{IsPalindromeRecursive.java}{14}{23}


This particular implementation of the recursive algorithm is very inefficient,
because we are creating a new string each time we do the recursive call (the
string middle is new with each recursion).

In the self assessment exercises you will find tips on how to make this method
efficient, reusing the same string in each recursive call.


\subsection{Finding the greater common divisor}

ESTE MEJOR PARA EJERCICIOS.


\subsection{Fibonacci}

The Fibonacci sequence looks as follows:

\begin{blackboard}
Fibonacci sequence:

  0, 1, 1, 2, 3, 5, 8, 13, 21, 34 ...
\end{blackboard}

Each number is the sum of its two previous numbers: $34 = 13 + 21$, $21 = 8 + 13$ ...

And $F(0) = 0$ and $F(1) = 1$ by definition.

So you can write the Fibonacci sequence as follows:

\begin{equation*}
  F_n =
  \begin{cases}
    0, & \text{if } n = 0\\
    1, & \text{if } n = 1\\
    F_{n-1} + F_{n-2}, & \text{otherwise}
  \end{cases}, \forall n \in \mathbb{Z}^{*}.
\end{equation*}

Write a program that receives an index and returns the number in the Fibonacci
sequence for that index.

An iterative solution is not difficult, but you will have to think a little bit
about it (I have left this as a problem for your self-assessment). The
recursive solution is trivial though:

\codelinesinput{../java/FibonacciRecursive.java}{FibonacciRecursive.java}{35}{43}

Can you see the base cases and how the problem is reduced with each recursion?

\subsection{Stacks, queue...}

Recursive algorithms for data structure manipulation are very common also, for
example:

Write a LinkedStack with a \texttt{boolean contains(E e)} method.

The iterative version of this program will go through all the elements in the
stack using a for or a while loop, until we reach the buttom of the stack or
until we find the data.

The recursive version is simpler:

\begin{center}
\includegraphics[width=12cm]{./img/linkedstack/linkedstack.eps}
\end{center}

\begin{multicols}{2}
\codelinesinput{../java/LinkedStackRecursiveContains.java}{LinkedStackRecursiveConstains.java}{76}{84}
\columnbreak
\codelinesinput{../java/LinkedStackRecursiveContains.java}{LinkedStackRecursiveContains.java}{18}{27}
\end{multicols}

Can you see the reduction in each step and the base cases?

Of course, you know by the ``Church-Turin'' thesis that you can also write
recursive version for array based collections, I will left that as a
self-assessment exercise.

\section{Recursion Optimizations}

Even when some problem can be easily solved by recursion, the straighforward
recursive solution can be pretty inefficient, let us see how to optimize
recursive calls.


\subsection{Stack Overflows}

As each call to a method in Java creates a new Stack Frame in the stack, you can easily run out of Stack space using recursion:

Example:

\begin{multicols}{2}
\begin{blackboard}
$ java Count1ToNIterative 100000
...
99987
99988
99989
99990
99991
99992
99993
99994
99995
99996
99997
99998
99999
100000
\end{blackboard}
\columnbreak
\begin{blackboard}
$ java Count1ToNRecursive 100000
Exception in thread "main" java.lang.StackOverflowError
        at Count1ToNRecursive.count1ToN(Count1ToNRecursive.java:25)
        at Count1ToNRecursive.count1ToN(Count1ToNRecursive.java:26)
        at Count1ToNRecursive.count1ToN(Count1ToNRecursive.java:26)
        at Count1ToNRecursive.count1ToN(Count1ToNRecursive.java:26)
        at Count1ToNRecursive.count1ToN(Count1ToNRecursive.java:26)
        at Count1ToNRecursive.count1ToN(Count1ToNRecursive.java:26)
        at Count1ToNRecursive.count1ToN(Count1ToNRecursive.java:26)
        at Count1ToNRecursive.count1ToN(Count1ToNRecursive.java:26)
        at Count1ToNRecursive.count1ToN(Count1ToNRecursive.java:26)
        at Count1ToNRecursive.count1ToN(Count1ToNRecursive.java:26)
        at Count1ToNRecursive.count1ToN(Count1ToNRecursive.java:26)
        at Count1ToNRecursive.count1ToN(Count1ToNRecursive.java:26)
        at Count1ToNRecursive.count1ToN(Count1ToNRecursive.java:26)
...
\end{blackboard}
\end{multicols}

The iterative version runs just fine with big numbers, but the recursive
version runs out of stack space pretty quickly.

There are two ways to solve this:

\subsubsection{Make the stack bigger}

First let us try the brute force approach: if we do not have enough stack space
for our program, just make the stack bigger.

You can make the stack space bigger in Java by passing the \texttt{-Xss} to the virtual machine.

The \texttt{-X} stands for \texttt{EXTENSION} and the \texttt{ss} stands for \texttt{Stack Size}.

For example:

\begin{blackboard}
$ java -Xss8000k Count1ToNRecursive 100000
...
99996
99997
99998
99999
100000
\end{blackboard}

It seems that in my laptop, with 8,000 kylobytes of stack is enought to count from 10,000 to 1.



\subsubsection{Tail recursion optimization}

But sometimes we can not make the stack bigger, because we are reducing the
heap and maybe we wont be able to fit all our objects in memory.

In general there is a better solution than making the stack bigger.

Take a look a the \texttt{Count1ToNRecursive} code again.

\codelinesinput{../java/Count1ToNRecursive.java}{Count1ToNRecursive.java}{24}{29}

Now let us try to make a \emph{Tail recursive} version of the same method.

A \emph{tail recursive method} is a recursive method where the recursive call
is the last thing to do in the method.

This is the method does not do any further operations after the recursive call.

\begin{multicols}{2}
\codelinesinput{../java/Count1ToNRecursive.java}{Count1ToNRecursive.java}{24}{29}
\columnbreak
\codelinesinput{../java/Count1ToNTailRecursive.java}{Count1ToNTailRecursive.java}{24}{35}
\end{multicols}

When the last thing to do in a method is to call another method, there is no
need to keep the stack frame of the first method in the stack until the second
method has finished, because the first method has already finished.

This means we can delete the stack frame of the first method right away, before
calling the second method.

This is called \emph{Tail Recursion Optimization}.

When your compilar support \emph{Tail Recursion Optimization}, tail recursive
methods only use one stack frame, so you will never run out of stack space.

This is way learning how to turn recursive methods into tail recursive methods
is so important, and such a popular exam question.

This is the optimum solution for solving the stack space problem, instead of
doing your stack bigger.

Now the bad news: JAVA compilers do not support TAIL RECURSION OPTIMIZATION, so
this beautifull method is useless in JAVA.

But turning recursive methods into tail recursive methods is still one of the
most important abilities of a programmer, because most serious languages
supoprt tail recursion optimization.



\subsection{Exponential growth}

Another typicall problem of recursion is exponential growth.

Let us see an example:

Do you remember the previous version of the Fibonacci program?

That is the worst implementation ever, no serious programmer will use it, ever.

Let us take a look at the code again, Why do you think this is the code ever?

\codelinesinput{../java/FibonacciRecursive.java}{FibonacciRecursive.java}{35}{43}

This code is very slow!

\begin{blackboard}
$ time java FibonacciRecursive 45
1134903170

real    0m14.930s
user    0m14.909s
sys     0m0.020s
\end{blackboard}

It takes 15 seconds to calculate the Fibonacci of 45, but it sould take less than a second.

What is the problem?

\begin{center}
\includegraphics[width=14cm]{./img/fibonacci/fibonacci.eps}
\end{center}

All these kind of problems can be easily solved using Dynamic Programming which
is a very important technique for optimizing solutions to problems.

Dynamic programming is a huge topic and we are not going to see all the details
in this course.

But let me show you a solution to this problem using one of the techniques of
Dynamic programming, called MEMOIZATION, which means basically REMEMBERING:

Instead of executing fibonacci 17 trillion times, let us just remember the
result for each execution so we do not have to calculate the same numbers over
a over.

TACHAR CIRCULOS EN EL DIBUJO

\begin{multicols}{2}
\codelinesinput{../java/FibonacciRecursive.java}{FibonacciRecursive.java}{35}{43}
\columnbreak
\codelinesinput{../java/FibonacciMemoization.java}{FibonacciMemoization.java}{29}{42}
\end{multicols}

Now let us compare times:

\begin{multicols}{2}
\begin{blackboard}
$ time java FibonacciRecursive 45
1134903170

real    0m14.930s
user    0m14.909s
sys     0m0.020s
\end{blackboard}
\columnbreak
\begin{blackboard}
$ time java FibonacciMemoization 45
1134903170

real    0m0.105s
user    0m0.088s
sys     0m0.016s
\end{blackboard}
\end{multicols}

Wow, that was fast!

Companies will pay you a lot of money to make a program work 10\% faster.

We have just make this program 14,000 times faster just by using a simple int array.


\subsection{Mutual recursion}

Mutual recursion is when you have indirect recursion, this is a method calls itself by the means of other, intermediate methods.

For example:

\begin{blackboard}
class A {
  void m1() {
    // some operations
    m2();
  }

  void m2() {
    // some other operations
    m1();
  }
}
\end{blackboard}

A very inefficient example:

\codelinesinput{../java/EvenOrOdd.java}{EvenOrOdd.java}{11}{25}

\end{document}
