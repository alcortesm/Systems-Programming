\documentclass[a4paper, 9pt]{extarticle}

\usepackage[notes]{style}

\newcommand{\realtitle}{Session 20 - Recursion}

\begin{document}

\makebox[\linewidth]{\rule{\textwidth}{0.4pt}}
UC3M \hfill Alberto Cortés Martín\\
Systems Programming, 2014-2015 \hfill version: \today\\
\makebox[\linewidth]{\rule{\textwidth}{0.4pt}}
\begin{center}
  \Large{\realtitle}\\Lecture Notes
\end{center}
\makebox[\linewidth]{\rule{\textwidth}{0.4pt}}
\vspace{1cm}


\section{Today's Topics}
\begin{blackboard}
Recursion
\end{blackboard}


\section{The Concept}

In mathematics, the factorial of a nonnegative integer is calculated as follows:

\begin{multicols}{2}
\begin{equation*}
  n! =
  \begin{cases}
    1, & \text{if } n = 0\\
    \prod_{i=1}^{n}i, & \text{otherwise}
  \end{cases}, \forall n \in \mathbb{Z}^{*}.
\end{equation*}

\columnbreak

\begin{center}
  \begin{tabular}{|l|l|r|}
    \hline
    $n$ & calculation & $n!$\\
    \hline
    0   & $1$          & 1 \\
    1   & $1$          & 1 \\
    2   & $1 \cdot 2$ & 2 \\
    3   & $1 \cdot 2 \cdot 3$ & 6 \\
    4   & $1 \cdot 2 \cdot 3 \cdot 4$ & 24 \\
    5   & $1 \cdot 2 \cdot 3 \cdot 4 \cdot 5$ & 120 \\
    \ldots & \ldots & \ldots \\
    \hline
  \end{tabular}
\end{center}
\end{multicols}

This is, $n!$ is the product of all the natural numbers from 1 to $n$ and $0!$ is 1.


ASK THE STUDENTS TO WRITE A PROGRAM: write a program that calculates the factorial of a number.

\begin{blackboard}
$ java Factorial 0
1
$ java Factorial 1
1
$ java Factorial 2
2
$ java Factorial 3
6
$ java Factorial 4
24
$ java Factorial 5
120
...
\end{blackboard}


You probably wrote an \emph{iterative} version of the program, this is, a program that \emph{iterates}, this is, a program that uses a loop.

Something like this:

\codelinesinput{../java/FactorialIterative.java}{FactorialIterative.java}{1}{36}


But there is \emph{another} definition of the factorial:
\begin{multicols}{2}
\begin{equation*}
  n! =
  \begin{cases}
    1,              & \text{if } n = 0\\
    n(n-1)!,    & \text{otherwise}
  \end{cases}, \quad \forall n \in \mathbb{Z}^{*}.
\end{equation*}

\columnbreak

\begin{center}
  \begin{tabular}{|l|l|r|}
    \hline
    $n$ & calculation & $n!$\\
    \hline
    0   & $1$          & 1 \\
    1   & $1 \cdot 0!$ & 1 \\
    2   & $2 \cdot 1!$ & 2 \\
    3   & $3 \cdot 2!$ & 6 \\
    4   & $4 \cdot 3!$ & 24 \\
    5   & $5 \cdot 4!$ & 120 \\
    \ldots & \ldots & \ldots \\
    \hline
  \end{tabular}
\end{center}
\end{multicols}

In math, this is called a \emph{recursive} definition, as the definition of the factorial uses the factorial itself.

You have already seen some recursive definitions in Java, for example a \verb+Node<E>+ has an attribute \texttt{next} of type \verb+Node<E>+.

According with this new definition of the factorial, we can write an
alternative version of the \verb+factorial+ method above:

\codelinesinput{../java/FactorialRecursive.java}{FactorialRecursive.java}{29}{35}

This method is shorter and simpler to understand than the previous one, as it
directly mirrors the mathematical definition.

In programming, this is called a \emph{recursive} method, this is a method that calls itself.





\section{More Examples}

Let us see more examples of recursive methods:




\end{document}
