\documentclass[a4paper, 9pt]{extarticle}

\usepackage[notes]{style}

\newcommand{\realtitle}{Session 20 - Recursion}

\begin{document}

\makebox[\linewidth]{\rule{\textwidth}{0.4pt}}
UC3M \hfill Alberto Cortés Martín\\
Systems Programming, 2014-2015 \hfill version: \today\\
\makebox[\linewidth]{\rule{\textwidth}{0.4pt}}
\begin{center}
  \Large{\realtitle}\\Lecture Notes
\end{center}
\makebox[\linewidth]{\rule{\textwidth}{0.4pt}}
\vspace{1cm}


\section{Today's Topics}
\begin{blackboard}
Recursion
\end{blackboard}


\section{Concept}

Let's write a program that calculates the factorial of a natural number:

\begin{equation*}
  n! = n(n-1)\cdots2\cdot1=\prod_{i=1}^{n}i, \quad \forall n \in \mathbb{N}.
\end{equation*}

Its output should be:

\begin{center}
  \begin{tabular}[!ht]{|r|r||r|r|}
    \hline
    1 & 1      & 8  & 40\_320 \\
    2 & 2      & 9  & 362\_880 \\
    3 & 6      & 10 & 3\_628\_800 \\
    4 & 24     & 11 & 39\_916\_800 \\
    5 & 120    & 12 & 479\_001\_600 \\
    6 & 720    & 13 & 6\_227\_020\_800 \\
    7 & 5\_040 & \ldots & \ldots \\
    \hline
  \end{tabular}
\end{center}

ASK THE STUDENTS TO WRITE THE PROGRAM.

You probably wrote an \emph{iterative} version of the program, this is, a program that \emph{iterates}, this is, a program that uses a loop.

Something like this:

\codelinesinput{../java/Factorial.java}{Factorial.java}{1}{37}


But there is another definition of the factorial function:

\begin{equation*}
  n! =
  \begin{cases}
    1,              & \text{if } n = 0\\
    n\cdot(n-1)!,   & \text{otherwise}
  \end{cases}, \quad \forall n \in \mathbb{N}.
\end{equation*}

This is called a \emph{recursive} definition, as the definition of the factorial of n, uses the factorial itself.

You have already seen some recursive definitions in Java, for example a \verb+Node<E>+ has an attribute next of type \verb+Node<E>+.

According with this alternative definition of a factorial we can write an
alternative version of the method \verb+factI+ above, called \verb+factR+
(Recursive factorial):

ASK THE STUDENTS TO WRITE THE ALTERNAVE METHOD "FACTR".

\codelinesinput{../java/Factorial.java}{Factorial.java}{38}{50}








\section{Definition}

A method is called recursive if it calls itself (directly or indirectly).

For example:




\end{document}
