\documentclass[a4paper, 11pt]{article}

\usepackage[assessment]{style}

\newcommand{\realtitle}{Session 20 - Recursion}

\begin{document}

\makebox[\linewidth]{\rule{\textwidth}{0.4pt}}
UC3M \hfill Alberto Cortés Martín\\
Systems Programming, 2014-2015 \hfill version: \today\\
\makebox[\linewidth]{\rule{\textwidth}{0.4pt}}
\begin{center}
  \Large{\realtitle}\\Self-Assessment
\end{center}
\makebox[\linewidth]{\rule{\textwidth}{0.4pt}}
\vspace{1cm}

\section{Triangular Numbers}

In mathematics, the following sequence is known as the \emph{sequence of the
triangular numbers}:

\begin{center}
\begin{blackboard}
1 3 6 10 15 21 28 36 ...
\end{blackboard}
\end{center}

Each number in the sequence is given by the following expresion:
\begin{align*}
  T_n & = 1 + 2 + 3 + \cdots + (n-1) + n \\
      & = \sum_{i=1}^{n}i, \quad \forall n \in \mathbb{N}.
\end{align*}

This is, the triangular number $n$ is the sum of all the naturals numbers from
1 to $n$.

Triangular numbers can also be defined recursively:

\begin{multicols}{2}
\begin{equation*}
  T_n =
  \begin{cases}
    1,              & \text{if } n = 1\\
    n + T_{n-1},    & \text{otherwise}
  \end{cases}, \quad \forall n \in \mathbb{N}.
\end{equation*}

\columnbreak

\begin{center}
  \begin{tabular}{|l|l|r|}
    \hline
    $n$ & calculation & $T_n$\\
    \hline
    1   &           1 & 1 \\
    2   & $2 + T_1$   & 3 \\
    3   & $3 + T_2$   & 6 \\
    4   & $4 + T_3$   & 10 \\
    \ldots & \ldots & \ldots \\
    \hline
  \end{tabular}
\end{center}
\end{multicols}

In the following exercisess you will be asked to calculate triangular numbers,
but triangular numbers of numbers bigger than 65,535 are too big to fit into a
Java \texttt{int}.

Do not worry about this practical detail, just use \texttt{int} for simplicity,
as we are not really interested in calculating big triangular numbers, just on
understanding recursion.

\subsection{}

Write a program called \texttt{TriangularIterative} that receives the index
number and returns the corresponding triangular number. For example:

\begin{cmd}
$ java TriangularIterative -13
USAGE: java TriangularIterative number
     number: integer from 1 to 65535
$
$ java TriangularIterative 0
USAGE: java TriangularIterative number
     number: integer from 1 to 65535
$
$ java TriangularIterative 65536
USAGE: java TriangularIterative number
     number: integer from 1 to 65535
$
$ java TriangularIterative alcortes
USAGE: java TriangularIterative number
     number: integer from 1 to 65535
$
$ java TriangularIterative 12 alcortes
USAGE: java TriangularIterative number
     number: integer from 1 to 65535
$
$ java TriangularIterative 1
1
$ java TriangularIterative 2
3
$ java TriangularIterative 3
6
$ java TriangularIterative 4
10
$ java TriangularIterative 65535
2147450880
\end{cmd}

Write the program using an iterative style.

\textsl{Tip: this program is pretty similar to a program that calculates
factorials.}

\solutioninput{../java/TriangularIterative.java}

\subsection{}

Write a program called \texttt{TriangularRecursive} that does the same as the
previous one, but using a recursive style.

\solutionlinesinput{../java/TriangularRecursive.java}{29}{45}

\subsection{}

Write a program called \texttt{TriangularFormula} that does the same as the
previous one, but executes in constant time, no matter the input argument.

\textsl{Tip: Carl Friedrich Gauss
\href{http://www.americanscientist.org/issues/pub/gausss-day-of-reckoning/1}{is
said to have solved this problem in the 1780s}, when he was 8 years old.}

\solutionlinesinput{../java/TriangularFormula.java}{29}{38}

\subsection{}

What happens when you try to calculate the triangular number of 65,535 using
your recursive version?

Explain why and look for a way to execute your recursive program with such a
big input parameter.

\subsection{}

Write a tail recursive version of your recursive program and explain why this
new version still does not solve the problem in the previos exercise when using
Java.


\subsection{}

Try to calculate the triangular number of 65,535 using your formula program.

Explain why it does not work and fix it.


\subsection{}

Which of the three versions of your program is the fastest one for big numbers?

Explain why.

\section{Palindromes}

\subsection{}

The recursive implementation of the algorithm you saw during the lecture was
very inefficient as we were creating a new string (the ``middle'') with each
recursive call.

Write a \texttt{IsPalindromeRecursiveEfficent} program that solves this
problem.

\textsl{Tip: the trick is to reuse the same string on all the calls to the
  recursive algorithm an use two indexes (first and last, for example) to know
what portions of the string we must consider in each recursion.}

\textsl{The two indexes will indicate the first and last chararcter of the
string on the first call but will advance towards the center of the string as
the recursion goes.}

\solutioninput{../java/IsPalindromeRecursiveEfficient.java}

\end{document}
