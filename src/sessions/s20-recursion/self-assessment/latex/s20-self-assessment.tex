\documentclass[a4paper, 11pt]{article}

\usepackage[assessment]{style}

\newcommand{\realtitle}{Session 20 - Recursion}

\begin{document}

\makebox[\linewidth]{\rule{\textwidth}{0.4pt}}
UC3M \hfill Alberto Cortés Martín\\
Systems Programming, 2014-2015 \hfill version: \today\\
\makebox[\linewidth]{\rule{\textwidth}{0.4pt}}
\begin{center}
  \Large{\realtitle}\\Self-Assessment
\end{center}
\makebox[\linewidth]{\rule{\textwidth}{0.4pt}}
\vspace{1cm}

\section{Triangular Numbers}

In mathematics, the following sequence is known as the \emph{sequence of the
triangular numbers}:

\begin{center}
\begin{blackboard}
1 3 6 10 15 21 28 36 ...
\end{blackboard}
\end{center}

Each number in the sequence is given by the following expresion:
\begin{align*}
  T_n & = 1 + 2 + 3 + \cdots + (n-1) + n \\
      & = \sum_{i=1}^{n}i, \quad \forall n \in \mathbb{N}.
\end{align*}

This is, the triangular number $n$ is the sum of all the naturals numbers from
1 to $n$.

Triangular numbers can also be defined recursively:

\begin{multicols}{2}
\begin{equation*}
  T_n =
  \begin{cases}
    1,              & \text{if } n = 1\\
    n + T_{n-1},    & \text{otherwise}
  \end{cases}, \quad \forall n \in \mathbb{N}.
\end{equation*}

\columnbreak

\begin{center}
  \begin{tabular}{|l|l|r|}
    \hline
    $n$ & calculation & $T_n$\\
    \hline
    1   &           1 & 1 \\
    2   & $2 + T_1$   & 3 \\
    3   & $3 + T_2$   & 6 \\
    4   & $4 + T_3$   & 10 \\
    \ldots & \ldots & \ldots \\
    \hline
  \end{tabular}
\end{center}
\end{multicols}

In the following exercisess you will be asked to calculate triangular numbers,
but triangular numbers of numbers bigger than 65,535 are too big to fit into a
Java \texttt{int}. Do not worry about this practical detail, just use
\texttt{int} for simplicity, as we are not really interested in calculating big
triangular numbers, just on understanding recursion.

\subsection{}

Write a program called \texttt{TriangularIterative} that receives the index
number and returns the corresponding triangular number. For example:

\begin{cmd}
$ java TriangularIterative -13
USAGE: java TriangularIterative number
     number: integer from 1 to 65535
$
$ java TriangularIterative 0
USAGE: java TriangularIterative number
     number: integer from 1 to 65535
$
$ java TriangularIterative 65536
USAGE: java TriangularIterative number
     number: integer from 1 to 65535
$
$ java TriangularIterative alcortes
USAGE: java TriangularIterative number
     number: integer from 1 to 65535
$
$ java TriangularIterative 12 alcortes
USAGE: java TriangularIterative number
     number: integer from 1 to 65535
$
$ java TriangularIterative 1
1
$ java TriangularIterative 2
3
$ java TriangularIterative 3
6
$ java TriangularIterative 4
10
$ java TriangularIterative 65535
2147450880
\end{cmd}

Write the program using an iterative style.

\textsl{Tip: this program is pretty similar to a program that calculates
factorials.}

\solutioninput{../java/TriangularIterative.java}

\subsection{}

Write a program called \texttt{TriangularRecursive} that does the same as the
previous one, but using a recursive style.

\solutionlinesinput{../java/TriangularRecursive.java}{29}{45}

\subsection{}

Write a program called \texttt{TriangularFormula} that does the same as the
previous one, but executes in constant time, no matter the input argument.

\textsl{Tip: Carl Friedrich Gauss
\href{http://www.americanscientist.org/issues/pub/gausss-day-of-reckoning/1}{is
said to have solved this problem in the 1780s}, when he was 8 years old.}

\solutionlinesinput{../java/TriangularFormula.java}{29}{38}

\subsection{}

What happens when you try to calculate the triangular number of 65,535 using
your recursive version?

Explain why and look for a way to execute your recursive program with such a
big input parameter.

\subsection{}

Write a tail recursive version of your recursive program and explain why this
new version still does not solve the problem in the previous exercise when using
Java.


\subsection{}

Try to calculate the triangular number of 65,535 using your formula program.

Explain why it does not work and fix it.


\subsection{}

Which of the three versions of your program (\texttt{TriangularIterative},
\texttt{TriangularRecursive} and \texttt{TriangularFormula} is the fastest
for big numbers?

Explain why.

\section{Palindromes}

\subsection{}

The recursive implementation of the algorithm you saw during the lecture was
very inefficient as we were creating a new string (the ``middle'') with each
recursive call.

Write a \texttt{IsPalindromeRecursiveEfficent} program that solves this
problem.

\textsl{Tip: the trick is to reuse the same string on all the calls to the
  recursive algorithm an use two indexes (first and last, for example) to know
what portions of the string we must consider in each recursion.}

\textsl{The two indexes will indicate the first and last chararcter of the
string on the first call but will advance towards the center of the string as
the recursion goes.}

\solutioninput{../java/IsPalindromeRecursiveEfficient.java}


\section{A Recursion Classic: Euclid's Algorithm}

(From Knuth's \textsl{The Art of Computer Programming, Vol. 1: Fundamental
Algorithms}, third ed.)

The Euclidean algorithm (c. 300 B.C.) for finding the greater common divisor of
two numbers $m$ and $n$, that is, the largest positive integer that evenly
divides both $m$ and $n$ goes as follows:

\begin{itemize}

  \item[\textbf{E1.}] [Find remainder.] Divide $m$ and $n$ and let $r$ be the remainder. (We will have $0 \le r \le n$).

  \item[\textbf{E2.}] [Is it zero?] If $r = 0$, the algorithm terminates; $n$ is the answer.

  \item[\textbf{E3.}] [Reduce.] Set $m \leftarrow n$, $n \leftarrow r$, and go back to step E1.

\end{itemize}

\subsection{}

Calculate, using pen and paper, the gcd (gratest common divisor) of $m = 119$
and $n = 544$ using Euclid's algorithm.

\begin{solution}
17
\end{solution}

\subsection{}

Write a program called \texttt{GcdIterative} that calculates the gcd of two
numbers using an iterative version of Euclid's algorithm.

\solutioninput{../java/GcdIterative.java}

\subsection{}

Explain why

\begin{equation}
  \label{eq:euclid}
  gcd(m, n) = gcd(n, r)\text{.}
\end{equation}

\textsl{Tip: Any number that divides $m$ and $n$ must divide $m - qn$.}

\begin{solution}
  $m = qn + r$
\end{solution}

\subsection{}

Based on equality \eqref{eq:euclid}, write a program called
\texttt{GcdRecursive} that returns the gcd of two numbers using recursion.

\solutioninput{../java/GcdRecursive.java}


\section{Fibonacci Again}

\subsection{}

Write a program called \texttt{FibonacciIterative} that calculates the Fibonacci number of its argument using an iteration.

\solutioninput{../../lecture-notes/java/FibonacciIterative.java}




\section{Recursion on Collections}

\subsection{}

Write a class called \texttt{ArrayStackRecursiveContains} that use a recursive
version of the \texttt{boolean contains(E e)} method.

This class must implement the \texttt{Stack} interface using an array.


\subsection{}

Write an \texttt{ArrayQueue} class and a \texttt{LinkedQueue} class, that
implement the \texttt{Queue} interface, using an array and a linked list
respectively.

The \texttt{toString} method for these two classes must be written using recursion.


\section{Tail Recursion}

\subsection{}

Write a tail recursive version of a program to calculate Factorials.



\section{Recursion Madness}

\subsection{}

In a book store there is a \underline{\textsl{list}} of cash register machines.

On each cash register machine there is a \underline{\textsl{queue}} of clients.

Each client IS-A person and HAS-A \underline{\textsl{stack}} of books he/she wants to buy.

A person has a DNI and an age.

A book has a title and a price (\texttt{BigDecimal}).

The queues does not accept duplicated clients, but the stack of books do, as
you can buy several copies of the same book.

Write a program that calculates:

\begin{itemize}

  \item the average age of the clients waiting to pay for their books

  \item the total price of all the books from all the clients that are waiting to pay.

\end{itemize}

Implement this program and all its classes using recursion instead of iteration
in every method.

Implement your list using a linked list, your queue using an array and your
stack using a linked list.

\textsl{Tip: this problem is quite hard, the point is for you to practice and
think, not to finish it. In any case, if you manage to finish it, come and see
me, I would like to congratulate you in person.}

\end{document}
