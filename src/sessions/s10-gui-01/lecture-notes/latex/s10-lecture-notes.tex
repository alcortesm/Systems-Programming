\documentclass[a4paper, 9pt]{extarticle}

\usepackage[notes]{style}

\newcommand{\realtitle}{Session 10 - GUI 01}

\begin{document}

\makebox[\linewidth]{\rule{\textwidth}{0.4pt}}
UC3M \hfill Alberto Cortés Martín\\
Systems Programming, 2014-2015 \hfill version: \today\\
\makebox[\linewidth]{\rule{\textwidth}{0.4pt}}
\begin{center}
  \Large{\realtitle}\\Lecture Notes
\end{center}
\makebox[\linewidth]{\rule{\textwidth}{0.4pt}}
\vspace{1cm}


\section{Today's Topics}
\begin{blackboard}
Functors, Nested classes & Anonymous classes
GUIs
\end{blackboard}







\section{Classes we are going to use today}

Today we are going to use the Java Runnable interface in our examples:

\begin{multicols}{2}
  \begin{center}
    \verb+runnable.eps+\\
    \includegraphics[width=5cm]{./img/runnable/runnable.eps}
  \end{center}
  \columnbreak
  \begin{blackboard}
interface Runnable {
    public void run();
}
\end{blackboard}
\end{multicols}
















\section{Funtors}


Let's write a simple program that prints "Hello":

\codeinput{../java/Hello1.java}{Hello1.java}

Now let's write another version of this same program:

\codeinput{../java/Hello2.java}{Hello2.java}

We have hide what we want to do inside a method and call that method instead.

Now let's write another, more complicated version of this program:

\begin{multicols}{2}
  \codeinput{../java/Hello3.java}{Hello3.java}
  \columnbreak
  \codeinput{../java/PrintFunctor.java}{PrintFunctor.java}
\end{multicols}

We have hide what we want to do inside an object, and then call the run()
method of that object.

This seems like a very complicated way to something very simple, and you are
right, but stay with me for a minute.

Whenever you want to print "Hello" you can create an object from this class
and call its \verb+run()+ mehtod.

The sole purpuse in life of this class is to carry what we want to do, and then
do it when we call its run method.

This class is just what we want to do, disguised as an object.

This class is called a FUNCTOR, a bunch of code disguised as an object.

Funtors are very important because you can not pass code or methods, but you can pass objects.

For example:

Let's say we have a program that prints 10 times "hello" and adds 10 times
"Alberto" to an \verb+ArrayList<String>+.

You can do that by hand, or create a method that runs an object 10 times and
then you call that method two times: first with a functor that prints "hello",
then with a functor that adds "Alberto" to an \verb+ArrayList<String>+.

The conceptual jump is huge, you are no longer creating methods that does
something, you are creating methods that do "anything" you want to do.




\subsection{Nested Classes}

Now, in Java there is yet a better way to do this.

As you are only going to use the PrintFuntor class to create a single object,
pass it to a method and then forget about it, creating a PrintFunctor.java
files seems like an overkill.

It is not a very important class, you are only going to use it once to create a
single object and then you forget about it.

In Java, you can create classes inside another class, they are called NESTED
CLASSES.

So instead of creating a new java file for this class, you can simply write
this class inside the class that is going to use it.

\codeinput{../java/Hello4.java}{Hello4.java}




\subsection{Anonymous Classes}

Now, in Java there is even a better way to do that.

You don't even have to create a nested class, you can just write the class in
the same line where you are creating an object from that class.

An anonymous class is how you create an object from a class that you only need
for that single object, and then you forget about it forever.

In spanish there is a perfect way to describe them: "una clase anónima es una
clase de usar y tirar".

The best english translation I have found for this is, anonymous classes are
"one use classes" or "disposable classes".

You use them once to create a single object and then you throw them away.

For example, if you need an object that implements the Runnable interface, and
that prints hello when you call its run method, you can write something like this:

\codeinput{../java/Hello5.java}{Hello5.java}





















\section{GUI vs. TUI}

Let's run a program that prints the prime numbers smaller than its argument:

\verb+ DEMO: java PrimesTui+

Now let's run a Graphical program that does the same:

\verb+ DEMO: java PrimesGui+

Let's talk about their similarities and differences:

\paragraph{Similarities}

\begin{itemize}

  \item They have the same functionality.

    In fact, the core of both programs is the same, a class that calculates
    prime numbers, only 35 loc.

\end{itemize}

\paragraph{Differences}

\begin{itemize}

  \item The GUI version is much complex:

    \begin{itemize}

      \item PrimeTui.java is 17 loc, PrimesGui.java is 151 loc.

      \item PrimeTui uses 3 classes: String and 2 Exceptions. But PrimesGui
        uses 12 classes, 2 interfaces and defines 2 nested classes, 4 functors
        and 4 anonymous classes.

        Most of these classes belong to the packages \verb+java.swing+ or
        \verb+java.awt+.

        Those packages are Java libreries to create graphical interfaces.

        \verb+java.awt+ is an old version, \verb+java.swing+ is the new version
        that still use some classes from the old one.

    \end{itemize}

  \item The GUI version is a multithreaded application, this means our GUI
    program is doing several things at the same time.

    \begin{itemize}

      \item I have write a class that returns the prime numbers smaller than a
        given number, it looks like this:

        \begin{center}
          \includegraphics[width=8cm]{./img/primegenerator/primegenerator.eps}
        \end{center}

      \item The TUI version is just doing this:

        \begin{center}
          \includegraphics[width=8cm]{./img/tui_thread/tui_thread.eps}
        \end{center}

      \item the GUI version is doing several things at the same time:

        \verb+DEMO DE REDIMENSIONAR LA VENTANA MIENTRAS EJECUTA+

        \begin{center}
          \includegraphics[width=14cm]{./img/gui_threads/gui_threads.eps}
        \end{center}

    \end{itemize}

  \item the GUI version is not really doing as much work as you might think,
    for example the windows decorations are being drawn by a process in my
    operating system, not by the Java application itself, for example, if I ran
    this same application in a windows computer it will look different, like
    this:

    \verb+DIBUJAR LAS WINDOW DECORATION EN WINDOWS+

\end{itemize}




\section{My first GUI}

Let's create an program that creates and shows a window.

\codeinput{../java/Gui01.java}{Gui01.java}

The main threads ends, as you can see by the message, but the swing event
dispacher threads continues executing, so you can perform graphical operations.

But as you can see, if we try to close the window, the window is closed, but
the program still continues.

Let's fix this by telling our window that when it closes, the program must end.

\verb+ENLACE: JFrame setDefaultCloseOperation+

\codeinput{../java/Gui02.java}{Gui02.java}

Just to keep things organized let's group all this graphical code in a method,
let's call it, for example: createAndShowGui:

\codeinput{../java/Gui04.java}{Gui04.java}

But this way of creating GUI is very dangerous, because all this graphical code
is being executed by the main thread.

In swing, all GUI code must be executed by the EDT or you will have extremely
difficult bugs.

So, how do you tell your main thread, that you want this createAndShowGui
method to be executed by EDT thread instead?

In Swing, there is a class called \verb+javax.swing.SwingUtilities+ with an
static method \verb+invokeLater()+ that allows you to send code to the EDT for
execution.

This invokeLater method receives a functor of course.

\codeinput{../java/Gui05.java}{Gui05.java}

\verb+DIBUJAR EL ESQUEMA DE HILOS+






\subsection{Extending from JFrame}

Now everything is fine and works properly.

But let me explain a different way to do the same thing, that has some
advantadges for small applications like the ones we are going to program in
this course.

Sadly, it is quite hard to understand now why this alternative way is better.

This is the kind of thing you learn after having some expirience doing
graphical interfaces.

So I'm not even going to try to explain to you why this way is better, just
believe me and I can guaranty that your life will be easier.

Now, what is this alternative way:

\verb+MOSTRAR java PrimesGui+

Many people think about this a graphical application as an specialized version
of a standard Java Frame.

There are standard Java empty frames (JFrames), and then there is my
application that is an especial version of a frame, with some buttons and
things.

From that point of view, your application IS A frame, and therefore, extends
from JFrame.

\codeinput{../java/Gui06.java}{Gui06.java}

As I told you before, this way of creating graphical interfaces has some
advantadges when the application is small, but I am not going to try to explain
why, ok, just believe me on this.

If you don't believe me, just try and build 20 or 30 applications, and after
that, I'm quite sure you will understand.






\subsection{Adding graphical components to your window}

Now let's do something interesting with the window, let's add some graphical
components, for example a Label, which is just a String of text that you can
write inside a window.

Now, a frame has several parts:

\verb+DIBUJAR JFRAME: MENU, CONTENTPANE, and GLASSPANE+

The contentpane is where you add graphical componentes, like buttons and
labels.

This contentpane is an object of the class JPanel.

A JPanel is like a corkboard, where you can pin several things, like buttons or
labels for example.

To add a graphical object to our cork, we use its \verb+add+ method.

So let's add a label to our window:

\codeinput{../java/Gui07.java}{Gui07.java}

The default font size of swing is quite small, so let me call a method I have
write myself to make fonts bigger:

\codeinput{../java/Gui08.java}{Gui08.java}

Ok, now you can see things much better.

Now lets add some decorations to our text.

A JLabel inherits from JComponent and all JComponents have a background color
and a foreground color.

Another important thing to know about Java Components is that their background
is transparent by default, so you will need to set them opaque if you want to
see the background color.

\codeinput{../java/Gui09.java}{Gui09.java}

Now that we are actually showing something inside our window, we no longer need
to tell the window its size, we can ask the frame to adjust its side to
whatever it has inside, the pack() method does this:

\codeinput{../java/Gui10.java}{Gui10.java}


Now let's add some more graphical elements.

When you add graphical elements to a panel, they position they are going to get
depends on the order you add them and an allocation policy.

The policy of the panels is configurable, and you have some standard policies
you can chose from, they are called \verb+LayoutManagers+.

\verb+ENLACE: A Visual Guide to Layout Managers+

\codeinput{../java/Gui11.java}{Gui11.java}

You can define your own graphical components to add to your panels, but there
are a bunch predefined graphical components that you will find usefull.

\verb+ENLACE: A Visual Guide to Swing Components+

Now, let's do something a little more interesting, let's pretend we want to
program a Java source editor, with the ability to open Java files, save Java
files and compile Java files.

The GUI will look like this:

\verb+dibujar textarea, label abajo y una barra de botones arriba con open, save y compile+

\codeinput{../java/Gui12.java}{Gui12.java}

The buttons are very big because the grid is trying to fill all the available
space, let's change the grid for a BoxLayout, that will group the buttons
horizontally, but without trying to fill all the available space.

\codeinput{../java/Gui13.java}{Gui13.java}

You can see that when you try to write pass the borders of the textarea, you
can not see what you are typing.

ScrollPanes are made to see an small portion of a bigger graphical element, by
adding some scroll bars around it, let's see if this solves our problem:

\codeinput{../java/Gui14.java}{Gui14.java}


\end{document}
