\documentclass[a4paper, 9pt]{extarticle}

\usepackage[notes]{style}

\newcommand{\realtitle}{Session 10 - GUI 01}

\begin{document}

\makebox[\linewidth]{\rule{\textwidth}{0.4pt}}
UC3M \hfill Alberto Cortés Martín\\
Systems Programming, 2014-2015 \hfill version: \today\\
\makebox[\linewidth]{\rule{\textwidth}{0.4pt}}
\begin{center}
  \Large{\realtitle}\\Lecture Notes
\end{center}
\makebox[\linewidth]{\rule{\textwidth}{0.4pt}}
\vspace{1cm}


\section{Today's Topics}
\begin{blackboard}
Functors, Nested classes & Anonymous classes
GUIs
\end{blackboard}







\section{Classes we are going to use today}

Today we are going to use two classes in our examples:

\begin{multicols}{2}
  \begin{center}
    \verb+person.eps+\\
    \includegraphics[width=5cm]{./img/person/person.eps}
  \end{center}
  \columnbreak
  \codeinput{../java/Person.java}{Person.java}
\end{multicols}

\begin{multicols}{2}
  \begin{center}
    \verb+runnable.eps+\\
    \includegraphics[width=5cm]{./img/runnable/runnable.eps}
  \end{center}
  \columnbreak
  \begin{blackboard}
interface Runnable {
    public void run();
}
\end{blackboard}
\end{multicols}
















\section{Funtors}


Let's write a simple program that prints "Hello":

\codeinput{../java/Hello1.java}{Hello1.java}

Now let's write another version of this same program:

\codeinput{../java/Hello2.java}{Hello2.java}

Now let's write another, more complicated version of this program:

\begin{multicols}{2}
  \codeinput{../java/Hello3.java}{Hello3.java}
  \columnbreak
  \codeinput{../java/PrintFunctor.java}{PrintFunctor.java}
\end{multicols}

This seems like a very complicated way to something very simple, and you are
right, but stay with me for a minute.

Whenever you want to print "Hello" you can create an object from this class
and call its \verb+run()+ mehtod.

The sole purpuse in life of this class is to carry a method around.

This class is just a method, disguised as a class.

This class is called a FUNCTOR, a method disguised as a class.








\subsection{Nested Classes}

Now, in Java there is yet a better way to do this.

In Java, you can create classes inside another class, they are called NESTED
CLASSES.

They are used a lot when you want a class that is only going to be used inside
another class, but not outside it.

So instead of creating a new java file for this class, you can simply write
this class inside the class that is going to use it.

\codeinput{../java/Hello4.java}{Hello4.java}








\subsection{The power of Functors: passing methods to other methods}

Let's say we want to write a program that prints "hello" 5 times, adds 10 years
to Alberto and adds Alberto 3 times to a bag of Persons.

\codeinput{../java/Repeat1.java}{Repeat1.java}

Can you see the pattern here?

Wouldn't it be nice to write a method that DO SOMETHING N times in a row.

\begin{blackboard}
void repeat(XXXX something, int n) {
  for (int i=0; i<n; i++) {
    something();
  }
}

repeat(System.out.println("Hello"), 5);
repeat(alberto.addOneYear(), 10);
repeat(bag.add(alberto), 3);
\end{blackboard}

But in Java we can not pass a method as an input parameter to another method.

In Java, we can only pass primtive types and references to objects as input
parameters to a method.

So how do we solve this?

Well, we pass a functor, which is a method, disguised as an object:

\codeinput{../java/Repeat2.java}{Repeat2.java}

We can do this even better using Java anonymous classes.

An anonymous class is how you create an object from a class that you only need
for that single object, and then you forget about it forever.

In spanish there is a perfect way to describe them: "una clase anónima es una
clase de usar y tirar".

The best english translation I have found for this is, anonymous classes are
"one use classes" or "disposable classes".

You use them once to create a single object and then you throw them away.

For example, if you need an object that implements the Runnable interface, and
that prints hello when you call its run method, you can write something like this:

\begin{blackboard}
// FUNCTOR implemented as an ANONYMOUS CLASS
Runnable r = new Runnable() {
  public void run() {
    System.out.println("Hello");
  }
};
\end{blackboard}

This will create a new reference called r to an object of a class that
implements the interface Runnable (\verb+Runnable r+).

Then it will create a new object (\verb+new+) and point the reference r to it
(\verb+=+).

The class of this object is not in any Java file and it is not a defined as a
nested class later, instead you are providing the full source code for the
class just there in that very moment.

The will not have a name (that why they are called anonymous classes).

It will implement the \verb+Runnable+ interface.

Objects from that class will have a \verb+run()+ method that prints "hello".

Then, once the object is created, it will forget about that class forever.

\codeinput{../java/Repeat3.java}{Repeat3.java}











\section{GUIs}



\end{document}
