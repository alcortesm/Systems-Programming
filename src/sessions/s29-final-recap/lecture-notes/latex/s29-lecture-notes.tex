\documentclass[a4paper, 9pt]{extarticle}

\usepackage[notes]{style}

\newcommand{\realtitle}{Session 29 - Final Recap}

\newcommand{\separator}{\begin{center}%
\noindent\makebox[\linewidth]{\rule{0.75\paperwidth}{0.4pt}}%
\end{center}}

\begin{document}

\makebox[\linewidth]{\rule{\textwidth}{0.4pt}}
UC3M \hfill Alberto Cortés Martín\\
Systems Programming, 2014-2015 \hfill version: \today\\
\makebox[\linewidth]{\rule{\textwidth}{0.4pt}}
\begin{center}
  \Large{\realtitle}\\Lecture Notes
\end{center}
\makebox[\linewidth]{\rule{\textwidth}{0.4pt}}
\vspace{1cm}

\section{Final Exam 2014-2015}

\subsection{Problem 1}

GIVE 10 MINUTES TO READ THE PROBLEM TEXT.

YOU WILL HAVE TO EXPLAIN WHAT IS THE PROBLEM IS ABOUT IN 2 PARRAGRAPHS, AS IF
IT WERE A PROJECT YOU ARE WORKING ON.


\subsubsection{Discussion}

Problem description:
\begin{multicols}{2}
\begin{itemize}
  \item You should be able to create houses and shops, from their description
    (area, age, rooms\ldots) and be able to calculate the IBI and TRU of each
    one.

    For this, you need a \texttt{Housing} class, a \texttt{Establishment} class
    and a \texttt{CityCouncilTaxes} interface.
\end{itemize}

\columnbreak

\begin{itemize}
  \item You should be able to proccess several buildings at once and generate
    some statistics: average IBI and maximum TRU.

    For this, you need an \texttt{Estate} class.

\end{itemize}
\end{multicols}

\begin{center}
\includegraphics[width=14cm]{./img/estate-advanced/estate-advanced.eps}
\end{center}

WARNINGS: Size reduction of classes, protected \texttt{age}, common part of
\texttt{calculateIBI()} in the abstract class.

\newpage


Interesting things:

\begin{itemize}

  \item This problem deals with money, so probably most attributes should be
    \texttt{BigDecimal}. As you alredy know we usually do not use
    \texttt{BigDecimal}s in the exams, to save time and for better readability
    of your answers.

  \item Usually you should never implement the age of something as an int,
    because if you execute this program in a year from know, all the ages will
    be wrong. It is much better to use the classes \texttt{Date} or
    \texttt{Calendar} that are Java classes especifically designed to represent
    a moment of time. The age of something will be the date of today minus the
    date of its birthday.

    Despite that, we still use ints for age in the exams to save time and for
    simplicity.

\end{itemize}


Test program:

Let us create a program to test our solution to this problem, first, let us see
if we understand the problem by filling the blanks in this table:

\begin{center}
\begin{tabular}{|c|l|c|c|}
  \hline
  Estate & Properties & IBI & TRU \\
  \hline
  h1     & Housing, 5 years, 100 $m^{2}$, 1 room & \textbf{1200} & \textbf{15} \\
  h2     & Housing, 10 years, 100 $m^{2}$, 3 rooms & \textbf{1000} & \textbf{45} \\
  e1     & Establishment, 15 years, 100 $m^{2}$, 1 m long & \textbf{1100} & \textbf{20} \\
  e2     & Establishment, 20 years, 100 $m^{2}$, 2 m long & \textbf{1000} & \textbf{40} \\
  \hline
\end{tabular}
\end{center}

Now, let us write a program to test this and the print method from the last section of the problem:

\codeinput{../java/EstateTest.java}{EstateTest.java}


\newpage

\subsubsection{Section 1}

\codelinesinput{../java/Estate.java}{Estate.java}{1}{20}


\subsubsection{Section 2}

\codeinput{../java/Housing.java}{Housing.java}
\codeinput{../java/Establishment.java}{Establishment.java}


\newpage

\subsubsection{Section 3}

\begin{multicols}{2}
Pay attention to:

\begin{itemize}

  \item What is the best class to place the static method?

  \item Bad input argument (null array)

  \item Beware of arrays with holes!.

  \item Avoid traversing the array twice (calculate the average and
    the max in a single run).

  \item Do not call \emph{calculate} methods more than once, they may be slow!
    (not the case of this example, but this is a very good rule of thumb).

  \item Assing and compare in a single line, super common idiom!

  \item Division by ZERO!

\end{itemize}
\columnbreak
\codelinesinput{../java/Estate.java}{Estate.java}{22}{54}
\end{multicols}


\subsection{Using a List Instead of an Array}

\begin{multicols}{2}
\begin{itemize}

  \item As size is known from the beginning, exceptions can be thrown at the
    beginning.


  \item (getting vs. removing) Emptying the list would be faster than getting
    (for a linked list), unless you know about \emph{iterators} (more on this
    later), but please do not do it, I may want to use my list later for
    something!

  \item There is no need to check for holes in the array if the array does not
    allow for nulls. No need for the estateCounter either.

  \item Do not divide in each iteration, wait until the end for efficiency.

  \item List implemented with linked nodes are expensive to traverse, avoid
    \texttt{get}ting elements more than once in each iteration of the loop.

    If you want to learn a more efficient way of solving this issue, please do
    some research on \emph{iterators}, they are quite awesome!.

\end{itemize}
\columnbreak
\codelinesinput{../java/Estate.java}{Estate.java}{56}{78}
\end{multicols}


\end{document}
