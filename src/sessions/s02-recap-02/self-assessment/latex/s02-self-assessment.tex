\documentclass[a4paper, 11pt]{article}

\usepackage[assessment]{style}

\newcommand{\realtitle}{Session 02 - Recap 2}

\begin{document}

\makebox[\linewidth]{\rule{\textwidth}{0.4pt}}
UC3M \hfill Alberto Cortés Martín\\
Systems Programming, 2014-2015 \hfill version: \today\\
\makebox[\linewidth]{\rule{\textwidth}{0.4pt}}
\begin{center}
  \Large{\realtitle}\\Self-Assessment
\end{center}
\makebox[\linewidth]{\rule{\textwidth}{0.4pt}}
\vspace{1cm}

\section{Today}

Write a program called \texttt{Today} that prints the current date using the
\verb+java.util.Date+ class.

\begin{cmd}
$ java Today
Thu Jan 29 16:35:51 CET 2015
\end{cmd}

\solutioninput{../java/Today.java}

\section{Magic Numbers}

Write a program called \texttt{DaysInMonths} that prints the number of days in
each month of the current year.

Avoid using magic numbers.

Repeat this exercise twice, first using only \verb+if+ statements, then using
only a \verb+switch+ statement.

For simplicity, you can assume that the current year is not a leap year.

\begin{cmd}
$ java DaysInMonths
1 31
2 28
3 31
4 30
5 31
6 30
7 31
8 31
9 30
10 31
11 30
12 31
\end{cmd}

\solutioninput{../java/DaysInMonths.java}

\section{Limits of integer values}

\subsection{}

What is the value of the biggest \texttt{int} in Java?
Why?

\begin{solution}
2,147,483,647

Java ints are 32 bit values. The first half of them, from $0$ to $2^{31}-1$,
are used for positive integers. The other half of them, from $2^{31}$ to
$2^{32}-1$, are used for negative integers.

Therefore, the maximum int in Java is $2^{31}-1$ = 2\_147\_483\_647.
\end{solution}

\subsection{}

The \texttt{Integer} class (from the Java API) has an static field
that stores the maximum value of an \texttt{int} in Java.
What is the name of that field?

Write a Java program that prints its value.

\begin{solution}
Integer.MAX_VALUE or java.lang.Integer.MAX_VALUE if you prefer the long name.
\end{solution}
\solutioninput{../java/IntMaxValue.java}
\solutioninput{../java/IntMaxValue.output}

\subsection{}

Write a Java program that prints the mininum and maximum values of
a Java \texttt{byte}, \texttt{short}, \texttt{int} and \texttt{long}.

\solutioninput{../java/Limits.java}
\solutioninput{../java/Limits.output}

\subsection{}

What happens when you overflow an \texttt{int}?
As an example, what is the result of adding \newline
\texttt{Integer.MAX\_VALUE} and 1?

\solutioninput{../java/AddOverflowDemo.java}
\solutioninput{../java/AddOverflowDemo.output}

\subsection{}

Does Java warn you about the failed sum in the previous exercise
or is the programmer responsible for checking arithmetic overflows?

\begin{solution}
Java does NOT warn you about integer sum overflows.

The programmer is resposible for checking arithmetic overflows.
\end{solution}

\subsection{}

Write a Java method that returns the sum of its two \texttt{int} arguments
and that throws an exception if an overflow occurs.

Test your method with the following operations:

\begin{itemize}
  \item $\texttt{Integer.MAX\_VALUE} + 1$
  \item $\texttt{Integer.MAX\_VALUE} + (-1)$
  \item $\texttt{Integer.MIN\_VALUE} + 1$
  \item $\texttt{Integer.MIN\_VALUE} + (-1)$
  \item $1,000,111,222 + 2,000,111,222$
\end{itemize}

\solutioninput{../java/AddWithOverflowDetection.java}
\solutioninput{../java/AddWithOverflowDetection.output}

\section{Precision of floating point values}

\subsection{}

Compile and run the following Java program an explain why its output looks
weird:

\blackboardinput{../java/FloatPrecision.java}
\solutioninput{../java/FloatPrecision.output}
\begin{solution}
Real numbers are uncountable.  This means there are infinite real numbers even
in the smallest interval between two other real numbers. Therefore, it is
imposible to represent all real numbers in a computer between a given minimum
and a maximum, as there are infinite posible values.

Java floats and doubles are stored in memory usgin "IEEE 754 binary floating
point representation".  In this format, real numbers are approximated by a
finite set of real numbers that represent them.  For example, the value 0.1
cannot be represented exactly, so and approximate value is stored instead.

This aproximation is quite good and you can print 0.1 as if it where the exact
number.  But the approximation error is still there and can be amplified by
operating on it.

When adding an approximation of 0.1 three times, we are making the
approximation error so big, that Java actually recognices the result as another
number (the next number in the finite set of representatives).
\end{solution}

\section{Strings}

\subsection{}

Write an interactive Java program that asks the user for the
"Answer to The Ultimate Question of Life, the Universe, and Everything".

The program must print "RIGHT!" if the answer is either
\texttt{42} or \texttt{Forty-two} (case insensitive).
and must print "WRONG!" otherwise.

Tip: You can learn about Java IO streams
\href{https://docs.oracle.com/javase/tutorial/essential/io/streams.html}{here}.

\solutioninput{../java/UltimateQuestion.java}
\solutioninput{../java/UltimateQuestion.output}

\subsection{}

Write a \textsl{Unix filter} that translates
\begin{itemize}
\item all 'a's into '4's
\item all 'e's into '3's
\item all 'i's and 'I's into '1's
\item and all 'o's into '0's
\end{itemize}

Tip: A Unix filter is a programa that keeps reading lines from the standard
input until and End-Of-File character (EOF) is found. Each time a line is
readed, the program translates it acording to some predefined rules and prints
the result of that translation.

Use the following execution example as a reference (user inputs are lines 1, 2,
4 and 6. The program outputs are lines 3, 5 and 7):

\begin{blackboard}
; java Eleet
hi there!
h1 th3r3!
noob hacker
n00b h4ck3r
I hate Java
1 h4t3 j4v4
\end{blackboard}

\solutioninput{../java/Eleet.java}

\section{Arrays}

\subsection{}

Write a \texttt{daysOfTheWeek()} method that returns an array of strings,
holding the names of the days of the week in chronological order as per
\href{http://en.wikipedia.org/wiki/ISO_8601}{ISO 8601} (i.e., Mondays come
  first).

\subsection{}

Write a \texttt{print(String[] a)} method that prints the strings in an array
in the same order as they appear in the array, using ", " as a separator and a
'.' as the terminator.

Tests this method using the output of the \texttt{daysOfTheWeek} method from
the previous exercise. The output should look exactly like this:

\begin{blackboard}
Monday, Tuesday, Wednesday, Thursday, Friday, Saturday, Sunday.
\end{blackboard}

\subsection{}

Write a \texttt{reverse(String[] a)} method that returns a copy of its input
array, but with its elements in reverse order. The input array must be left
untouched.

Test this method by printing the days of the week in reverse order, using the
methods above. The output should look exactly like this:

\begin{blackboard}
Sunday, Saturday, Friday, Thursday, Wednesday, Tuesday, Monday.
\end{blackboard}

\subsection{}

Write a \texttt{sort(String[] a)} method that returns a copy of its input
array, but with its elements sorted in alphabetical order. The input array
should be left untouched.

Test this method by printing the days of the week in reverse alphabetical
order, using the methods above. The output must look exactly like this:

\begin{blackboard}
Wednesday, Tuesday, Thursday, Sunday, Saturday, Monday, Friday.
\end{blackboard}

\subsection{}

Write a \texttt{shuffle(String[] a)} method that returns a copy of its input
array, with its elements in a random order. The input array must be left
untouched.

Test this method by printing the result of five invocations to this method
using the days of the week as its input, as shown bellow:

\begin{blackboard}
Wednesday, Sunday, Thursday, Saturday, Friday, Monday, Tuesday.
Sunday, Wednesday, Saturday, Thursday, Tuesday, Friday, Monday.
Monday, Wednesday, Sunday, Friday, Tuesday, Saturday, Thursday.
Wednesday, Saturday, Tuesday, Thursday, Friday, Monday, Sunday.
Wednesday, Tuesday, Friday, Saturday, Thursday, Sunday, Monday.
\end{blackboard}

Tip: you will find the standard Java \texttt{java.util.Random} class quite
useful to generate random numbers.

\subsection{}

Write a method that takes an array of strings as its input, and keeps shuffling
them and printing them until the they end up in the same order as the original
one.

The output might look something like this:

\begin{blackboard}
Monday, Sunday, Thursday, Saturday, Friday, Tuesday, Wednesday.
Thursday, Sunday, Tuesday, Wednesday, Saturday, Friday, Monday.
Tuesday, Friday, Thursday, Wednesday, Sunday, Monday, Saturday.
Tuesday, Friday, Thursday, Wednesday, Monday, Sunday, Saturday.
Sunday, Saturday, Tuesday, Wednesday, Monday, Friday, Thursday.
Monday, Thursday, Tuesday, Wednesday, Friday, Sunday, Saturday.
.
.     [some thousand lines later...]
.
Monday, Sunday, Wednesday, Saturday, Tuesday, Friday, Thursday.
Monday, Sunday, Tuesday, Thursday, Saturday, Friday, Wednesday.
Wednesday, Monday, Sunday, Friday, Tuesday, Thursday, Saturday.
Tuesday, Sunday, Thursday, Friday, Monday, Wednesday, Saturday.
Wednesday, Sunday, Saturday, Monday, Tuesday, Thursday, Friday.
Monday, Tuesday, Wednesday, Saturday, Friday, Sunday, Thursday.
Monday, Tuesday, Wednesday, Thursday, Friday, Saturday, Sunday.
\end{blackboard}

\solutioninput{../java/DaysOfTheWeek.java}

\end{document}
