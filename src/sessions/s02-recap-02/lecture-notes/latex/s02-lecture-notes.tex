\documentclass[a4paper, 9pt]{extarticle}

\usepackage[notes]{style}

\newcommand{\realtitle}{Session 02 - Recap 2}

\begin{document}

\makebox[\linewidth]{\rule{\textwidth}{0.4pt}}
UC3M \hfill Alberto Cortés Martín\\
Systems Programming, 2014-2015 \hfill version: \today\\
\makebox[\linewidth]{\rule{\textwidth}{0.4pt}}
\begin{center}
  \Large{\realtitle}\\Lecture Notes
\end{center}
\makebox[\linewidth]{\rule{\textwidth}{0.4pt}}
\vspace{1cm}

\section{Last Session Review}

\begin{blackboard}
The World according to Java
  - Stack
    + Stack frames
    + Pass by value
    + method parameters are automatic variables
    + this variable
  - Heap
  - Static

Java Primitive Data Types
  - booleans
  - byte, short, int, long
  - float, double
  - char
  - Literals
  - Operators

Program Arguments
Remainder Operator
Primes
Flowcharts
Measuring execution time of programs
\end{blackboard}

\newpage
\section{Constant Variables}

This looks like an oximoron, but it is not.

To declare a variable so that, once initiliazed, its value can not be changed,
use the \verb+final+ keyword.

\begin{blackboard}
final int DAYS_IN_A_WEEK = 7;
DAYS_IN_A_WEEK = 8; // compilation error: cannot assign a value to final variable

final char ALPHA = '\u03B1'; // '\uXXXX' way to write chars by its UNICODE value

final int NUM_SESSIONS;
NUM_SESSIONS = 29;
NUM_SESSIONS = 34; // compilation error: cannot assign a value to final variable
\end{blackboard}

The use of uppercase letters in the names of constants is a tradition in
programming, it allows to know at a glance, which variables you can modify
later in your programs, and which not.






\section{Java Types Again}

\begin{center}
\includegraphics{img/variables_and_references/variables_and_references.eps}
\end{center}

\begin{center}
\includegraphics{img/comparing/comparing.eps}
\end{center}









\section{Java References}

Are used to store the memory address of Objects.

\begin{blackboard}
Declaration:

  Name_of_class Name_of_reference ; // allocates memory and alias

  Object a;       (Dibujos)
  String b;
  Date   c;

Operators:
  = : assignment, store a memory address in the reference (points the reference
      to something)
  == : Equal to, compare two references (not two Objects), returns a boolean
  != : Not Equal to, the opposite.
\end{blackboard}

\section{Java Objects}

\subsection{The problem}

If your program deals with integers (like calculating prime numbers) you will
create ints and operate on them.

celsius degrees $\rightarrow$ doubles.

But what if your program deals with points?, dates?, or words?, or with
people?, or airplanes?

You will like to create points, dates, phrases, people and airplanes and
operate on them, but you cannot do that with primitive data types.

SIMPLE SOLUTION: simulate complex things using a bunch primitive types:

\begin{blackboard}
- a point is 2 ints (x and y)
- a date is 3 ints (year, month and day)
- a word is several chars
- a person may be a name (word) and birthdate (date)
- a plane is thousands of ints and chars
\end{blackboard}

And, how do you operate on them?

\begin{blackboard}
  - plane1 * plane2 ?

  - person1 / person2 ?
\end{blackboard}

So using operators don't make sense with this complex things.

SOLUTION:
\begin{itemize}
\item lets call this complex things OBJECTS
\item lets allow for ways to create them (CONSTRUCTORS instead of literals)
\item lets operate on them with METHODS (instead of operators).
\end{itemize}

Objects belong to a class, which defines:

\begin{itemize}

  \item its constructors (how to create a new object)

  \item its methods (what you can do to an object)

  \item and sometimes fields (its internal data)

\end{itemize}

\begin{blackboard}
myPlane and yourPlane are both planes.
Alberto and Alicia are both persons.
Madrid and New York are both words or points in a map.

Plane, person, String and Points are classes.

MyPlane, yourPlane, Alberto, Alicia, Madrid and New York are Objects from
that classes.
\end{blackboard}

How do you know what constructors, methods and fields there are in a class? The
Java API.

Show the Java API and look for:

\begin{itemize}
  \item \verb+String.isEmpty()+
\end{itemize}





\section{Methods}

Methods are what you can do to an object.

The list of methods of a class is the list of things you can ASK an object to
do.

So calling methods on an object is like giving orders to your minions so they
can do your work for you.

A good designed class will have interesting methods, useful to the programmer.

So to write a program, just create enought minions (objects) of the appropiate
classes, and start giving order to them.

For example:

\begin{blackboard}
What would you like to order to a String:

String a = new String("Hello");
String b = new String("Bye");

- tell me how long you are: a.length() // 5
- what is your first character: a.charAt(0) // H
- what is your thrid character: a.charAt(2) // first l
- what is your last character: a.charAt(a.length() - 1); // o
- build me a new Uppercase version of yourself: a.toUpperCase(); // HELLO
- are you empty: a.isEmpty(); // false
- do you have the same text as b: a.equals(b); // false
- concat yourself with b: a.concat(b); // HelloBye
- do you end with "lo": a.endsWith("lo"); // true
- replace "l" with "X": a.replace('l', 'X'); // Hexxo
\end{blackboard}

Methods are defined by:

\begin{blackboard}
- the class they belong to
- their access modifiers
- if they are static or not
- their return value
- their name
- their parameters
- the exceptions they throw

public String Object.toString()  <- no arguments, no exceptions
public char   String.charAt(int index) throws IndexOutOfBoundsException
public void   Date.setYear(int newYear)
\end{blackboard}

CLASS

An String have different methods than a Point or a Person.

So methods belong to the class.

ACCESS MODIFIERS

\verb+public, private, protected+.

Don't worry about access modifiers for now.

PARAMETERS:

Methods need some information from you to operate:

\begin{blackboard}
String a = new String("Hello");
a.charAt(3); // what char do you want: the forth
a.equals(b); // what String do you want me to comparte to: b
\end{blackboard}

When you call a method a new stack frame is created, and variables for their
arguments are allocated inside that frame.

In addition to that, a \verb+this+ reference is created pointing to the object
you are operating on.

These variables are initialized with the contents of the variables or
references used in its invocation, by copying their values. This is call "pass
by value" and has strong implications.

\newpage

\begin{multicols}{2}
\begin{blackboard}
// what will this program print?


class Bla {
  static void increment(int a) {
    a++;
  }

  public static void main(String[] args) {
    int a = 0;
    increment(a);
    System.out.println(a);
  }
}
\end{blackboard}
\columnbreak
\begin{blackboard}
// what will this program print?
import java.awt.Point;

class Foo {
  static void increment(Point a) {
    a.move(3, 7);
  }

  public static void main(String[] args) {
    Point a = new Point(0, 0);
    increment(a);
    System.out.println(a);
  }
}
\end{blackboard}
\end{multicols}

Other programming languages also allow to "pass by reference", but Java only
allows you to pass by value.

Other methods don't need any data to operate:

\begin{blackboard}
a.length()
a.isEmpty()
\end{blackboard}

Their parameters lists is empty, so no authomatic variables are created in the
stack, except for \verb+this+.


RETURN VALUE:

Some methods are questions you make to the object:

\begin{blackboard}
String a = new String("Hello");
a.isEmpty(); // false
a.charAt(3); // second l
a.toUpperCase(); // HELLO
\end{blackboard}

The return value is the answer to that question.

Other methods are just operations you request on an object.
They don't return anything, in Java, they return \verb+void+.

\begin{blackboard}
Point a = new Point(2, 4);
a.move(4, 8);
\end{blackboard}

EXCEPTIONS

Sometimes a method fails, so a value cannot be returned or the operation can
not be performed.

When this happens, the method throws an exception instead of returning a value:

\begin{blackboard}
String a = new String("Hello");
a.charAt(-3); // throws an IndexOutOfBoundsException
a.charAt(15); // throws an IndexOutOfBoundsException
a.equals(null); // throws a NullPointerException
\end{blackboard}

As a programmer you WANT to know when a method fails, to fix your inputs to the
method and try again.

You can know when a method throws an exception using a \verb+try...catch+ clause:

\begin{multicols}{2}

\begin{blackboard}
void print8thCharacter(String s) {
  char c;
  c = s.charAt(7);
  System.out.println(c);
}
\end{blackboard}

\columnbreak

\begin{blackboard}
void print8thCharacter(String s) {
  char c;
  try {
    c = s.charAt(7);
  } catch (IndexOutOfBoundsException ex) {
    System.out.println("Error: The string is too short");
    return;
  }
  System.out.println(c);
}
\end{blackboard}
\end{multicols}

If you don't catch the exception the receiving method will fail, throwing that
same exception to his caller, and so on and so forth until main is reached and
the program exit with an Error to the user, and all the stackframes are printed
on the screen (so you know where the problem was).

STATIC vs. NON-STATIC:

Static methods don't operate on objects.

They belong to the class, not to its objects.

They are utilities for the class. Like really usefull actions you would like to
do, that are somehow related with the class.

\begin{blackboard}
What would be the string representing the number 7?
C2F.c2f(10.0); // 50.0
Math.abs(-12); // 12
\end{blackboard}

It is the same with static fields.  There is only copy of static fields, no
matter how many objects of that class you have created.

\begin{blackboard}
System.out // is and static field of the class System
  // you have not really create any objects from that class, but you can still
  // use its statics fields (and methods if it has any).
\end{blackboard}

SIGNATURES

A method signature is the TRUE name of a method, it includes its name and the
lists of arguments:

\begin{blackboard}
toString()
charAt(int)
\end{blackboard}

There can not be two methods with the same signature on the same class.
Otherwise, Java will not know which method are you calling.

Other programming languages don't include the number of arguments in the
signature. So there can not be two methods with the same name.

But in Java, as method signatures includes the parameter lists, you can have
several methods with the same name but different arguments.

\begin{blackboard}
println(boolean a)
println(int a)
println(String a)
max(int a, int b)
max(int a, int b, int c)
max(float a, float b)
\end{blackboard}

This is called method overloading

\subsection{Constructors}

A constructor is a special kind of method.

You use them for initializing an Object, just after it is allocated in memory.

Just after the "new".

They are always called after the class.

You can use method overloading with constructors, so you can have different
ways to initialize and object.

\begin{blackboard}
Point p1 = new Point();
Point p2 = new Point(3, 5);
Point p3 = new Point(p2);

// constructors can also throw exceptions
Point p4 = null;
Point p5 = new Point(p4); // runntime error: NullPointerException!
\end{blackboard}

\section{Essential Classes}

There are 4024 classes in the Standard Edition Java API.

The most important one is String, for two reasons:

\begin{itemize}

  \item Is widely used: most programs deal with words in some way or another.

  \item It has lots of sintactic sugar: exceptions to Java rules.

\end{itemize}

\subsection{String}

\begin{itemize}

  \item{Literals}:

\begin{blackboard}
String a = "hello";
String b = new String("hello");
\end{blackboard}

  String literals create an String if it has not been used before in the program.

  But reuse that string afterwards.

\begin{blackboard}
String c = "hello";
boolean z1 = (a == b); // false
boolean z2 = (a == c); // true

// Empty string
String d = "";
// String with "
String e = "String literals are surrounded by the character \"";
\end{blackboard}

\item{Inmutable}

  Strings are inmutable, you can not change the contents of an String once it
  has been created.

  All the methods of the string class, that seems to modify an string, return a
  new String instead, leaving the original unmodified.

\begin{blackboard}
String a = "hello";
String b = a.toUpperCase();
\end{blackboard}

  Very few objects in Java are inmutable.

  A mutable version of String exits, it is called a StringBuilder.

\item{toString() method}

  All objects in Java have this method, it returns a String, a textual
  representation of the object.

\begin{blackboard}
Date a = new Date();
String b = a.toString(); // Thu Jan 29 13:44:41 CET 2015
\end{blackboard}

\item{Operators}:

    Strings has methods, just like any other object.

\begin{blackboard}
String a = "hello";
boolean b = String.isEmpty(); // false
String b = a.concat(" and bye"); // "hello and bye"
\end{blackboard}

  But they also have one operator \verb'+' (no other object has that):

\begin{blackboard}
String a = "hello";
String b = " and bye";
String c = a + b;
\end{blackboard}

  When \verb+adding+ anything to a String, it gets converted into an String, if
  it is a primitive type, it generates its String representation, if it is a
  reference type, .toString() method is called.

\begin{blackboard}
String a = 7 + " " + 32.0 + " " + 'a' + " " + true + " " + new Date();
// "7 32.0 a true Thu Jan 29 13:44:41 CET 2015"
\end{blackboard}

When using the \verb.+. operator, the JVM internally creates a StringBuilder,
concats everything and call .toString() over that StringBuilder.

\end{itemize}



\subsection{Primitive Data Types Wrappers}

TODO: para required reading.

\subsubsection{Autoboxing and Unboxing}

\section{Control Flow statements}

\subsection{Conditionals}

\subsubsection{If-then}

Evaluate a block if a condition is true.

\begin{multicols}{2}
\begin{blackboard}
        if (condition) {
            actions;
        }

if (b != 0) {
    result = a / b;
}
\end{blackboard}

\columnbreak

  \begin{center}
    \includegraphics[height=5cm]{img/if/if.eps}
  \end{center}
\end{multicols}


\subsubsection{If-then-else}

\begin{multicols}{2}
\begin{blackboard}
        if (condition) {
            actions-1;
        } else {
            actions-2;
        }

if (b != 0) {
    result = a / b;
} else {
    System.err.println("ERROR: division by zero");
    System.exit();
}

if (a > b) {
  return a;
} else {
  return b;
}
\end{blackboard}

\columnbreak

  \begin{center}
    \includegraphics[height=5cm]{img/if_else/if_else.eps}
  \end{center}
\end{multicols}


\subsubsection{Switch}

Multiple comparison.

Works with byte, short, char and int (using == to compare)

Also with Strings, Byte, Short, Integer, Char (using .equals() to compare)

\begin{multicols}{2}
\begin{blackboard}
        switch (var) {
            case value1:
                actions-1;
            case value2:
                actions-2;
            .
            .
            .
            case valueN:
                actions-N;
            default:
                actions-default;
        }
\end{blackboard}

\columnbreak

  \begin{center}
    \includegraphics[width=8cm]{img/switch/switch.eps}
  \end{center}
\end{multicols}

A \verb+break+ will jump to the end of the switch.

Ejemplo:

\begin{blackboard}
Write a program that prints the number of working hours per day in the week.

Example:

$ java WorkHours
1 0 // sundays
2 8
3 8
4 8
5 8
5 7 // fridays
6 0 // saturdays
\end{blackboard}

DEMO:

(print days from 1 to 7)

(function to return 8)

(add exception to function)

(return 0, 7 or 8)

(remove magic numbers using finals)

(join days with the same amount of hours)

(join the exceptions and move them to the beginning)

(add another function doing the same with a switch)

(join cases in a single line)

\blackboardinput{../java/WorkHours.java}

\section{Review}

\begin{blackboard}
Constants (final)
  Magic numbers

References
  Declaration
  Operators

Objects
  Methods
    stack frames
    parameters
      automatic variables
      this
    return values
    exceptions
    static
    overloading
  Constructors
  Essential Classes
    Strings

Control Flow statements
  conditionals
    if-then
      if-then-else
    switch
\end{blackboard}

\end{document}
