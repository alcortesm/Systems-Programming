\documentclass[a4paper, 9pt]{extarticle}

\usepackage[notes]{style}

\newcommand{\realtitle}{Session 27 - Midterm 2 recap}

\newcommand{\separator}{\begin{center}%
\noindent\makebox[\linewidth]{\rule{0.75\paperwidth}{0.4pt}}%
\end{center}}

\begin{document}

\makebox[\linewidth]{\rule{\textwidth}{0.4pt}}
UC3M \hfill Alberto Cortés Martín\\
Systems Programming, 2014-2015 \hfill version: \today\\
\makebox[\linewidth]{\rule{\textwidth}{0.4pt}}
\begin{center}
  \Large{\realtitle}\\Lecture Notes
\end{center}
\makebox[\linewidth]{\rule{\textwidth}{0.4pt}}
\vspace{1cm}

\section{2008-2009, Extraordinary call, Problem 2, Section 1}

GIVE SOME TIME TO READ THE PROBLEM TEXT.

There is one interesting thing about this problem:

\begin{itemize}

  \item It is a problem about stacks, but you do not have to implement an stack,
    instead, you are given an already working stack and you
    have to use it to do something interesting.

\end{itemize}


\subsection{Implementing a simple stack}

Let us make the most of this problem by reviewing how to implement an stack.

This is not part of the original problem, but it is a nice exercise to practice
for the exams.

I have written the following interface based on the stack defined in the problem:

\codeinput{../java-simple/Stack.java}{java-simple/Stack.java}

Take your time to read the interface carefully, because there are some
interesting things about it.

GIVE SOME TIME HERE TO THINK AND WRITE DOWN THINGS THAT LOOK WEIRD.

There are 6 interesting things about the stack in this problem:

\begin{enumerate}

  \item You cannot use this stack to store Java objects.

    This is not a generic stack. This means we can not insert
    \texttt{Points}, \texttt{Circles} or \texttt{Person}s (people).

    It is not even an stack of java \texttt{Objects}.

    It is an stack of ints, a primitive type.

    This will have some interesting immplications with the garbage collector,
    as we will see later.

  \item it lacks a \texttt{size} method, but that is OK, we will survive without it.

  \item it lacks a \texttt{peek} method, but that is OK, we will survive without it.

  \item the \texttt{toString} method is not mentioned in the problem or in the
    interface. We are sure that every class that implements this interface will
    have a \texttt{toString} method, because all Java classes have that method,
    but we can not be sure that the inherited \texttt{toString} method from
    \texttt{Object} has been overriden to do something meaningfull for a stack,
    so we can not count on that for printting the stack.

  \item the \texttt{push} method does not throw an
    \texttt{StackOverflowException}, this can only mean that the stack must be
    infinite.

  \item What should the \texttt{pop} method return when the stack is empty? a null?, a -1?.

    The nice thing to do is to throw a \texttt{java.util.EmptyStackException},
    there is no need to say so in the definition of the interface, but it is
    good manners to do so. CHANGE THE INTERFACE TO THROW THIS EXCEPTION.

\end{enumerate}

\codeinput{../java/Stack.java}{java/Stack.java}





\subsubsection{Implementing an Array Based Stack}

OK, so let us implement this simple stack using an array.

Do you remember how to do that?

Let us say we have a stack with four elements, DRAW THAT.

How are we going to store that elements in our array? (left-to-right or
right-to-left?)

How are we going to keep track of the last one?

OK, let us quickly write the \texttt{ArrayStack} class:

THE STUDENTS DICTATE THE IMPLEMENTATION

\begin{multicols}{2}
\codeinput{../java/ArrayStack.java}{java/ArrayStack.java}
\end{multicols}

\separator

\begin{multicols}{2}

BEGIN WITH THIS:

\codeinput{../java-simple/ArrayStack.java}{java-simple/ArrayStack.java}

1) FIRST ASK FOR THE CONSTRUCTOR AND THE empty METHOD

2) THEN ASK FOR THE push METHOD, START WITH ITS LAST LINE, THEN ADD THE grow METHOD.

3) THEN ASK FOR THE pop METHOD, START WITH ITS LAST LINE, THEN ADD THE shrink METHOD.

4) TEST THE CLASS WITH:

\begin{blackboard}
$ java StackTest A
$
\end{blackboard}
\end{multicols}





\subsubsection{Implementing a Linked Stack}

Let us say we have a stack with four elements, DRAW THAT.

How are we going to store that elements in out linked stack? (left-to-right or
right-to-left?)

OK, let us quickly write a \texttt{LinkedStack} class:

THE STUDENTS DICTATE THE IMPLEMENTATION

\begin{multicols}{2}
\codeinput{../java/LinkedStack.java}{java/LinkedStack.java}
\end{multicols}

\separator

\begin{multicols}{2}

BEGIN WITH THIS:

\codeinput{../java-simple/LinkedStack.java}{java-simple/LinkedStack.java}

\columnbreak

1) FIRST SEE IF THEY WANT TO USE GETTERS AND SETTER FOR THE NODES

2) ASK FOR THE CONSTRUCTOR AND THE empty METHOD

2) THEN ASK FOR THE push METHOD

3) THEN ASK FOR THE pop METHOD

4) TEST THE CLASS WITH:

\begin{blackboard}
$ java StackTest L
$
\end{blackboard}
\end{multicols}

\newpage







\subsection{The Show Method}

OK, enough with the stacks, remember that the original exam problem did not ask
us to write any stack, so we have been loosing time here.

In fact, the original problem give us a stack written by others, and ask us to
do something with it.

Let us read again what we need to do.

GIVE SOME TIME TO RE-READ THE PROBLEM.

OK, so we have an unknow stack class and we do not know who or how it is
implemented.

Let us call that class the \texttt{ExamStack}. I have a secret stack class
right here, but I am not going to show you its code.

I have add a few more lines to the \texttt{Exam} \texttt{main} method, so it is
more interesting:

\begin{multicols}{2}
\codeinput{../java-simple/Exam.java}{java-simple/Exam.java}
\columnbreak
WHAT DOES THIS PROGRAM DO?
\begin{blackboard}
$ java Exam
-----------
-- again --
-----------
-----------
3
-- again --
3
-----------
-----------
5
3
-- again --
5
3
-----------
-----------
7
5
3
-- again --
7
5
3
-----------
\end{blackboard}

WHY ARE THINGS PRINTED IN THIS ORDER? -> desing choice
\end{multicols}





\newpage
\subsubsection{Destructive Iterative Version}

Let us forget about the exam for a second, and let us try to write a much simpler version of show.

Let us make it iterative and let us not care about what happens to the elements
in the stack after showing them.

We will call this version of the method \texttt{show1}.

WRITE THE show1 METHOD.

\begin{multicols}{2}
\codelinesinput{../java/Exam.java}{java/Exam.java}{34}{45}
\columnbreak
\begin{blackboard}
$ java Exam
-----------
-- again --
-----------
-----------
3
-- again --
-----------
-----------
5
-- again --
-----------
-----------
7
-- again --
-----------
\end{blackboard}
WHAT IS WRONG? show empties the stack
\end{multicols}



\newpage

\subsubsection{Non-Destructive Iterative Version}

HOW DO WE AVOID EMPTYING THE STACK WHILE WE PRINT ITS ELEMENTS?

We need some kind of storage to store the elements and then push them back into
the stack after printting them.

Well, the most simple storage class we know is the Java \texttt{ArrayList} class.

WRITE THE show2 METHOD, THAT PRINTS THE STACK WITHOUT DESTROYING IT, BY USING AN arraylist.

\begin{multicols}{2}
\codelinesinput{../java/Exam.java}{java/Exam.java}{47}{68}
\columnbreak
\begin{blackboard}
$ java Exam
-----------
-- again --
-----------
-----------
3
-- again --
3
-----------
-----------
5
3
-- again --
3
5
-----------
-----------
7
5
3
-- again --
3
5
7
-----------
\end{blackboard}
WHAT IS WRONG NOW? shows reverse the order of the elements in the stack
\end{multicols}


\newpage

We can solve this by adding the elements in the reverse order or getting
them in the reverse order.

WHICH VERSION IS FASTER?

Difficult to say, both ways are slow.

WRITE show3, THAT PRINTS THE STACK USING AN ARRAYLIST AND LEFT THE STACK UNMODIFIED.

\begin{multicols}{2}
\codelinesinput{../java/Exam.java}{java/Exam.java}{70}{91}
\columnbreak
\begin{blackboard}
$ java Exam
-----------
-- again --
-----------
-----------
3
-- again --
3
-----------
-----------
5
3
-- again --
5
3
-----------
-----------
7
5
3
-- again --
7
5
3
-----------
\end{blackboard}
NOW THE PROGRAM IS WORKING.
\end{multicols}

But in an exam we usually are not allowed to use external classes, this is, we cannot use a Java \texttt{ArrayList}.

WHAT CAN WE USE INSTEAD?

\separator
\newpage

Let us try with an array of ints.

WRITE show4, THAT PRINTS THE STACK USING AN ARRAY OF INTS AND LEFT IT UNMODIFIED.

\begin{multicols}{2}
\codelinesinput{../java/Exam.java}{java/Exam.java}{93}{124}
\columnbreak
\begin{blackboard}
$ java Exam
-----------
-- again --
-----------
-----------
3
-- again --
3
-----------
-----------
5
3
-- again --
5
3
-----------
-----------
7
5
3
-- again --
7
5
3
-----------
\end{blackboard}
NOW THE PROGRAM IS WORKING.
\end{multicols}

But I think this is too complicated, having to make the array bigger and all that stuff.

We can write a much simpler solution using another storage class that we are allowed to use in this exam.

That is, using a stack.

\newpage

Let us use an aunxiliary stack to store the elements extracted from the original stack when printting:

WRITE show5, THAT PRINTS THE STACK USING AN AUXILIARY STACK.

\begin{multicols}{2}
\codelinesinput{../java/Exam.java}{java/Exam.java}{126}{144}
\columnbreak
\begin{blackboard}
$ java Exam
-----------
-- again --
-----------
-----------
3
-- again --
3
-----------
-----------
5
3
-- again --
5
3
-----------
-----------
7
5
3
-- again --
7
5
3
-----------
\end{blackboard}
NOW THE PROGRAM IS WORKING.
\end{multicols}

So now, we have a show method, that prints an stack, without modifiying its
contents. And it is done without using any other class not mentioned in the
exam and it is also much easier than using an array.

The problem is that the exam is asking us to do this RECURSIVELY.


\newpage
\subsubsection{Destructive Recursive Version}

So, let us try to solve this problem recursively.

Let us begin with a simpler problem:

WRITE A show6 METHOD THAT PRINTS AN STACK RECURSIVELY, BUT DO NOT WORRY ABOUT MODIFYING THE STACK.

\begin{multicols}{2}
\codelinesinput{../java/Exam.java}{java/Exam.java}{146}{153}
\columnbreak
\begin{blackboard}
$ java Exam
-----------
-- again --
-----------
-----------
3
-- again --
3
-----------
-----------
5
-- again --
-----------
-----------
7
-- again --
-----------
\end{blackboard}
THIS MIGHT BE RECURSIVE, BUT IT IS EMPTYING THE STACK.
\end{multicols}

So, we have a recursive version, the problem is that it is emptying the stack.

The way to solve this problem is the same as before, we need to store the
popped elements into some storage space.

But thanks to recursion we do not need another extra stack to store the
elements, we can use the interval variables defined in the stack frames of each
recursive call to the method.



\newpage
\subsubsection{Non-Destructive Recursive Version}

WRITE A show7 RECURSIVE METHOD THAT PRINTS AN STACK AND LEFT IT UNMODIFIED.

\begin{multicols}{2}
\codelinesinput{../java/Exam.java}{java/Exam.java}{155}{164}
\columnbreak
\begin{blackboard}
$ java Exam
-----------
-- again --
-----------
-----------
3
-- again --
3
-----------
-----------
5
3
-- again --
5
3
-----------
-----------
7
5
3
-- again --
7
5
3
-----------
\end{blackboard}
\end{multicols}









\newpage

\section{2008-2009, Extraordinary call, Problem 2, Section 3}

Let us skip section 2, as it is not very interesting.

GIVE SOME TIME TO READ SECTION 3.

So this problem is about counting leaves in a binary tree.

There are several definitions of a binary tree, the one we are going to use in this
exercise is the \emph{recursive definition with empty trees}:

\begin{blackboard}
A binary tree is

  - empty

  - or a node with two binary trees
\end{blackboard}

The tree looks like this:

\begin{multicols}{2}
  \codelinesinput{../java-simple/Tree1.java}{java-simple/Tree1.java}{1}{53}
\end{multicols}

CHANGES:

\begin{enumerate}

  \item The class in the exam does not explain how to create trees that are not
    empty, and the node constructor is never used for anything.

    This means we will not be able to create any interesting tree with this class.

    To fix this, I have changed 3 things:

    \begin{enumerate}

      \item I have added a new \texttt{Node} constructor that receives the
        element, a left tree and a right tree. This constructor must check for
        null as the trees because the definition of the tree that we are using
        does not alloes for nulls in a node, a node must reference a left tree
        and a right tree.

      \item I have added a new \texttt{Tree} constructor for creating leaves. A
        leaf has empty trees as it sleft and right childs.

      \item I have added a new \texttt{Tree} constructor that merges two trees into a new tree.

        This way, we can create big trees by merging a lot of leaves, which is
        the usual way of creating trees.

    \end{enumerate}

  \item I have added an \texttt{isEmpty} method to the \texttt{Tree} class,
    because it is very useful and very simple.

  \item I have added a \texttt{toString} method to the \texttt{Tree} class,
    because it is very simple and useful for debugging.

    In this case I have chose to return the in-order version of \texttt{toString}.

\end{enumerate}


So now, before trying to solve the exam, let us make the most of this problem
and try to review how to implement binary trees.

\newpage
\subsection{How to create big trees}

First let us try to create a big tree, like this one:

\begin{center}
  \includegraphics[width=10cm]{./img/tree/tree.eps}
\end{center}

We start by creating the lower leaves, and them we combine each leaf with other
leaves to create small trees, then we combine small trees into large trees:

\begin{multicols}{2}
  \codelinesinput{../java-simple/Tree1.java}{java-simple/Tree1.java}{55}{95}
\end{multicols}

Now, what will be printted if I call \texttt{s.o.p(t0)}?
and if I call \texttt{s.o.p(t1)}?

Now, what will be printted if I call \texttt{s.o.p(t0.countLeafs())}?
and if I call \texttt{s.o.p(t1.countLeafs())}?
and if I call \texttt{s.o.p(t7.countLeafs())}?

OK, so now that we know how to create big trees and use them, let us write an
implementation for this.

\newpage
\subsection{Binary Tree Implementation 1}

\begin{multicols}{2}
  \codelinesinput{../java/Tree1.java}{java/Tree1.java}{1}{53}

  1) BEGIN WITH THE CTORS.

  2) THEN isEmpty.

  3) THEN toStringInOrder.
\end{multicols}



\newpage
\subsubsection{The countLeafs Method}

\begin{multicols}{2}
  \codelinesinput{../java/Tree1.java}{java/Tree1.java}{55}{63}
  \columnbreak
\begin{blackboard}
  $ java Tree1
  t0.toStringInOrder() =  3  6  1  4  0  2  7  5
  t1.toStringInOrder() =  3  6  1  4
  t2.toStringInOrder() =  2  7  5
  t3.toStringInOrder() =  3  6
  t4.toStringInOrder() =  4
  t5.toStringInOrder() =  7  5
  t6.toStringInOrder() =  6
  t7.toStringInOrder() =  7
  t0.countLeafs() = 3
  t1.countLeafs() = 2
  t2.countLeafs() = 1
  t3.countLeafs() = 1
  t4.countLeafs() = 1
  t5.countLeafs() = 1
  t6.countLeafs() = 1
  t7.countLeafs() = 1
\end{blackboard}
\end{multicols}

\newpage
\subsubsection{Problems of this implementation}

The main problem with this implementation is that when you merge trees, the
trees survive, so you can merge the same tree in several tress, leading to a
non-tree:

\begin{center}
  \includegraphics[width=10cm]{./img/non-tree/non-tree-3.eps}
\end{center}


\newpage
\subsection{Binary Tree Implementation 2}

One way to avoid this problem is to change the definition of a tree, instead of
empty or a node with two subtress, we can use:

\begin{blackboard}
A binary tree is

  - empty

  - or a node with zero, one or two nodes
\end{blackboard}

The tree will now look like this:

\codelinesinput{../java-simple/Tree2.java}{java-simple/Tree2.java}{1}{43}

\newpage
\begin{multicols}{2}
  \codelinesinput{../java/Tree2.java}{java/Tree2.java}{1}{71}

  1) BEGIN WITH THE CTORS.

  2) THEN isEmpty.

  3) THEN toStringInOrder (BOTH IN tree2 AND IN node).
\end{multicols}




\newpage
\subsection{The countLeafs Method}

\begin{multicols}{2}
  \codelinesinput{../java/Tree2.java}{java/Tree2.java}{73}{79}
  \codelinesinput{../java/Tree2.java}{java/Tree2.java}{23}{31}
  \columnbreak
\begin{blackboard}
  $ java Tree2
  t0.toStringInOrder() =  3  6  1  4  0  2  7  5
  t1.toStringInOrder() =
  t2.toStringInOrder() =
  t3.toStringInOrder() =
  t4.toStringInOrder() =
  t5.toStringInOrder() =
  t6.toStringInOrder() =
  t7.toStringInOrder() =
  t0.countLeafs() = 3
  t1.countLeafs() = 0
  t2.countLeafs() = 0
  t3.countLeafs() = 0
  t4.countLeafs() = 0
  t5.countLeafs() = 0
  t6.countLeafs() = 0
  t7.countLeafs() = 0
\end{blackboard}
\end{multicols}

\newpage
\subsubsection{Problems of this implementation}

This version allows to merge trees without any problems, but its recursive
methods are more complex as you have to constantly check for nulls.

In the previous version a tree was either empty of has exactly 2 childs.

Now, you have 4 possibilities:

\begin{enumerate}

  \item no childs

  \item two children

  \item one child, the left one

  \item one child, the right one

\end{enumerate}

So all methods are mor complex.

A PARTIR DE AQUI, A GUSTO DEL PROFESOR, ESTO SE SALE DEL TEMARIO DE ESTE CURSO.

There is a simple way to fix this, using sentinels like in doubly-linked list.

\begin{multicols}{2}
  \codelinesinput{../java/Tree3.java}{java/Tree3.java}{1}{76}
\end{multicols}

\end{document}
