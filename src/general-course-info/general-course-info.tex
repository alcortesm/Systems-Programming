\documentclass[11pt, a4paper, twoside]{article}

\usepackage[utf8]{inputenc}
\usepackage{hyperref}

\begin{document}

\title{Systems Programming\\General Course Information}
\author{Alberto Cortés}
\date{\today}
\maketitle


\section{Intended Audience}

\emph{Systems Programming} is a second-term course on Computer Science for
first year undergraduate students by the \href{http://it.uc3m.es}{Department of
Telematic Engeneering} at \href{http://uc3m.es}{UC3M}.

%% It is a second term course well suited for students enrolled in
%% \href{http://www.uc3m.es/ss/Satellite/UC3MInstitucional/en/Detalle/Estudio_C/1383578151622/1371206607588/Bachelor_s_Degree_in_Telematics_Engineering}{Bachelor's
%% Degree in Telematics Engineering},
%% \href{http://www.uc3m.es/ss/Satellite/UC3MInstitucional/en/Detalle/Estudio_C/1383577971816/1371206607588/Bachelor_s_Degree_in_Communication_System_Engineering}{Bachelor's
%% Degree in Communication System Engineering},
%% \href{http://www.uc3m.es/ss/Satellite/UC3MInstitucional/en/Detalle/Estudio_C/1383577905316/1371206607588/Bachelor_s_Degree_in_Audiovisual_System_Engineering}{Bachelor's
%% Degree in Audiovisual System Engineering},
%% \href{http://www.uc3m.es/ss/Satellite/UC3MInstitucional/en/Detalle/Estudio_C/1383578083201/1371206607588/Bachelor_s_Degree_in_Telecommunication_Technologies}{Bachelor's
%% Degree in Telecommunication Technologies}

Its only prerequisite is \emph{Programming} (first course, first term).

\section{Course Topics}

The main objective of this course is to further advance in the study of a
programming language and its use for programming small systems. We will be
focusing on:

\begin{itemize}

  \item Object-Oriented Programming

  \item Graphical User Interfaces

  \item Recursion

  \item Basic Data Structures: linked lists, stacks, queues and trees.

\end{itemize}

\section{Teaching Methodology}

This course is divided into units, where each unit consists of:

\begin{itemize}

  \item \textbf{lectures} in large groups: one teacher and about 40 students in
    an ordinary classroom (no computers).

  \item \textbf{lab sessions} in small groups: one teacher and about 20
    students in a computer room.

\end{itemize}

Each week we will be having one lecture and one lab. Please, check the
\href{http://www.it.uc3m.es/alcortes/asig/1415/ps-ging/schedule.pdf}{course
schedule} for further details.

\section{Teaching Staff}

Check the
\href{http://it.uc3m.es/alcortes/asig/1415/ps-ging/index.html}{course home
page}.

\section{Course Materials}

All necessary material will be available from the course home web page in
advance (in PDF or HTML format) except for bibliographic references to required
readings.

For lectures, there will be the \textbf{official course slides}, the lecture
\textbf{handouts} (students should print them in advance), some
\textbf{self-assesement exercises} (to use after the lectures) and some
required \textbf{bibliographic references} (usually chapters from books).

For the labs, there will be \textbf{lab scripts} with the assignments to be
carried out in pairs during the lab sessions.

For more information see the
\href{https://www3.uc3m.es/reina/titulaciones.html}{Ficha Reina}.

\section{Grading}

The student's final grade will be computed as follows:

\begin{itemize}

  \item 25\% midterm exam 1

  \item 25\% midterm exam 2

  \item 50\% final exam

\end{itemize}

A minimum final score of 5.0 out of 10.0 is required to pass this course. Also,
a minimum score of 3.5 points out of 10.0 in the final exam is required to pass
the course with the continuous evaluation system.

Please, note that these three exams are the only events affecting the student
final grade. Class attendance, lab attendance, self-assesment exercises and lab
exercises are not required to pass and they will not affect the student's final
grade. Also there are no projects to submit in this course.

\subsection{Exams}

All exams will be closed-books and closed-electronics. They will be graded from
$0.0$ (worst score) to $10.0$ (best score) or with an ``NP'' (\emph{no
presentado} in Spanish, in case of student absence). Each exam will have two
parts:

\begin{itemize}

  \item a test, worth 30\% of the exam score and based on the official course
    slides

  \item and some problems, worth 70\% of the exam score and based on the lab
    scripts.

\end{itemize}

\section{Attendance and Absences}

Class and lab attendance is not required to pass this course, nor will it be
taken into account for the final grade. Exams attendance is required in order to avoid getting an score of ``NP''.

\section{Regulation, Ethics and Student Behaviour}

In any case, the
\href{http://www.uc3m.es/portal/page/portal/organizacion/secret_general/normativa/estudiantes/estudios_grado/normativa-evaluacion-continua-31-05-11_FINALx.pdf}{regulations
for the evaluation of bachelor studies approved by the University on May 31st,
2011} applies (sorry, only in Spanish).

Along the evaluation process, students are expected to behave according to the
\href{http://www.uc3m.es/portal/page/portal/conocenos/nuestros_estudios/grados/tu_compromiso_universidad}{ethics
of the University} and the
\href{http://www.uc3m.es/ss/Satellite/UC3MInstitucional/en/TextoMixta/1371206782958/Guia_de_las_buenas_practicas}{Good
Practices Guide}.

In particular, students are expected to fulfil the evaluation norms and to
reject fraudulent behaviours, such as plagiarism or any other kind of cheating.
Likewise, the student is responsible for guarding their evaluation assets in
order to avoid such fraudulent actions from other peers.

Any behaviour breaking these rules will be penalised and brought to the
attention of the relevant bodies, and they should take appropriate action
in accordance with current regulations. Where it is established that a student
has committed a fraud situation that impedes the exercise by the teachers, from
the power of knowledge verification, the calification will be fail (0.0).


\end{document}
