\documentclass[11pt, a4paper, twoside]{article}

\usepackage[utf8]{inputenc}
\usepackage{hyperref}
\usepackage{amsmath}
\usepackage{nicefrac}

\begin{document}

\title{Systems Programming\\General Course Information}
\author{Alberto Cortés}
\date{\today}
\maketitle


\section{Intended Audience}

\emph{Systems Programming} is a second-term course on Computer Science for
first year undergraduate students by the \href{http://it.uc3m.es}{Department of
Telematic Engineering} at \href{http://uc3m.es}{UC3M}.

%% It is a second term course well suited for students enrolled in
%% \href{http://www.uc3m.es/ss/Satellite/UC3MInstitucional/en/Detalle/Estudio_C/1383578151622/1371206607588/Bachelor_s_Degree_in_Telematics_Engineering}{Bachelor's
%% Degree in Telematics Engineering},
%% \href{http://www.uc3m.es/ss/Satellite/UC3MInstitucional/en/Detalle/Estudio_C/1383577971816/1371206607588/Bachelor_s_Degree_in_Communication_System_Engineering}{Bachelor's
%% Degree in Communication System Engineering},
%% \href{http://www.uc3m.es/ss/Satellite/UC3MInstitucional/en/Detalle/Estudio_C/1383577905316/1371206607588/Bachelor_s_Degree_in_Audiovisual_System_Engineering}{Bachelor's
%% Degree in Audiovisual System Engineering},
%% \href{http://www.uc3m.es/ss/Satellite/UC3MInstitucional/en/Detalle/Estudio_C/1383578083201/1371206607588/Bachelor_s_Degree_in_Telecommunication_Technologies}{Bachelor's
%% Degree in Telecommunication Technologies}

Its only prerequisite is \emph{Programming} (first course, first term).

\section{Course Topics}

The main objective of this course is to further advance in the study of a
programming language and its use for programming small systems. We will be
focusing on:

\begin{itemize}

  \item Object-Oriented Programming

  \item Graphical User Interfaces

  \item Recursion

  \item Basic Data Structures: linked lists, stacks, queues and trees.

\end{itemize}

\section{Teaching Methodology}

This course is divided into sessions, where each session is either:

\begin{itemize}

  \item A \textbf{lecture}: 1 teacher and about 40 students in an ordinary
    classroom (no computers),

  \item or a \textbf{lab session}: 1 teacher and about 20 students in a
    computer room.

\end{itemize}

Each week we will be having one lecture and one lab. Please, check the
\href{http://www.it.uc3m.es/alcortes/asig/1415/ps-ging/schedule.pdf}{course
schedule} for further details.

\section{Teaching Staff}

Check the
\href{http://it.uc3m.es/alcortes/asig/1415/ps-ging/index.html}{course home
page}.

\section{Course Materials}

All necessary material will be available from the course home web page in
advance (in PDF or HTML format) except for bibliographic references to required
readings.

For lectures, there will be the \textbf{official course slides}, the lecture
\textbf{handouts} (students should print them in advance), some
\textbf{self-assessment exercises} (to use after the lectures) and some
required \textbf{bibliographic references} (usually chapters from books).

For the labs, there will be \textbf{lab scripts}. Students will working in
pairs during the lab session to solve them. Solutions to the labs scripts will
be published one week after each lab.

There will not be official solutions to the self-assessment exercises. Instead,
there is a \textbf{\href{http://it.uc3m.es}{course forum}} where students can
post questions about the self-assessment exercises and discuss about them.

This course forum is \emph{for students by students}. Teachers will not answer
direct questions there or correct wrong answers from other students, although
teachers will post a few lines from time to time just to spice up discussions
or to make class-wide announcements.


\section{Grading}

The student's overall grade is a number between 0.0 (worst) and 10.0 (best),
and will be computed as follows:

\begin{itemize}

  \item 25\% midterm exam 1

  \item 25\% midterm exam 2

  \item 50\% final exam.

\end{itemize}

A minimum overall score of 5.0 is required to pass this course. In addition,
a minimum score of 3.5 in the final exam is required to pass
the course with the continuous evaluation system.

A student can also be graded ``NP'' (\emph{no presentado} in Spanish) in case
of student absence to the final exam.

Here are some grading examples:

\begin{center}
  \begin{tabular}{|c|c|c||c|l|}
    \hline
    \multicolumn{3}{|c||}{Exams grades} & \multicolumn{2}{|c|}{Overall grade} \\
    \hline
    Midterm 1 & Midterm 2 & Final & Overall & Comments\\
    \hline
    $x$  & $y$  & NP   & NP   & \\
    NP   & NP   & $z$  & $\nicefrac{z}{2}$  & pass $\iff z >= 9.9$ (round-off error)\\
    \hline
    0.0  & 0.0  & 0.0  & 0.0  & fail \\
    5.0  & 5.0  & 5.0  & 5.0  & pass \\
    10.0 & 10.0 & 10.0 & 10.0 & pass \\
    \hline
    0.0  & 0.0  & $z$ & $\nicefrac{z}{2}$ & pass $\iff z >= 9.9$ (round-off error)\\
    \hline
    7.2  & 8.5  & 3.5  & 5.7  & pass \\
    7.2  & 8.5  & 3.4  & 3.4  & fail (final exam $< 3.5$) \\
    \hline
    NP   & 10.0 & 5.0  & 5.0  & pass (NP $\rightarrow$ 0.0) \\
    NP   & 9.7  & 5.0  & 5.0  & pass (NP $\rightarrow$ 0.0, round-off error) \\
    NP   & 9.6  & 5.0  & 4.9  & fail (NP $\rightarrow$ 0.0) \\
    \hline
  \end{tabular}
\end{center}

Please, note that these three exams are the only events affecting a student's
overall grade. Class and lab attendance or answers to exercises are not
required to pass and they will not affect the student's overall grade.  In
addtition, there are no projects to submit in this course.

\subsection{Exams}

All exams will be closed books and closed-electronics. Each exam will have two
parts:

\begin{itemize}

  \item a test, worth 30\% of the exam score and based on the official course
    slides,

  \item and some problems, worth 70\% of the exam score and based on the lab
    scripts.

\end{itemize}

\section{Attendance and Absences}

Class and lab attendance is not required to pass this course, nor will it be
taken into account for the overall grade. Exams attendance is required in order
to avoid getting a score of ``NP'' in that exam and attendance to the final
exam is required to avoid getting an overall score of ``NP.''

\section{Regulation, Ethics and Student Behavior}

In any case, the
\href{http://www.uc3m.es/portal/page/portal/organizacion/secret_general/normativa/estudiantes/estudios_grado/normativa-evaluacion-continua-31-05-11_FINALx.pdf}{regulation
  for the evaluation of bachelor studies approved by the University on May 31,
2011} applies (sorry, only in Spanish).

Along the evaluation process, students are expected to behave according to the
\href{http://www.uc3m.es/portal/page/portal/conocenos/nuestros_estudios/grados/tu_compromiso_universidad}{ethics
of the University} and the
\href{http://www.uc3m.es/ss/Satellite/UC3MInstitucional/en/TextoMixta/1371206782958/Guia_de_las_buenas_practicas}{Good
Practices Guide}.

In particular, students are expected to fulfil the evaluation norms and to
reject fraudulent behaviors, such as plagiarism or any other kind of cheating.
Likewise, the student is responsible for guarding their evaluation assets in
order to avoid such fraudulent actions from other peers.

Any behavior breaking these rules will be penalized and brought to the
attention of the relevant bodies, and they should take appropriate action in
accordance with current regulations. Where it is established that a student has
committed a fraud situation that impedes the exercise by the teachers, from the
power of knowledge verification, the grade will be \emph{fail} (0.0).


\end{document}
